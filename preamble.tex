\usepackage{amsmath,amssymb,amsfonts, mathtools}
\usepackage{tikz}
\usepackage [utf8] {inputenc}
\usepackage{cmap}
\usepackage [T2A] {fontenc} 
\usepackage[english, russian]{babel}
\usepackage{geometry}
\geometry{left=20mm,right=20mm,top=25mm,bottom=20mm} % задание полей текста
\usepackage[colorlinks=true,linkcolor=red]{hyperref}
%===============================================================================

% Работа с переносами
\binoppenalty=10000
\relpenalty=10000
\sloppy
%===============================================================================

\usepackage{graphicx}
\usepackage{wrapfig}


%===============================================================================

% Красивые страницы 

%===============================================================================

\usepackage{fancyhdr}
\newcommand{\changefont}{%
	\fontsize{9}{11}\selectfont
}
\fancyhf{}
\pagestyle{fancy}
% этим мы убеждаемся, что заголовки глав и 
% разделов используют нижний регистр.
\renewcommand{\sectionmark}[1]{\markright{\thesection\ #1}}
\fancyhf{}  % убираем текущие установки для колонтитулов
\fancyhead[RE, RO]{\changefont\thepage}
\fancyhead[LO,LE]{\changefont\rightmark}
\renewcommand{\headrulewidth}{0.5pt}
\renewcommand{\footrulewidth}{0pt}
\addtolength{\headheight}{0.5pt} % оставляем место для линейки
\fancypagestyle{plain}



\usepackage{quiver} 
\usetikzlibrary{babel}
%===============================================================================

% Настройка нумерации

%===============================================================================



\addto\captionsrussian{% Replace "english" with the language you use
	\renewcommand{\contentsname}%
	{Содержание}%
}

%===============================================================================

% Theorem styles

%===============================================================================
\usepackage{amsthm}
\theoremstyle{definition}
\usepackage{ mathrsfs }
 \usepackage{stackrel} 
% ----------- Math short-cats
\newcommand{\bb}{\mathbb}
\newcommand{\R}{\ensuremath{\mathbb{R}}}
\newcommand{\N}{\ensuremath{\mathbb{N}}}
\newcommand{\Cx}{\ensuremath{\mathbb{C}}}
\newcommand{\Z}{\ensuremath{\mathbb{Z}}}
\newcommand{\E}{\ensuremath{\mathbb{E}}}
\newcommand{\Q}{\ensuremath{\mathbb{Q}}}
\def\L{\mathbb{L}}
\def\CB{\mathcal{B}}
\def\CC{\mathcal{C}}
\def\CE{\mathcal{E}}
\def\CR{\mathcal{R}}
\def\CA{\mathcal{A}}
% \def\CF{\mathcal{F}}
\def\CG{\mathcal{G}}
\def\CS{\mathcal{S}}
\def\CD{\mathcal{D}}
\def\CH{\mathcal{H}}
\def\CP{\mathcal{P}}
\def\CM{\mathcal{M}}
\def\CL{\mathcal{L}}
\def\CK{\mathcal{K}}
\def\FB{\mathfrak{B}}
\def\deff{\stackrel{def}\Leftrightarrow}
\def\rrr{\longrightarrow}
\def\rr{\rightarrow}
\def\eps{\varepsilon}
\def\bs{\backslash}
% You can write your commands below
\usepackage{gensymb}
\usepackage{enumitem}
\usepackage{amsmath}
\newcommand{\F}{\ensuremath{\mathcal{F} }}
\newcommand{\Anu}{\ensuremath{\mathcal{A}_{\nu}}}
\DeclareMathOperator{\FDU}{FDU}
\DeclareMathOperator{\supp}{supp}
\DeclareMathOperator{\Lin}{Lin}
\DeclareMathOperator{\Rea}{Re}
\DeclareMathOperator{\Ima}{Im}
\usepackage{ulem}
\DeclareMathOperator{\Int}{int}
\DeclareMathOperator{\Ker}{Ker}
\DeclareMathOperator{\Gr}{Gr}
\DeclareMathOperator{\conv}{conv}
% ----------- Math and theorems -----------
\usepackage[many]{tcolorbox}
\usepackage{mdframed}

\renewcommand{\thesection}{\arabic{section}}

\newtheorem*{remark}     {Замечание}
\newtheorem*{next0}      {Следствие}
\newtheorem*{next1}      {Следствие 1}
\newtheorem*{next2}      {Следствие 2}

\newtheorem{theorem}{Теорема}[section]
\newtheorem{lemma}[theorem]{Лемма}
\newtheorem{definition}[theorem]{Определение}
\newtheorem{claim}[theorem]{Утверждение}
\newtheorem{example}[theorem]{Пример}
\renewcommand\qedsymbol{$\blacksquare$}



\newcommand{\lab}[1]{(\ref{#1})}


\usepackage{hyperref}

% Обводка кружочком множеств
\usepackage{tikz}
\usetikzlibrary{decorations.pathreplacing,shapes.misc}
\newcommand*\circled[1]{\tikz[baseline=(char.base)]{
		\node[shape=circle,draw,inner sep=2pt] (char) {#1};}}
