\newpage
\section{Слабая* топология в $X^*$ для ТВП. Теорема Шмульяна и Банаха-Алаоглу}
По определению $X^*$ --- пространство непрерывных линейных функционалов в $\Cx$. Поэтому можно погрузить его в пространство всех функций $\Cx^X = \{g \colon X \to \Cx\}$. В пространстве функций можно ввести топологию Тихонова и тогда 
\begin{definition}
	Слабой* топологией $\tau_{w^*}$ в пространстве $X^*$ сопряженном топологическому векторному пространству $(X,\tau)$ называется индуцированная с пространства $\Cx^X$ топология Тихонова. 
\end{definition}
Тогда ясно, что предбаза $\tau_{w^*}$:
$$
\sigma_{w^*} = \{V_*(f,x,\eps) \mid f \in X^*, x\in X, \eps > 0\}, \quad V_*(f,x,\eps) = \pi_x^{-1}(B_\eps(f(x)) \cap X^*)
$$
Множество $V_x(f,x,\eps)$ можно так же представить как $V_*(f,x,\eps) = \{g \in X^* \mid |g(x) -f(x)| < \eps\}$. Топология $\tau_{w^*}$ векторная, доказательство этого является очень простым и при этом техническим фактом, которое мне лень полностью техать. Кроме того, ясно что $\tau_{w^*}$ имеет выпуклую локальную базу
$$
\beta_{0w^*} = \left\{\bigcap_{k=1}^N V_*(0,x_k, \eps) \mid x_1,\dots, x_N \in X, \eps > 0\right\}
$$ Значит пространство является локально выпуклым. Следующая теорема проясняет структуру сопряженного к этом пространству.
\begin{theorem}[Шмульян]\label{th:shulian}
	Пусть $(X,\tau)$ --- топологическое векторное и $\Phi \in (X^*, \tau_{w^*})^*$, тогда $\exists x \in X \colon\forall f \in X^* \Rightarrow   \Phi(f) = f(x)$. 
\end{theorem}
\begin{proof}
	В силу линейности $\Phi(0) = 0$. Тогда в силу векторности топологии 
	$$
	\exists U_*(0) \in \tau_{w^*}\colon \forall f \in U_*(0) \Rightarrow |\Phi(f)| < 1
	$$
	Эта окрестность содержит элемент базы
	$$
	U_*(0) \supset \bigcap_{k = 1}^N\{V_*(0, x_k, \eps)\}
	$$
	Тогда если мы рассмотрим произвольный $f \in X^*$ такой что $f(x_1) = \dots = f(x_N) = 0$. То
	$$
	\forall n \in \N \Rightarrow nf \in \bigcap_{k = 1}^N\{V_*(0, x_k, \eps)\} \subset U_*(0)
	$$
	Но тогда 
	$$
	|\Phi(nf)| < 1 \Rightarrow |\Phi(f)| < \frac{1}{n} \forall n \in \N
	$$
	Следовательно $\Phi(f) = 0$. Значит мы показали, что 
	$$
	\bigcap_{k=1}^N \Ker F_{x_k} \subset \Ker \Phi
	$$
	Где $F_{x_k}$ --- функционал порожденный каноническим вложением $x_k$ в $(X^*,\tau_{w^*})^*$. Теперь рассмотрим
	$$
	M = \left\{\begin{pmatrix}
		f(x_1) \\
		\vdots \\
		f(x_N)
	\end{pmatrix} \mid f \in X^*\right\} \subset \Cx^N
	$$
	$M$ --- конечномерное подпространство в $\Cx^N$. Заведем на нем функционал $\Lambda \colon M \to \Cx$:
	$$
\forall f \in X^* \quad	\Lambda(f(x_1), \dots, f(x_N)) = \Phi(f)
	$$
	Проверим что $\Lambda$ определен корректно, действительно, пусть $f(x_k) = g(x_k)$, тогда $f - g \in \bigcap_{k=1}^N \Ker F_{x_k} \subset \Ker \Phi$, тогда 
	$$
	\Phi(f -g)  = 0 \Leftrightarrow \Phi(f) = \Phi(g)
	$$
	Таким образом $\Lambda$ --- функционал над конечномерным пространством, тогда из линейной алгебры известно, что 
	$$
	\exists a_1, \dots, a_N \in \Cx\colon \Phi(f) = \Lambda(f(x_1), \dots, f(x_N)) = \sum_{k=1}^N a_k f(x_k) = f\left(\sum_{k=1}^N a_k x_k\right)
	$$
	Таким образом искомый $x = \sum_{k=1}^N a_k x_k$. Что и требовалось.
\end{proof}
\begin{remark}
	Если исходное пространство $(X,\tau)$ является локально выпуклым, то по следствию теоремы Хана-Банаха, все точки отделяются функционалами, и $x$ из теоремы выше будет единственным.
\end{remark}

\begin{theorem}[Банах-Алаоглу]\label{th:banach-alaoglu}
	Пусть $(X,\tau)$ --- топологическое векторное пространство и $ V\in \tau$ --- окрестность нуля. Рассмотрим
	$$
	\Gamma(V) = \{ f\in X^* \mid |f(x)| \leq 1  \ \forall x \in V\}
	$$
	$\Gamma(V)$ --- называется поляром окрестности. Тогда $\Gamma(V)$ --- $\tau_{w^*}$-компакт.
\end{theorem}
\begin{proof}
	Рассмотрим произвольны $x \in X$, $x\cdot 0 = 0 \in V$, тогда 
	$$
	\exists \delta_x > 0 \ \forall \lambda \in \Cx : \ |\lambda| < \delta_x \Rightarrow \lambda x \in V
	$$
	Возьмем $t_x = \frac{\delta_x}{2}$, тогда 
	$$
	t_x \cdot x \in V \Rightarrow \forall f \in \Gamma(V) \Rightarrow |f(t_x x)| \leq 1 \Rightarrow |f(x)| \leq \frac{1}{t_x}
	$$
	Обозначим $r_x = \frac{1}{t_x}$. Таким образом любой функционал на конкретном $x$ ограничен по модулю числом $r_x$. Теперь рассмотрим замкнутые круги в $\Cx$:
	$$
	K_r = \{\lambda \in \Cx \mid |\lambda| \leq r\}
	$$
	Для каждого $x$ рассмотрим такой круг и построим декартово произведение
	$$
	\bigtimes_{x\in X} K_{r_x} = \{g \in X \to \Cx \mid \forall x \in X\colon  g(x) \in K_{r_x}\}
	$$
	Тогда в силу ограничений выше получаем 
	$$
	\Gamma(V) \subset 	\bigtimes_{x\in X} K_{r_x} 
	$$
	Наделим это декартово произведение топологией Тихонова. Так как для каждого $x$, $K_{r_x}$ является компактом, то по теореме Тихонова $	\bigtimes_{x\in X} K_{r_x} $ --- компакт в топологии Тихонова, но сужение топологии Тихонова на непрерывные функции и есть $\tau_{w^*}$, тогда $	\bigtimes_{x\in X} K_{r_x}  \cap X^*$ является $\tau_{w^*}$-компактом, при этом $\Gamma(V)$ является его подмножеством, значит нам остается показать, что $\Gamma(V)$ --- замкнуто. 
	
	Пусть $ g \in [\Gamma(V)]_{\tau_T} \subset \Cx^X$. Берем 
	$$
	x_{1,2} \in X,  \quad V_T(g,x,\eps) = \{h \in \Cx^X \mid |g(x) - h(x)| < \eps\}
	$$
	Тогда рассмотрим 
	$$
	U(g) = V_T(g, x_1 + x_2, \eps) \cap V_T(g,x,\eps) \cap V_T(g,x_2,\eps) \in \tau_T
	$$
	Тогда 
	$$
	\exists f \in U(g) \cap \Gamma(V)
	$$
	Такой что 
	$$
	|f(x_1 + x_2) - g(x_1 + x_2)| = |f(x_1) + f(x_2) - g(x_1 + x_2)| < \eps, \quad|g(x_{1,2} - f(x_{1,2})| < \eps
	$$
	Тогда пользуясь умным нулем получаем
	$$
	|g(x_1) + g(x_2) - g(x_1 + x_2)| < 3\eps
	$$
	Устремляя $\eps$ к нулю, получаем, что $g$ --- аддитивен. 
	
	Теперь пусть $x \in X, \lambda \in \Cx$, теперь возьмем окрестность 
	$$
	U(g) = V_T(g,\lambda x, \eps) \cap V_T(g,x,\eps)
	$$
	Аналогично аддитивности получаем $g(\lambda x) = \lambda g(x)$. 
	
	Теперь $g$ --- линеен, осталось показать непрерывность. Но $\forall x$ $g(x) \in K_{r_x}$, тогда 
	$$
	\forall x \in V \Rightarrow t_x = r_x = 1 \Rightarrow |g(x)| \leq 1
	$$
	Значит линейный функционал ограничен на окрестности нуля, что равносильно его непрерывности, значит $g \in X^*$, тогда $g \in \Gamma(V)$, то есть 
	$$
	[\Gamma(V)] = \Gamma(V)
	$$
	Таким образом замкнутость доказана, значит $\Gamma(V)$ --- компакт, что и требовалось.
\end{proof} 
\begin{remark}
	В случае нормированного пространства $\Gamma(O_1(0))$ является единичным шаром в сопряженном пространстве. 
\end{remark}