\newpage
\section{Больше теорем Фредгольма}
\begin{theorem}(Фредгольм)
	Пусть $X,Y$ --- ЛНП, $A \in \CL(X, Y)$, тогда $ A $--- компактный оператор если только если $A^*$ --- компактный оператор. 
\end{theorem}
\begin{proof}
	\hfill
	\begin{enumerate}
		\item[$\Rightarrow$] Пусть $A$ --- компактный оператор, то есть $AB_1^X(0)$ --- вполне ограничено в $Y$. $Y$ --- неполон, поэтому рассмотрим пополнение. Пусть $Z$--- банахово, $$Z = [Y]_{\|\|}, \ \|\|_Z\big |_{Y} = \|\|_Y$$
		Тогда $AB_1^X(0)$ --- вполне ограничено в банаховом $Z$ и значит $[AB_1^X(0)]_Z$ --- компакт в $Z$. Обозначим $K =[AB_1^X(0)]_Z$. Нам нужно доказать, что образ шара из сопряженного пространства $Y^*$ под действием $A^*$ вполне ограничен в $X^*$. Мы знаем, что
		$
		A^*(B_1^{Y^*}(0))$ --- вполне ограничен тогда и только тогда, когда $\forall \{g_n\}_{n=1}^\infty \subset A^*(B_1^{Y^*}(0)) \ \exists n_1, n_2, \dots\colon \{g_{n_k}\}_{k=1}^\infty$ --- фундаментальна в $X^*$. $g_n\colon Y \to \Cx$ --- липшицевы функции с константой Липшица 1. Пусть $g \in Y^*$ посмотрим на 
		$$
		g \colon AB_1^X(0) \to \Cx
		$$
		Тогда $\forall z \in [A^*B_1^X(0)]_Z$ мы можем сказать, что $\exists y_n \in A^*B_1^X(0)\colon y_n \to z$, тогда
		$$
		|g(y_n) - g(y_m)| \leq \|g\| \|y_m - y_n\| \to 0
		$$
		Выражение выше означает, что последовательность образов имеет предел. Будет ли предел зависеть от выбора последовательности $\{y_n\}$. Конечно, нет пусть $\tilde{y_n} \in AB_1^X(0)$, $\tilde{y_n} \to z$, тогда
		$$
		|g(y_n) - g(\tilde{y_n})| \leq \|g\| \|y_n - \tilde{y_n}\|_Y \to 0 
		$$
		Таким образом предел не зависит от выбора последовательности, а зависит только от $z$. То есть мы построили функционал 
		$$
		h\colon K \to Z \quad h (z) = \lim\limits_{n \to \infty}(y_n) = \lim\limits_{n \to \infty}(\tilde{y_n})
		$$
		Причем $\displaystyle h \big |_{AB_1^X(0)} = g$. $g$ является непрерывным линейным функционалом на шаре, а нам он нужен на компакте, который мы получили, замкнув шар в пополнении. Теперь наш $h$ действует из компакта. Липшицевость $g$ дает нам липшицевость $h$:
		$$
		z_{1,2} \in K \Rightarrow |h(z_1)- h(z_2)| = \lim\limits_{n \to \infty}|g(y'_n) - g(y''_n)| \leq \lim\limits_{n \to \infty} \|g\| \|y'_n - y''_n\| = \|g\|\|z_1-z_2\|
		$$
		Теперь для каждого $g_n$ исходной последовательности проделаем данную процедуру продолжения. Получим $\{h_n\}\subset K$. Все $h_n$ --- липшецевы с константой 1 (так как исходные $g_n$ живут на сфере). Таким образом $h_n$ --- непрерывные функции на компакте, значит мы попали в пространство $C(K)$.
		В пространстве $C(K)$ норма супремальная, а не операторная. Найдем ее для $h_n$:
		$$
		\|h_n\|_c = \max_{z \in K} |h_n(z)| = \sup_{y \in AB_1^X(0)}|g_n(y)| = \sup_{x \in B_1^X(0)}|g_n(Ax)| = \sup_{x \in B_1^X(0)}|A^*g_n(x)|  = \|A^*(g_n)\|
		$$
		Получилось, что супремальная норма $h_n$ в $C(K)$ совпадает с операторной нормой образов $g_n$ под действием сопряженного оператора $A^*$. Если мы теперь рассмотрим норму разности, то 
		$$
		\|h_n - h_m\|_c = \|A^*g_n - A^*g_m\|
		$$
		Таким образом вопрос о выделении фундаментальной подпоследовательности в образе $A^*B_1^{Y^*}(0)$ (это то, что нам надо) сводится к выделению фундаментальной подпоследовательности в $\{h_n\}$. А для этого  нам известен хороший критерий вполне ограниченности в $C(K)$, а именно теорема Арцела-Асколе (\ref{th:arzela}). 
		$$
		S = \{h_n\}_{n=1}^\infty \subset C(K)
		$$
		В силу теоремы Арцела-Асколе $S$ --- вполне ограниченно если и только если
		$$
		\begin{cases}
			\exists R>0 \ \forall n \in \N \Rightarrow \|h_n\|_c \leq R \\
			\forall \eps > 0 \exists \delta(\eps)\colon \forall n \in \N \forall z_{1,2} \in K \colon \|z_1 - z_2\| \leq \delta \Rightarrow |h_n(z_1) - h_n(z_2)| < \eps
		\end{cases}
		$$
		Эти условия выполнены с очевидностью. Ограниченность:
		$$
		\|h_n\|_c = \|A^*g_n\| \leq \|A^*g_n\|\|g_n\| \leq \|A^*\| \Rightarrow R = \|A^*\|
		$$
		Равностепенная непрерывность:
		$$
		\delta(\eps) = \eps \Rightarrow |h_n(z_1) - h_n(z_2)| \leq \|h_n\|\|z_1-z_2\| \leq \delta = \eps
		$$
		Таким образом $S$ --- вполне ограниченно в $C(K)$ и значит существует подпоследовательность $\{h_{n_k}\}_{k=1}^\infty$ --- фундаментальная в $C(K)$. Но тогда 
		$$
		\|A^*g_{n_k} - A^*g_{n_m}\| = \|h_{n_k} - h_{n_m}\| \to 0 
		$$
		Это и означает вполне ограниченность образа шара под действием $A^*$ 
		\item[$\Rightarrow$] Пусть теперь $A^*\colon Y^* \to X^*$ --- компактный оператор. Мы только что доказали, что это означает компактность сопряженного к $A^*$, то есть 
		$A^{**}\colon X^{**} \to Y^{**}$ --- компактный оператор. Воспользуемся изометрией банаха. Пусть
		$$
		\Phi\colon X \to X^{**} \quad \Psi \colon Y \to Y^{**}
		$$
		Изометрии Банаха, то есть $\Phi x(f) = f(x), \Psi y (g) = g(y)$, $\|\Phi x\| = \|x\|, \|\Psi y \| = \|y\|$. Посмотрим на картинку. 
		% https://q.uiver.app/?q=WzAsNCxbMCwwLCJYIl0sWzEsMCwiWF57Kip9Il0sWzAsMSwiWSJdLFsxLDEsIlleeyoqfSJdLFswLDEsIlxcUGhpIl0sWzIsMywiXFxQc2kiXSxbMCwyLCJBIl0sWzMsMSwiQV57Kip9IiwyLHsic3R5bGUiOnsidGFpbCI6eyJuYW1lIjoiYXJyb3doZWFkIn0sImhlYWQiOnsibmFtZSI6Im5vbmUifX19XV0=
		\[\begin{tikzcd}[ampersand replacement=\&]
			X \& {X^{**}} \\
			Y \& {Y^{**}}
			\arrow["\Phi", from=1-1, to=1-2]
			\arrow["\Psi", from=2-1, to=2-2]
			\arrow["A", from=1-1, to=2-1]
			\arrow["{A^{**}}"', tail reversed, no head, from=2-2, to=1-2]
		\end{tikzcd}\]
	В силу \sout{коммутативности данной диаграммы} следующей выкладки 
	$$
	\forall g\in Y^*\colon	(A^{**} \Phi x)(g) = (\Phi x)(A^* g) = (A^*g)(x) = g(Ax) =( \Psi Ax)(g)
	$$
	 Имеем $A = \Phi A^{**} \Psi^{-1}$. Но $\Phi, \Psi$ --- изометрии, значит, так как $A^{*}$ --- компактный оператор, то $A$ --- компактный оператор, что и требовалось.
	\end{enumerate}
\end{proof}