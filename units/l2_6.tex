\newpage
\section{Свойства сопряженного оператора, теоремы Фредгольма}
Пусть $X,Y$ --- ЛНП. $A \in \CL(X,Y)$. Установим связь $\Ker A$, $\Ima A$ и $\Ker A^*$, $\Ima A^*$. 
\begin{definition}
	Пусть $S \subset X$, тогда правым аннулятором множества $S$ называется 
	$$
	S^\perp = \{f \in X^* \mid \forall x \in S: f(x) = 0\}
	$$
\end{definition}
Очевидно, что $S^\perp \subset X^*$ --- подпространство. 
\begin{claim}
	$S^\perp$ --- замкнуто в $X^*$ по норме. 
\end{claim}
\begin{proof}
	Пусть $f_n \subset S^\perp\colon f_n \xrightarrow{\|\|}f \in X^*$. Тогда так как 
	$$
	|f_n(x) - f(x)| \leq \|f_n - f\| \|x\| \to 0
	$$
	То $f(x) = \lim\limits_{n \to \infty}f(x) = 0 \Rightarrow f \in S^\perp$
\end{proof}
\begin{definition}
	Пусть $X^*$ --- сопряженное к $X$ --- ЛНП. Тогда слабой* топологией в $X$ называется
	$
	\tau_{W^*}
	$ индуцированная из $\Cx^X$ с топологией Тихонова. Ее предбаза
	$$
	\sigma_{W^*} = \{V^*(f,x,\eps) \mid f \in X^*, x \in X, \eps> 0\} \quad V^*(f,x,\eps) = \{g \in X^* \mid |g(x) - f(x)| < \eps\}
	$$ 
\end{definition}
\begin{claim}\label{claim:twcl}
	$S^\perp$ --- $\tau_{W^*}$-замкнуто в $X^*$
\end{claim}
\begin{proof}
	Пусть $g \in [S^\perp]_{\tau_{W^*}}$, тогда построим любая окрестность $g$ пересекается с $S^\perp$ по непустому множеству. Это верно и для элементов предбазы, то есть
	$$
	\forall \eps > 0 \ \forall x \in S \Rightarrow V^*(g,x,\eps) \cap S^\perp \neq \varnothing
	$$
	Рассмотрим $f \in V^*(g,x,\eps) \cap S^\perp$, имеем
	$$
	|f(x) - g(x)| = |g(x)| < \eps 
	$$
	Так как это верно для любого $\eps$, то $g(x) = 0$, значит $g \in S^\perp$, что и требовалось.
\end{proof}
\begin{definition}
	Пусть $S \subset X^*$, тогда левым аннулятором множества $S$ называется
	$$
	^\perp S = \{x \in X \mid \forall f \in S\colon f(x) = 0\}
	$$
\end{definition}
Левый аннулятор является подпространством в $X$. 
\begin{claim}
	$^\perp S $ --- замкнуто в $X$ относительно нормы. 
\end{claim}
\begin{definition}
	Можно поступить как с правым аннулятором, а можно записать 
	$$
	^\perp S = \bigcap_{f \in S}\Ker f 
	$$
	Ядра функционалов замкнуты как прообразы $\{0\}$, a пересечение замкнутых множеств замкнуто.
\end{definition}
\begin{remark}
	На самом деле $^\perp S$ будет замкнуто относительно слабой топологии.
\end{remark}
\begin{theorem}[Фредгольм]\label{th:fr}
	Если $A \in \CL(X,Y)$, $X,Y$ --- ЛНП. Тогда
	$$
	\begin{cases}
		\Ker A = \prescript{\perp}{}{(\Ima A^*)} \\   
		\Ker A^* = (\Ima A)^\perp
	\end{cases}
	$$
\end{theorem}
\begin{proof}
 \begin{itemize}
 	\item Пусть $ x \in \Ker A$, это равносильно $Ax = 0 \in Y$, по следствию теоремы Хана-Банаха это равносильно 
 $$
 \forall g \in Y^* \Rightarrow g(Ax) = 0
 $$
 По определению сопряженного оператора получаем 
 $$\forall g \in Y^* \colon (A^*g)(x) = 0$$
 То есть $x \in \prescript{\perp}{}{(\Ima A^*)} $
 \item Пусть теперь $g \in \Ker A^*$ аналогичная цепочка равносильностей, только теперь вместо теоремы Хана-Банаха используем определение нулевого оператора
 $$
 A^*g = 0 \in X^* \Leftrightarrow \forall x \in X\colon (A^*g)(x) = 0 \Leftrightarrow \forall x \in X\colon g(Ax) = 0
 $$
 По определению последнее равносильно $g \in (\Ima A)^\perp$
 \end{itemize}
\end{proof}
\begin{lemma}\label{lem:densyty}
	\hfill
\begin{enumerate}
	\item[a)] 	Пусть $S$ --- всюду плотно в $X$, тогда $S^\perp = \{0\}$
	\item[б)] Пусть $M$ --- всюду плотно в $X^*$, тогда $^\perp M  = \{0\}$
\end{enumerate}
\end{lemma}
\begin{proof}
	\begin{enumerate}
		\item[a)] Так как $S$ --- всюду плотно, то
		$$
		\forall x \in X \colon \exists x_n \in S\colon x_n \xrightarrow{\|\|} x \Rightarrow \forall f \in S^\perp \Rightarrow 0  = f(x_n) \to f(x) \Rightarrow f(x) = 0
		$$
		Из произвольности $x$ следует  $f = 0 $. 
		\item[б)] $M$ --- всюду плотно в $X^*$, тогда
		$$
		\forall g \in X^*: \exists g_n \in M \colon g_n \xrightarrow{\|\|} g \Rightarrow \forall x \in {}^\perp M \Rightarrow 0  = g_n(x) \to g(x) \Rightarrow g(x) = 0
		$$
		По следствию теоремы Хана-Банаха $x = 0$
	\end{enumerate}
\end{proof}
Теперь из теоремы Френ можно вывести следующее следствие.
\begin{next1}\label{n:fr1}
	\hfill
	\begin{itemize}
		\item Если $\Ima A$ --- всюду плотен в $Y$, то $ \Ker A^* = \{0\}$
\item 	Если $\Ima A^*$ --- всюду плотен в $X^*$, то $\Ker A = \{0\}$
	\end{itemize}
\end{next1}
\begin{next2}
	$({}^\perp(\Ima A^*))^\perp = (\Ker A)^\perp$, ${}^\perp((\Ima A)^\perp) = {}^\perp(\Ker A^*)$
\end{next2}
Но что такое двойной аннулятор?
\begin{lemma}\label{lem:doubleann1}
	Пусть $L \subset X$ --- подпространство. Тогда
	$$
	{}^\perp(L^\perp) = [L]_{\|\|}
	$$
\end{lemma}
\begin{proof}
	Совершенно ясно, что $L \subset {}^\perp(L^\perp)$. При этом ${}^\perp(L^\perp)$--- замкнуто в $X$ относительно нормы, тогда 
	$$
	[L]_{\|\|} \subset  {}^\perp(L^\perp)
	$$
	Предположим, что включение строгое, то есть 
	$$
	\exists z \in  {}^\perp(L^\perp) \setminus [L]_{\|\|}
	$$
	По следствию теоремы Хана-Банаха 
	$$
	\exists f \in X^* \colon f\big|_{[L]_{\|\|}} = 0 \quad f(z) = 1
	$$
	В силу первого условия $f \in L^\perp$, но  так как $z \in  {}^\perp(L^\perp)$, то 
	$$
	f(z) = 0
	$$
	Противоречие с $f(z) = 0$, значит $[L]_{\|\|} =  {}^\perp(L^\perp)$.
\end{proof}
Тогда продолжая утверждение следствия 2 можем записать 
$$
[\Ima A]_{\|\|} = {}^\perp (\Ker A^*)
$$
\begin{theorem}[Шмульян]\label{th:shm}
	Пусть $\Phi (X^*, \tau_{W^*}) \to \Cx$ --- линейны и $\tau_{W^*}$-непрерывный функционал, тогда 
	$$
	\exists ! x \in X \ \forall f \in X^* \colon\Phi(f) = f(x)
	$$
\end{theorem}
\begin{proof}
	докажем позже (или не докажем я так и не понял)
\end{proof}
\begin{lemma}
	Пусть $M \subset X^*$ --- подпространство, тогда 
	$$
	({}^\perp M)^\perp = [M]_{\tau_{W^*}}
	$$
\end{lemma}
\begin{proof}
	Ясно, что $M \subset ({}^\perp M)^\perp$. В силу \ref{claim:twcl} $({}^\perp M)^\perp$ --- $\tau_{W^*}$-замкнуто, тогда 
	$$
	[M]_{\tau_{W^*}} \subset ({}^\perp M)^\perp
	$$
	Аналогично лемме \ref{lem:doubleann1} рассмотрим
	$$
	f \in ({}^\perp M)^\perp \setminus [M]_{\tau_{W^*}}
	$$
	По следствию теоремы Хана-Банаха (для локально выпуклых топологических векторных пространств)
	$$
	\exists \omega \in Y\colon \omega\big|_{[M]_{\tau_{W^*}}} = 0 \quad \omega(f) \neq 0 
	$$
	Где $Y$ --- множество линейных $\tau_{W^*}$-непрерывных функционалов над $X^*$. В силу \ref{th:shm} такие функционалы однозначно определяются элементом из $X$. $\exists x \in X\colon \forall g \in X^*\colon \omega(g) = g(x)$. Тогда $x \in ^\perp M$, но тогда 
	$$
	f(x) = \omega(f) = 0
	$$
	Противоречие.
\end{proof}
\begin{remark}
	В случае рефлексивного пространства слабая* топология равна слабой, и так как $M$ --- выпукло, то по теореме Мазура $[M]_{\tau_{W^*}} = [M]_{\|\|}$. 
\end{remark}
\begin{remark}
	На лекции Константинов доказал вложение $[M]_{\|\|}$ без использования теоремы Шмульяна, в силу замечания мне не хочется его техать. 
\end{remark}
Значит мы можем продолжить следствие 2 и записать 
$$
(\Ker A)^\perp = [\Ima A^*]_{\tau_{W^*}}
$$
Оказывается без рефлексивности, но с учетом полноты можно получить следующую теорему.
\begin{theorem}[в духе Фредгольма]\label{th:frspirit}
	Пусть $X,Y$ --- банаховы пространства, $A \in \CL(X,Y)$ и $\Ima A$ ---- замкнут. Тогда 
	$$
	(\Ker A)^\perp = \Ima A^*
	$$
\end{theorem}
\begin{proof}
	Пусть $f\in (\Ker A)^\perp$ то есть $\forall x \in X, Ax = 0 \Rightarrow f(x) = 0$
	Хотим доказать, что $f \in \Ima A^*$, тогда будет верно
	$$
	[\Ima A]_{\|\|} \subset (\Ker A)^\perp \subset \Ima A^* \subset [\Ima A^*]_{\|\|}
	$$
	Откуда сразу $(\Ker A)^\perp = \Ima A^*$.
	
	Рассмотрим $h \colon \Ima A \to \Cx$ по формуле
	$$
	\forall x \in X  \colon h(Ax) = f(x) 
	$$
	Если $y = Ax_1 = Ax_2 \in \Ima A$, тогда 
	$$
	x_1 - x_2 \in \Ker A \Rightarrow f(x_1 - x_2)  = 0 \Rightarrow h(y) = f(x_1) = f(x_2)
	$$
	Поэтому $h$ определен корректно. Ясно, что $h$ --- линейный функционал. Так как $\Ima A$ --- замкнуто в банаховом пространстве $Y$, то $\Ima A$ --- само банахово, с другой стороны $X$ --- банахово, тогда по теореме банаха об открытом отображении (\ref{th:banachopenmap}) $A : X \to \Ima A$ --- открытое отображение. Значит 
	$$
	\exists r > 0 \colon O_r^Y(0) \cap \Ima A \subset A(O_1^X(0))
	$$
	Тогда $\forall y \in \Ima A \setminus \{0\} \Rightarrow  \frac{r}{2}\frac{y}{\|y\|} \in AO_1^X(0)$. Значит
	$$
	\exists x \in X \colon \|x\| \leq x \Rightarrow Ax = \frac{r}{2}\frac{y}{\|y\|}
	$$
	В силу линейности $A$ получаем
	$$
	y = A\left(\frac{2\|y\|}{r}x\right)
	$$
	Теперь вспоминая про $h$ 
	$$
	h(y) = h\left(A\left(\frac{2\|y\|}{r}x\right)\right) = f\left(\frac{2\|y\|}{r}x\right)
	$$
	Равенство выше верно для любого $y$, тогда
	$$
	|h(y)| \leq \|f\|\frac{2}{r}\|y\| \Rightarrow \|h\| \leq \frac{2}{r}\|f\|
	$$
	Значит $h$ --- непрерывный, то есть $h \in (\Ima A)^*$, по теореме Хана-Банаха продолжим его на весь Y:
	$$
	\exists g \in Y^* \colon g\big|_{\Ima A} = h \quad \|g\| = \|h\| 
	$$
	Тогда $\forall x \in X$:
	$$
	f(x) = h(Ax) = g(Ax) = (A^*g)(x) \Rightarrow f = A^*g \in \Ima A^*
	$$
	Что и требовалось!
\end{proof}
