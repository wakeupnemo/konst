\newpage
\section{Слабая топология в локально выпуклом ТВП}
Пусть $(X, \tau)$ --- локально выпуклое ТВП, тогда в $X^*$ мы заводим слабую* топологию $\tau_{w^*}$. Тогда $(X^*, \tau_{w^*})$ тоже локально выпуклое ТВП. Теперь мы рассматриваем $(X^*, \tau_{w^*})^*$. Про это пространство мы знаем теорему Шмульяна (\ref{th:shulian}). Так как $(X, \tau)$ --- локально выпукло, то существует $F: X \to (X^*, \tau_{w^*})^*$ --- изоморфизм. В $(X^*, \tau_{w^*})^*$ можно рассмотреть слабую* топологию, назовем ее $\tau_{w^{**}}$. Ее предбаза 
$$
\sigma_{w**} = \{V_{**}(\Phi, f , \eps) \mid \Phi \in (X^*, \tau_{w^*})^*, f \in X^*, \eps > 0\}
$$
Тогда 
\begin{definition}
	Будем называть слабой топологией в $X$ прообраз $\tau_{w^{**}}$ под действием $F$:
	$$
	\tau_w = \{F^{-1}(G)  \mid G \in \tau_{w{**}}\}
	$$
\end{definition}
По построению $\tau_w$, $F$ становится гомеморфизмом, тогда $\tau_w$ векторная локально выпуклая топология. Ее предбаза есть прообраз предбазы $\tau_{w{**}}$:
$$
V(x,f,\eps) = F^{-1}(V_{**}(F(x), f, \eps)) = \{y \in X \mid |f(x) - f(y)| < \eps \}
$$
То есть 
$$
\sigma_w = \{V(x, f, \eps) \mid x \in X, f \in X^*, \eps > 0\}
$$
\begin{claim}
	$\tau_{w^*}$ в $X^*$ где $X$ --- локально выпукла --- хаусдорфова
\end{claim}
\begin{proof}
	Соответствующие окрестности $f,g \in X^*$, если $f \neq g \Rightarrow \exists x \in X: \ f(x) \neq g(x)$, тогда
	$$
	V_*(f,x,\eps) \cap V_*(g,x,\eps) = \varnothing
	$$
	где $\eps = \frac{|f(x) - g(x)|}{2} > 0$
\end{proof}
Таким образом слабая топология тоже является хаусдорфовой. 

\begin{claim}
	$\tau_w$ --- это слабейшая топология в $X$, относительно которой $\forall f \in X^*$ топологически непрерывен. 
\end{claim}
\begin{proof}
	Пусть $\tilde{\tau}$ топология относительно которой все функционалы $f\in X^*$ непрерывны. Тогда 
	$$
	\forall x \in X\ \forall f \in X^* \colon \forall \eps > 0 \Rightarrow\exists \tilde{U}(x) \in \tilde{\tau}\colon \forall y \in \tilde{U}(x) \Rightarrow |f(y) - f(x)| < \eps
	$$
	Тогда такой $y$ лежит в элементе предбазы $\tau_{w^*}$ порожденной $x,f,\eps$ то есть
	$$
	\tilde{U}(x) \subset V(x,f,\eps)
	$$
	Но тогда $\forall y \in V(x,f,\eps) \ \Rightarrow V(y,f,\delta) \subset V(x,f,\eps)$ где $\delta = \eps - |f(x) - f(y)|$ (неравество треугольника), таким образом
	$$
	\forall y \in V(x,f,\eps)\colon \exists \tilde{U} \in \tilde{\tau}: \tilde{U} \subset V(y,f,\delta) \subset V(x,f,\eps)
	$$
	Значит $V(x,f,\eps)$ является $\tilde{\tau}$-открытым для любого $x$ и $f$. Так как этим исчерпываются элементы предбазы, то  
	$$
	\tau_w \prec \tilde{\tau}
	$$
	Что и требовалось.
\end{proof}
\begin{theorem}[Мазур]\label{th:mazur}
	Пусть $(X,\tau)$ --- локально выпуклое ТВП и $M \subset X$ --- выпукло и $\tau$-замкнуто, тогда $M$ --- $\tau_w$-замкнуто.
\end{theorem}
\begin{proof}
	Рассмотрим $x$ из дополнения $ X \setminus M$. По теореме об отделимости \ref{th:strong_sep} точка это выпуклый компакт, $M$ выпукло и замкнуто, $M \cap \{x\} = \varnothing$ тогда они строго отделимы:
	$$
	\exists f \in X^*\ \exists \gamma_1 < \gamma_2 \in \R\colon \Rea f(y) \leq \gamma_1 < \gamma_2 =\Rea f(x)
	$$
	Тогда рассмотрим окрестность $x$: $V(x,f, \gamma_2 - \gamma_1)$, покажем, что эта окрестность полностью лежит вне $M$: 
	$$
	\forall z \in V(x,f,\gamma_2 - \gamma_2) \Rightarrow |f(z) - f(x)| < \gamma_2 - \gamma_1
	$$
	Тогда 
	$$
	\Rea f(x) - \Rea f(z) \leq |f(z) - f(x)| < \gamma_2 - \gamma_1 \Rightarrow \Rea f(z) > \gamma_1 + \Rea f(x)  - \gamma_2 = \gamma_1
	$$
	Таким образом $V(x,f,\gamma_2 - \gamma_1) \subset X \setminus M$, значит $M$ --- $\tau_w$-замкнуто.
\end{proof}
\begin{next0}
	Если $(X, \|\|)$ --- ЛНП, тогда $M \subset X$ --- выпукло , то $[M]_{\|\|}$ --- слабо замкнуто, значит
	$$
	\{x_n\} \subset M\colon x_n \xrightarrow{\tau_w} y \Rightarrow  y \in [M]_{\|\|}
	$$
	Таким образом слабо сходящаяся последовательность на выпуклом множестве не может убегать далеко даже по сильной топологии.
\end{next0}
\begin{next0}
	Если $x_n \xrightarrow{\tau_w} y$, то $\exists y_m \in \conv\{x_n\}\colon y_{m} \xrightarrow{\|\|} y$
\end{next0}

\begin{claim}
	Пусть $\{x_n\}$ --- слабо сходящаяся последовательность в $(X, \|\|)$ тогда $\{x_n\}$ сильно ограничена, то есть $\exists R > 0 \colon  \forall n \in \N \ \|x_n\| \leq \R$
\end{claim}
\begin{proof}
	Слабая сходимость равносильна сходимости в смысле гомоморфизма $F$: 
	$$
	F(x_n) \xrightarrow{\tau_{w^{**}}} F(x) 
	$$
	Обозначим $\Phi_n = F(x_n)$, тогда $\Phi_n \in \CL(X^*, \Cx)$, но $X^*$ --- полное и $\Phi_n(f)$ сходится для любого $f \in X^*$, тогда $\Phi_n(f)$ --- ограничена. Значит мы попали в условие теоремы Банаха-Штейнгауза (\ref{th:banach-sht}) (точнее нам нужно следствие ее и теоремы Бэра). Таким образом $\exists R > 0: \ \|\Phi_n\| \leq R$, но по следствию из теоремы Хана-Банаха $\|x_n\| = \|\Phi_n\|$. Что и требовалось.
\end{proof}
\begin{theorem}
	Пусть $X$ --- рефлексивно, тогда $B_1(0)$ является слабым компактом.
\end{theorem}
\begin{proof}
	Запишем равенство множеств функционалов
	$$
	(X^*, \tau_{w^*})^* = F(X) = (X^*, \|\|)^* = X^{**}
	$$
	Первое равенство обеспечено теоремой Шмульяна, второе равенство --- определение рефлексивности. Теперь рассмотрим слабую* топологию в $X^{**}$ (второе сопряженное относительно нормы)
	$$
	V^{**}(\Phi, f, \eps) = \{\Psi \in X^{**} \mid |\Psi(f) - \Phi(f)| < \eps\}
	$$
	В силу рефлексивности все такие функционалы $\Phi$ порождаются элементом $x \in X$, то есть 
	$$
	\{\Psi \in X^{**} \mid |\Psi(f) - \Phi(f)| < \eps\} = V_{**}(F(x), f, \eps)
	$$
	Таким образом слабая* топология (относительно нормируемой топологии в $X^*$) и $\tau_{w^{**}}$ в $X^{**}$ совпадают, значит это одно и тоже пространство. 
	
	Но в $(X^*, \|\|)^*$ шар $B^{**}_1(0)$, являясь полярой $B^{*}_1(0)$, по теореме Банаха-Алаоглу (\ref{th:banach-alaoglu} ) слабо* компактен. Значит он является и $\tau_{w^{**}}$-компактом в $X^{**}$, тогда его прообаз под действием гомоморфизма $F$ является $\tau_w$-компактом, что и требовалось.
\end{proof}

\begin{theorem}[Эберлейн-Шмульян]
	Если $(X,\|\|)$ --- ЛНП и $K \subset X$--- слабый компакт, тогда $K$ --- сильно ограничен и слабый секвенциальный компакт.
\end{theorem}
\begin{proof}
	\begin{itemize}
		\item $\forall f \in X^* \Rightarrow f(K)$ --- компакт в $\Cx$ тогда $f(K)$ ограниченно в $\Cx$, тогда рассмотрев 
		$$
		M = \{Fx \mid x \in K\} \subset X^**
		$$
		Тогда $\forall f \in X^*\colon \exists R_f > 0 \Rightarrow |\Phi(f)| = |f(x)| \leq R_f$
		Тогда по теореме Банаха-Штейнгауза, $ \exists R > 0 \colon \forall x \in K: \ \|\Phi\| = \|Fx\| = \|x\| \leq R$, значит $K$ --- сильно ограничен.
		\item $\forall \{x_n\} \subset K$. Без ограничения общности $x_n \neq x_m$. Тогда рассматриваем 
		$$
		L = [\Lin \{x_n\}_{n=1}^\infty]_{\|\|}
		$$
		По теореме Мазура $L$ --- слабо замкнуто. 
	\end{itemize}

\end{proof}