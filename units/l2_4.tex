\newpage
\section{Лекция 4. Два с половиной кита Банаха}
\begin{definition}
	Пусть $X,Y$, $A \in \CL(X,Y)$ называется открытым отображением, если
	$$
	\forall G \stackrel{open}{\subset} X \Rightarrow A(G) \stackrel{open}{\subset} Y
	$$
\end{definition}
\begin{remark}
	Если $\Ima A \neq Y$, но если $A \in \CL(X, \Ima A)$ --- открытое отображение, где на $\Ima A$ топология индуцирована из $Y$, то $A$ называется открытым на свой образ.
\end{remark}
\begin{claim}
	$A \in \CL(X,Y)$ открыто тогда и только тогда, когда
	$$
	\exists r > 0 : \ O_r(0) \subset A(O_1(0))
	$$
\end{claim}
Для доказательства основной теоремы понадобится техническая 
\begin{lemma}
	Пусть $S \subset Z$ --- ЛНП. $t \in \Cx \setminus \{0\}$, тогда
	$$
	[tS] = t[S], \quad \Int (tS) = t \Int S
	$$
\end{lemma}
\begin{proof}
	\hfill
\begin{itemize}
	\item 	Пусть $x \in [tS]$, так как в ЛНП топологическое замыкание совпадает с секвенциальным
	$$
	\exists y_n \in S: \ x = \lim\limits_{n \to \infty} t y_n  \Leftrightarrow \frac{x}{t} = \lim\limits_{n \to \infty} y_n
	$$
	Таким образом $\frac{x}{t}\in [S]$, что и требовалось.
	\item Пусть $ x \in \Int(tS)$. По определению
	$$
	\exists \eps > 0 : \ O_\eps(x) \subset t S \Rightarrow \frac{1}{t}O_\eps(x) \subset S \Leftrightarrow O_{\frac{\eps}{|t|}}\left(\frac{x}{t}\right) \subset S
	$$
	Что и требовалось.
\end{itemize}
\end{proof}
\begin{theorem}[Банаха об открытом отображении]\label{th:banachopenmap}
	Пусть $X,Y$ --- банаховы пространства. $A \in \CL(X,Y)$ такой что $\Ima A = Y$. Тогда $A$ --- открытое отображение.
\end{theorem}
\begin{proof}
	По условию $\Ima A = Y$, тогда можно представить $Y$ объединением шаров
	$$
	Y = \bigcup_{n=1}^\infty AO_n(0)
	$$
	Так как $Y$ --- банахово, то по теореме Бэра (\ref{th:bear}) существует $n_0 \in \N$: 
	$$
	\Int [AO_{n_0}(0)] \neq \varnothing
	$$
	В силу леммы имеем
	$$
		\Int [A(O_{n_0}(0))] = \Int (n_0 [A O_1(0)]) = n_0 \Int [A O_1(0)] \neq \varnothing
	$$
	Множество $\Int [A O_1(0)]$ выпукло, симметрично относительно нуля и не пусто: 
	$$
	\exists x \in \Int [A O_1(0)] \Rightarrow -x \in \Int [A O_1(0)]
	$$
	Тогда, пользуясь выпуклостью, 
	$$
	\frac{x - x}{2}  = 0 \in \Int [A O_1(0)]
	$$
	Таким образом
	$$
	\exists r > 0:  \ O_r(0) \subset \Int [A O_1(0)] 
	$$
	В силу леммы 
	$$
	\forall \eps > 0 : \ O_{r\eps} \subset \eps \Int [A O_1(0)] = \Int [A O_\eps(0)]
	$$
	Исследуем $[A O_1(0)]$. Рассмотрим произвольную точку $y \in [A O_1(0)]$ по определению любая окрестность $y$ должна иметь непустое пересечение с $A O_1(0)$. Любая окрестность точки содержит шар с центром в этой точке, или, что то же самое, шар с центром в нуле, сдвинутый на эту точку, тогда можно считать, что 
	$$
	y - [A O_{\frac{1}{2}}(0)] \cap AO_1(0) \neq \varnothing
	$$
	Тогда $\exists y_1 \in [A O_\frac{1}{2}(0)], \exists x_1 \in O_1(0)$:
	$$
	y - y_1 = A(x_1)
	$$
	Уменьшим радиус: 
	$$
	y_1 - [A O_{\frac{1}{4}(0)}] \cap AO_\frac{1}{2}(0) \neq \varnothing
	$$
	Тогда $ \exists y_2 \in [A O_\frac{1}{4}(0)], \exists x_2 \in O_\frac{1}{2}(0)$: 
	$$
	y_1 - y_2 = Ax_2
	$$
	Продолжив таким образом получим последовательности 
	$$
	\{y_n\}_{n=1}^\infty \subset Y \quad \{x_n\}_{n=1}^\infty \subset X
	$$
	Заметим, что 
	$$
	y_{n+1} = y_n - y_1 + y_1 - y_2 + \dots + y_n - y_{n+1} = Ax_1 + Ax_2 + \dots + Ax_{n+1} = A\left(\sum_{k=1}^{n+1} x_k\right)
	$$
	Кроме того
	$$
	\|y\| \leq \|A\| \frac{1}{2^n} \Rightarrow y_n \to 0, \quad \|x_n\| \leq \frac{1}{2^{n-1}}
	$$
	Для подходящего $n$ будем иметь:
	$$
	\left\|\sum_{k=1}^{n+m+1}x_k - \sum_{k=1}^{n+1} x_k\right\| \leq \sum_{k=n+1}^{n+m+1} \|x_k\| \leq \sum_{k=n+1}^{n+m} \frac{1}{2^{k-1}}\leq \frac{1}{2^{n-1}} < \eps
	$$
	Значит последовательность $\left\{\sum_{k=1}^n x_k\right\}$ --- фундаментальна в полном $X$, тогда
	$$
	\exists x = \lim\limits_{n \to \infty} \sum_{k=1}^n x_k \in X
	$$
	Причем
	$$
	\|x\| \leq \sum_{k=1}^\infty \|x_k\| \leq \sum_{k=1}^\infty \frac{1}{2^{k-1}} =2 < 3
	$$
	Таким образом $\forall y \in [AO_1(0)] \Rightarrow y = A(x), \ \|x\|  < 3$. То есть
	$$
	[AO_1(0)] \subset AO_3(0)
	$$
	Но тогда $O_\frac{r}{3} \subset AO_1(0)$ что и требовалось.
\end{proof}
\begin{next0}
	Если $Z,X$ --- банаховы пространства $A \in \CL(X, Z)$ и $\Ima A = Y$ --- замкнуто в $Z$, тогда $A \colon X \to Y $--- открытое отображение из $X$ на $Y$. 
\end{next0}
\begin{theorem}[Банаха об обратном операторе]\label{th:inv_op}
	Пусть $X,Z$ --- банаховы пространства, $A \in \CL(X,Z)$, тогда 
	$$
	\exists A^{-1} \in \CL(\Ima A, X)
	$$
	Тогда и только тогда когда
	\begin{itemize}
		\item $ \Ker A = \{0\}$ 
		\item $\Ima
		 A $ --- замкнуто в $Z$
	\end{itemize}
\end{theorem}
\begin{proof}
	Ясно, что $A \colon X \to Z$ --- инъективен если и только если $\Ker A = \{0\}$. Значит 
	$$
	\exists A^{-1}\colon \Ima A \to X
	$$ 
	Проверим линейность
	$$
	y_1, y_2 \in \Ima A \Rightarrow \exists! x_{1,2} \in X \colon A x_{1,2} = y_{1,2}
	$$
	Значит $A(\alpha_1 x_1 + \alpha_2 x_2) = \alpha_1 y_1 \alpha_2 y_2$ откуда следует линейность обратного оператора. Будем проверять второе условие.
	\begin{enumerate}
		\item[$\Rightarrow$] Пусть  $A^{-1} \in \CL(\Ima A, X)$. Пусть $y \in [\Ima A]$, тогда $\exists y_n \in \Ima A\colon y_n \to y$. Тогда последовательность прообразов $x_n \in A^{-1}(y_n) \in X$ фундаментальна:
		$$
		\|x_n - x_m \| = \|A^{-1}(y_n - y_m)\| \leq \|A^{-1}\| \|y_n - y_m\| \xrightarrow{n,m \to \infty} 0  
		$$
		Так как $X$ --- полно, то $\exists x = \lim\limits_{n \to \infty}x_n \in X$, но $A$ --- непрерывен, значит
		$$
		A(x) = \lim\limits_{n \to \infty} A(x_n) = \lim\limits_{n \to \infty} y_n = y \in \Ima A
		$$
		Что и требовалось.
		\item[$\Leftarrow$] Если $\Ima A $ --- замкнут в полном $Z$, то $\Ima A$ --- полон, тогда по теореме об открытом отображении 
		$$
		A \colon X \to \Ima A \text{ --- открытое отображение}
		$$
		Тогда для обратного оператора:
		$$
		\forall G \stackrel{open}{\subset} X \Rightarrow (A^{-1})^{-1}(G) = A(G) \text{ --- открыто в $\Ima A$}
		$$
		Таким образом прообраз любого открытого множества открыт, что означает непрерывность $A^{-1}$. Что и требовалось.
	\end{enumerate}
\end{proof}
