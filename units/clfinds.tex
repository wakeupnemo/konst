\newpage
\section{Замкнутость конечномерного подпространства}
Далее в этой секции $(X, \tau)$ --- топологическое векторное пространство, $L \subset X$ --- линейное подпространство. Которое мы наделяем топологией $\tau_L$ --- индуцированной с $\tau$. Совершенно ясно, что $(L, \tau_L)$ --- тоже ТВП. В этом разделе нам предстоит доказать, что если $L$ --- конечномерно, то $L$ --- замкнуто.
\begin{definition}
	Пусть $(X_1, \tau_1), (X_2, \tau_2)$ --- два топологических пространства. И $\psi: X_1 \rr X_2$ --- биекция, тогда если $ \psi (X_1, \tau_1) \rr (X_2, \tau_2)$ и $\psi^{-1}: (X_2, \tau_2) \rr (X_1, \tau_1)$ --- топологически непрерывны, то $\psi$ --- называется гомеоморфизм. А эти пространства называются гомеоморфными.
\end{definition}
Приведем несколько утверждений по поводу гомеоморфных пространств:
\begin{claim}
	Если $\psi$ --- гомеоморфизм, то 
	\begin{enumerate}
		\item $\forall G \in \tau_1 \Rightarrow \psi(G) \in \tau_2 $
		\item $M \subset X_1$ --- $\tau_1$-замкнуто, то $\psi(M)$ --- $\tau_2$-замкнуто. 
		\item Если $M \subset X_1$, то $\psi\left([M]_{\tau_1}\right) = \left[\psi(M)\right]_{\tau_2}$
		\item Если $K \subset X_1$ --- $\tau_1$-компакт, то $\psi(K)$ --- $\tau_2$-компакт.
	\end{enumerate}
\end{claim}
\begin{proof}
	\begin{enumerate}
		\item Так как обратное отображение непрерывно, то прообраз открытого открыт:
		$$
		\psi(G) = (\psi^{-1})^{-1}(G) = \{x_2 \in X_2 \mid \psi^{-1}(x_2) \in G\} \in \tau_2
		$$
		\item рассмотрев $X \setminus M$ и применив предыдущее утверждение получаем требуемое.
		\item 
		$$
		\psi\left(\bigcap_{
			\substack{
				{S \subset X_1} \\
				 {M \subset S } \\
				 {X_1\setminus S \in \tau}
			}
		}S\right) = (\psi^{-1})^{-1}\left(\bigcap S\right) = \bigcap (\psi^{-1})^{-1}(S) = \bigcap_{\substack{S \subset X_1 \\ M \subset S \\ X_1\setminus S \in \tau}} \psi(S)
		$$
		$\psi(S)$ --- $\tau_2$-замкнуто, а $M \subset S \Leftrightarrow \psi(M) \subset \psi(S) = N$ Тогда:
		$$
		\psi\left([M]_{\tau_1}\right) = \bigcap_{\substack{N \subset X_2 \\ \psi(M) \subset N \\ X_2\setminus N \in \tau}}N = \text{(по определению)} = \left[\psi(M)\right]_{\tau_2}   
		$$
		\item Пусть $P$ --- $\tau_2$-покрытие $\psi(K)$, тогда:
		$$
		\{\psi(V) \mid V \in P \} \text{ --- $\tau_1$-покрытие $K$}
		$$
		Значит найдется конечное подпокрытие: $\exists V_1, \dots ,V_N \in P$, такие что $\psi^{-1}(V_1), \dots, \psi^{-1}(V_N)$ --- подпокрытие $K$, тогда $V_1, \dots,V_N$ --- подпокрытие $\psi(K)$, значит $\psi(K)$ --- компакт.
	\end{enumerate}
\end{proof}
\begin{definition}
	ТВП $(X, \tau)$ --- называется локально компактным, если $\exists U(0) \in \tau$, такая что $[U(0)]_\tau$ --- компакт в $(X, \tau)$
\end{definition}
\noindent \textbf{Идея доказательства замкнуности конечномерного подпространства}. \newline
Следующее утверждение, которое мы докажем, свяжет локальную компактность и замкнутость. После мы воспользуемся следующим. Eсли $L$ конечномерно, то существует изоморфизм между $L$ и $\R^n$ мы докажем, что такой изоморфизм всегда является гомеоморфизмом. $\R^n$ с  является локально компактным пространством, кроме того ясно, что это свойство сохраняется при гомеоморфизме. Это и завершит полное доказательство.
\begin{claim}
	\label{claim:localcompact}
	Если $(X, \tau)$ --- ТВП и $L \subset X$ --- подпространство такое что $(L, \tau_L)$ --- локально компактно ($\tau_L$ --- индуцированная топология), то $L$ --- $\tau$-замкнуто в $(X,\tau)$
\end{claim}
\begin{proof}
	По условию $\exists U \in \tau, 0 \in U$, такая что $[U \cap L]_{\tau_L} = K$--- компакт в $(L, \tau_L)$, по лемме (\ref{lem:sym1}) найдется симметричная окрестность нуля $U_1 \in \tau$, такая что
	$$
	U_1 + U_1 \subset U
	$$
	Кроме того по лемме (\ref{lem:dest}) найдется окрестность нуля $U_2 \in \tau$, такая что $[U_2]_\tau \subset U_1$. Рассмотрев $V:= U_2\cap (-U_2)$ --- симметричную окрестность с таким же свойством, получим:
	$$
	[V]_\tau + [V]_\tau \subset U \in \tau
	$$
	Возьмем произвольную точку $x \in X$, и рассмотрим множество:
	$$
	S_x := L \cap (x + [V]_\tau)
	$$
	Так как трансляция замкнутого множества не меняет замкнутости, то $x + [V]_\tau$ --- $\tau$-замкнуто в $(X, \tau)$, тогда $S_x$ --- $\tau_L$-замкнуто в $(L, \tau_L)$. Если $S_x \neq \varnothing$, то $S_x$ --- будет компактом в $(L, \tau_L)$. Действительно:
	$$
	\exists x_0 \in S_x \Rightarrow \forall y \in S_x: y - x_0 = \underbrace{(y-x)}_{\in [V]_\tau} - \underbrace{(x_0-x)}_{\in [V]_\tau}\in [V]_\tau - [V]_\tau = [V]_\tau + [V]_\tau \in U
	$$
	Таким образом $y - x_0 \in U$,  при этом $y,x_0 \in S_x \subset L$, значит $y- x_0 \in L\cap U \subset K$ --- $\tau_L$-компакт,  тогда:
	$$
	S_x \subset x_0 + K
	$$
	Трансляция не меняет компактности, в силу непрерывности сложения. Таким образом мы погрузили замкнутое множество в компакт, значит оно является компактом.
	Тогда возьмем $\forall x \in [L]_\tau$, значит $\forall W \in \tau$ --- окрестности нуля:
	$$
	(x + W)\cap L \neq \varnothing
	$$
	Рассмотрим 
	$$
	\beta_V = \{W \in \tau \mid 0 \in W\subset V\}
	$$
	$\beta_V$ --- локальная база нуля в $(X, \tau)$, так как для любой окрестности нуля $U$:
	$$
	U \supset U \cap V \subset V  \text{ содержит ноль и открыто } \Rightarrow  U \cap V \in \beta_V
	$$
	Теперь смотрим на множество $S_W$ для каждого $W \in \beta_V$:
	$$
	S_W = (x + [W]_\tau)\cap L 
	$$ 
	\begin{itemize}
		\item Оно не пусто $ S_W = (x + [W]_\tau)\cap L \supset (x+ W)\cap L \neq \varnothing$
		\item Оно $\tau_L$-замкнуто по построению 
		\item $S_W \subset (x + [V]_\tau)\cap L$ --- компакт в $(L, \tau_L)$
	\end{itemize}
	Таким образом $\forall W \in \beta_V$ $S_W$ --- компакт в $(L, \tau_L)$. Теперь нам хочется доказать, что пересечение $\bigcap_{W \in \beta_V}S_W \neq \varnothing$. Докажем это в два этапа.
	\begin{itemize}
		\item Рассмотрим конечный набор $W_1, \dots, W_N \in \beta_V$, тогда покажем, что $\bigcap_{k=1}^N S_{W_k} \neq \varnothing$. Действительно:
		$$
		\bigcap_{k=1}^N S_{W_k} = L \cap (x + [W_1]_\tau)\cap \dots \cap (x + [W_N]_\tau) \supset L \cap \left( x + \bigcap_{k=1}^N [W_k]_\tau\right) \supset L \cap \left(x + \left[\bigcap_{k=1}^N W_k\right]_\tau\right) = S_{\bigcap_{k=1}^N W_k}
		$$
		Но $\bigcap_{k=1}^N W_k \in \beta_V$, тогда по доказанному выше $S_{\bigcap_{k=1}^N W_k}$ не пусто.
		\item Теперь предположим $\bigcap_{W \in \beta_V}S_W = \varnothing$, тогда рассмотрим $S_V \in \beta_V$, так как пересечение всех $S_W$ --- пусто, то 
		$$
		S_V = S_V \setminus \bigcap_{W \in \beta_V}S_W  = \bigcup_{W \in \beta_V} S_V \setminus S_W 
		$$
		Так как $S_W$ --- $\tau_L$-замкнуто, тогда $L \setminus S_W \in \tau_L$, тогда 
		$$
		S_V = \bigcup_{W \in \beta_V} S_V \setminus S_W \subset \bigcup_{W \in \beta_V}\underbrace{L \setminus S_W}_{\in \tau_L} 
		$$
		Таким образом получили открытое покрытие компакта $S_L$. Радостно получаем конечное подпокрытие:
		$$
		\exists W_1, \dots W_N \in \beta_V: \ S_V \subset \bigcup_{k=1}^N L \setminus S_{W_k} \Rightarrow S_V = \bigcup_{k=1}^N S_V \setminus S_{W_k}  = S_V \setminus \bigcap_{k=1}^N S_{W_k}
		$$
		Значит $\bigcap_{k=1}^N S_{W_k} = \varnothing$, противоречие с предыдущим пунктом.
	\end{itemize}  
Таким образом $\bigcap_{W \in \beta_V}S_W \neq \varnothing$. Значит $\exists z \in \bigcap_{W \in \beta_V}S_W  \subset L$ тогда:
$$
\forall W \in \beta_V: \ z \in x + [W]_\tau 
$$
Для любой окрестности нуля $\forall \tilde{U} \in \tau, 0 \in \tilde{U}$:
$$
\exists \hat{U}: \tilde{U} \supset 	[\hat{U}]_\tau  \supset [\hat{U}\cap V]_\tau = [W]_\tau
$$
Таким образом: 
$$
\forall \tilde{U}(0) \in \tau: z - x \in \tilde{U}(0) 
$$
Получили, что $z,x$ --- топологически неотделимы, но в силу свойств векторной топологи такого не может быть, значит $x = z \in L$, таким образом $L$ --- $\tau$-замкнуто. УРА
\end{proof}
Для доказательства того, что изоморфизм будет гомеоморфизмом нам понадобится следующая лемма.
\begin{lemma}[Критерий топологической непрерывности линейного функционала в ТВП]
	Пусть $(X, \tau)$ --- ТВП и $f: X \rr \R$ --- линейное отображение. Тогда $f: (X, \tau) \rr \R$ --- является топологически непрерывным тогда и только тогда, когда $Ker f $ --- $\tau$-замкнуто
\end{lemma}
\begin{proof}
	\hfill
	\begin{enumerate}
		\item[$\Rightarrow$] Если $f$ --- непрерывно, тогда $Ker f = f^{-1}(\{0\})$ --- замкнуто, так как $\{0\}$ замкнуто в $\R$ 
		\item[$\Leftarrow$] Пусть ядро $Ker f$ --- $\tau$-замкнуто. Тогда $ X \setminus Ker f \in \tau$. Если $Ker f  = X$, то $\forall x \in X : \ f(x) = 0$, константное отображение является непрерывным. 
		
		Теперь считаем, что $Ker f \neq X$. Значит 
		$$
		\exists x_0 \in X \setminus Ker f \in \tau
		$$
		Значит существует окрестность нуля $$V:= X\setminus Ker f - x_0: \  0 \in V, V \in \tau$$ По построению $(x_0 + V) \cap Ker f = \varnothing$. Так как $V$ --- окрестность нуля, то по лемме (\ref{lem:tvs1}) $\exists W \in \tau, W \subset V$ --- уравновешенная окрестность нуля, то есть 
		$$\forall \lambda\in \R, \ |\lambda| < 1:\ \lambda W \subset W$$
		Тогда, сужаясь на эту окрестность, имеем $(x_0 + W)\cap Ker f = \varnothing$. Рассмотрим $f(W) \subset \R$.
		
		Предположим, что $f(W)$ --- неограниченно в $\R$, тогда 
			$$
			\forall \alpha \in \R : \ \exists x \in W : \ |f(x)| > |\alpha|
			$$
			Рассмотрим $\lambda = \frac{\alpha}{f(x)} \in \R$. Ясно что $|\lambda| < 1$, тогда по свойствам окрестности $W$: $\lambda W \subset W$, значит 
			$$
			\lambda x \in W \Rightarrow f( \lambda x ) = \lambda f(x) = \alpha \in f(W)
			$$
			В силу произвольности $\alpha$ получаем, что $f(W)=\R$. Но в таком случае рассмотрев $\alpha = - f(x_0) \in \R$ мы всегда сможем  \sout{сварить суп} найти такой $x$, что $x \in W, \  f(x) = - f(x_0)$. Из этого равенства моментально следует, что 
			$$
			x + x_0 \in Ker f
			$$
			С другой стороны $x \in W$, получаем противоречие с $(x_0 + W)\cap Ker f = \varnothing$
			Значит $f(W)$--- ограниченно. Это ограниченность образа окрестности нуля некисло сигнализирует о непрерывности функционала, покажем это строго. Имеем:
			$$
			\exists M > 0: \ \forall x \in W \Rightarrow |f(x)| \leq M 
			$$
			Значит 
			$$
			\forall \eps > 0 \exists V = \frac{\eps}{M}W \in \tau.
			$$
			Тогда для произвольных $z, y \in X: z = y + V$ имеем:
			$$
			f(z) - f(y) \in f(\frac{\eps}{M}W) = \frac{\eps}{M}f(W) \Rightarrow |f(z) - f(y)| \leq \frac{\eps}{M}*M = \eps 
			$$
			Таким образом $f$ --- непрерывен. 
 	\end{enumerate}
\end{proof}

\begin{claim}
	\label{claim:hom}
	Для $L \subset X, \dim L = n$ изоморфизм $\varphi: \R^n \rr (L, \tau_L)$ является гомеоморфизмом
\end{claim}
\begin{proof}
	Нужно доказать, что $\varphi$ и $\varphi^{-1}$ --- топологически непрерывны. Будем проводить индукцию по размерности пространства $n$. 
	\begin{itemize}
		\item При $n = 1$ $\varphi: \R \rr L$ --- изоморфизм. Пусть $\varphi(1) = e \in L$, тогда 
		$$
		\forall \alpha: \ \varphi(\alpha) = \alpha \varphi(1) = \alpha e
		$$
		Так как $\alpha \overbrace{\varphi}{\mapsto} \alpha e \in X$ является $\tau$-непрерывным как умножение скаляр в векторной топологии, то $\varphi$ --- $\tau_L$-непрерывно. Теперь рассмотрим обратное отображение $\varphi^{-1}$. Для произвольной точки $\alpha e \in L$:
		$$
		\varphi^{-1}(\alpha e) = \alpha 
		$$
		Значит $\varphi^{-1}: L \rr \R$ --- линейный функционал. Причем так как это изоморфизм, то $Ker \varphi^{-1} = \{0\}$ --- замкнутое множество. Таким образом $\varphi^{-1}$ --- непрерывен по предыдущей лемме. Значит $L$ гомеоморфно $\R$.
		\item Пусть $n \in \N$ любое $n$-мерное подпространство гомеоморфно $R^n$ а значит $\tau_L$-замкнуто. Рассмотрим $\dim L = n+1$ и имеем изоморфизм $\varphi: \R^{n+1} \rr L$ Рассмотрим стандартный базис в $\R^{n+1}$: $\{g_i\}_i^{n+1}$, тогда пусть $\alpha \in \R^{n+1}$, имеем:
		$$
		\varphi(\alpha) = \varphi\left(\sum_{k=1}^{n+1} \alpha_k g_k\right) = \sum_{k=1}^{n+1}\alpha_k \varphi(g_k) = \sum_{k=1}^{n+1} \alpha_k e_k
		$$
		Тогда это отображение является $\tau_L$-непрерывным в силу непрерывности суммы и умножения скаляра на фиксированный вектор.
		Для обратного отображения имеем 
		$$
		\varphi^{-1}(x) = \varphi^{-1} \left(\sum_{k=1}^{n+1}\alpha_k(x) e_k\right) = \sum_{k=1}^{n+1}\alpha_k(x)g_k
		$$
		Где $\alpha_k(x): (L, \tau_L) \rr \R$ --- координаты вектора $x$. Таким образом $\alpha_k$ --- линейные функционалы, но
		$$
		Ker \alpha_k = \{x \in L \mid \alpha_k(x) = 0\} = Lin\{e_1, \dots e_{k-1}, e_{k+1}, \dots e_{n+1}\} \text{ $n$-мерное подпространство в $(X, \tau)$}
		$$
		По предположению индукции и по предположению индукции $Ker \alpha_k$ --- $\tau_L$-замкнуто, тогда по критерию топологической непрерывности $\alpha_k$ топологически непрерывно. Тогда $\varphi^{-1} \sum_{k=1}^{n+1}\alpha_k(x)g_k$ --- непрерывно, что и требовалось.
	\end{itemize}
\end{proof}
Таким образом мы доказали теорему
\begin{theorem}
	\label{th:clfinds}
	Пусть $L \subset X$ --- конечномерное подпространство топологического векторного пространства $(X, \tau)$ тогда $L$ --- $\tau$-замкнуто
\end{theorem}
\begin{proof}
	Так как $L$ --- конечномерно, существует изоморфизм $\varphi: \R^n \rr L$ по утверждению (\ref{claim:hom}) он является гомеоморфизмом, но $\R^n$ является локально компактным пространством, это свойство сохраняется при гомеоморфизме, значит $L$ --- локально компактно в $(X, \tau)$, тогда по утверждению (\ref{claim:localcompact}) $L$ --- $\tau$-замкнуто.
\end{proof}
