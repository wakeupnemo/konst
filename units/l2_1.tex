\newpage
\section{Лекция 1. Пространство непрерывных линейных операторов, сопряженное пространство, теорема Хана-Банаха}


\begin{definition}
	Пусть $X,Y$ --- линейные нормированные пространства относительно поля $\Cx$. Тогда множество всех непрерывных линейных отображений обозначается:
	$$
	\mathcal{{L}}(X, Y) = \{A: X \to Y \mid A \text{ --- линеен на $X$}, \ A \text{ --- непрерывен на $X$}\}
	$$
\end{definition}

\begin{claim}
	$\CL(X,Y)$ --- линейное пространство.
\end{claim}
\begin{proof}
	очевидно. 
\end{proof}
\begin{claim}
	В пространстве $\CL(X,Y)$ можно ввести норму:
	\begin{equation}
	\|A\|_{op} = \sup\limits_{\|x\|_X \leq 1} \|A(x)\|_Y = \sup\limits_{\|x\|_X = 1} \|A(x)\|_Y = \sup\limits_{x \neq 0}\frac{\|A(x)\|_Y}{\|x\|_X}
	\end{equation}
\end{claim}
\begin{remark}
	Далее я не буду писать индексы у норм. Чтобы понять какая из норм имеется в виду в том или иному случае, необходимо посмотреть на аргумент. 
\end{remark}
\begin{proof}
	Проверим все аксиомы нормы
	\begin{itemize}
		\item$A \in \CL(X,Y) \Rightarrow 0 \leq \|A\| \leq \infty$
		\item Из последнего равенства формулы (1) имеем
		$$
		\|A\| = 0 \Leftrightarrow \forall x \in X: \ A(x) = 0 \Leftrightarrow A = 0
		$$
		\item $ \lambda \in \Cx$ из первого равенства из формулы (1):
		$$
		\|\lambda A\| = |\lambda| \|A\|
		$$
		\item Неравенство треугольника следует из неравенства треугольника для соответствующей нормы:
		$$
		\|Ax + Tx\| \leq \|Ax\| + \|Tx\|
		$$
		Переходя к супремуму по единичному шару получаем требуемое.
		\end{itemize}
\end{proof}
\begin{claim}
	Линейный функционал $A : X \to Y$ непрерывен тогда и только тогда, когда его норма конечна
\end{claim}
\begin{proof}
	\begin{enumerate}
		\item[$\Rightarrow$] Пусть $A \in \CL(X, Y)$. Тогда для $\eps =1$ воспользуемся непрерывностью $A$ в нуле, тогда $\exists \delta > 0$: 
	$$
	\forall \|x\| \leq \delta: \ \|A(x)\| \leq 1 
	$$
	Тогда $\forall x \in X: \  \|x\| \leq 1$:
	$$
	\|A(x)\| = \frac{1}{\delta}\|A(\delta x)\| \leq  \frac{1}{\delta}
	$$

	Значит норма $\|A\| \leq \frac{1}{\delta}$, то есть норма конечна.
	\item[$\Leftarrow$] Если норма оператора конечна, то 
	$$
	\forall x \in X: \ \|A(x)\| \leq \|A\| \|x\|
	$$
	Тогда 
	$$
	\|A(x_1) - A(x_2)\| \leq \|A\| \|x_1 - x_2\|
	$$
	То есть оператор является липшецевым с константой $\|A\|$ откуда сразу следует его непрерывность.
	\end{enumerate}
\end{proof}
\begin{definition}
	$\tau_U$ --- топология в $\CL(X,Y)$ порожденная операторной нормой. Называется равномерной операторной топологией.
\end{definition}
\begin{remark}
	Индекс $U$ подчеркивает, что эта топология обеспечивает равномерную сходимость операторов на единичном шаре.
\end{remark}
Вспомним что на пространстве всех функций $Y^X$ можно ввести топологию Тихонова $\tau_T$
\begin{definition}
	Топологию индуцированную на пространство $\CL(X,Y)$ с пространства с $(Y^X, \tau_T)$ будем обозначать $\tau_s$. И называть сильной операторной топологией. 
\end{definition}
Обе введенные топологии являются векторными топологиями.
\begin{itemize}
	\item $(\CL(X,Y), \tau_U)$ --- линейное нормированное пространство, значит является топологическим векторным
	\item $(\CL(X,Y), \tau_S)$ --- топологическое векторное так как $Y$ --- линейное нормированное а значит топологическое векторное.
\end{itemize}
Обобщим определение полноты на топологические векторные пространства.
\begin{definition}
	Пусть $(Z, \tau)$ --- топологическое векторное пространство. Говорят, что $\{z_n\} \subset Z$ --- последовательность Коши если
	$$
	\forall U(0) \in \tau \ \exists N \in \N: \ \forall n,m \geq N \Rightarrow z_n - z_m \in U(0)
	$$
\end{definition}
\begin{remark}
	Если $(Z, \tau)$ --- нормируемое пространство, то определение выше соответствует обычному определению фундаментальной последовательности.
\end{remark}
\begin{definition}
	$(Z, \tau)$ называется полным, если любая последовательность Коши является сходящейся. 
\end{definition}

\begin{definition}
	Элемент пространства $\CL(X, \Cx)$ называется линейным непрерывным функционалом, а пространство $\CL(X, \Cx)$ с операторной нормой называется сопряженным пространством к пространству $X$ и обозначается $X^*$
\end{definition}

\begin{theorem}[Хан, Банах]\label{th:h-b}
	Пусть выполнены следующие условия:
	\begin{enumerate}
		\item $X$ --- вещественное линейное пространство.
		\item $L \subset X$ --- подпространство.
		\item $f: L \to \R$ --- вещественное линейное отображение.
		\item $\exists p: X \to \R$ --- функция такая что
		\begin{itemize}
			\item $p(x + y) \leq p(x) + p(y)$ (полуаддитивность)
			\item $\forall \lambda > 0: \ p(\lambda x ) = \lambda p(x)$ (положительная однородность)
		\end{itemize}
		\item $\forall x \in L: \ f(x) \leq p(x)$
	\end{enumerate}
	 Тогда существует $g: X \to \R$ --- вещественное линейное отображение, такое что 
	 $$
	 g\big\vert_{L_0} = f \text{ и }  \forall x \in X: g(x) \leq p(x)
	 $$
\end{theorem}
\begin{proof}
Рассмотрим семейство
$$
\Phi = \left\{(M,h) \mid\begin{matrix}
	 &M\subset X \text{ --- подпространство } \\ 
	 &L \subset M, \ h \colon M \to \R \text{ --- вещественно линейный функционал}, \\ 
	 &h\big|_L = f, \\ &\forall x \in M \colon h(x) \leq p(x)
\end{matrix}\right\}
$$	
Оно не пусто, так как $(L,f) \in \Phi$. Введем на $\Phi$ частный порядок
$$
(M_1, h_1) \leq (M_2, h_2)
$$
Проверка аксиом частичного порядка очевидна. Таким образом $(\Phi, \leq)$ --- ЧУМ. По теореме Хаусдорфа (\ref{th:Hausdorf}) в $(\Phi, \leq)$ существует максимальный по включению ЛУМ $N$. Рассмотрим
$$
M_* = \bigcup_{(M,h) \in N} M
$$
Тогда $M_*$ --- подпространство $X$, так как если $x, y \in M_*$, то $x \in M_x, y \in M_y$, но $M_x,M_y \in N$, значит сравнимы, не умаляя общности $M_x \subset M_y$, тогда $\alpha x + \beta y \in M_y \subset M_*$. Рассмотрим
$$
h_*\colon M_* \to \R \quad h_*\big\vert_{M} = h \ \forall(M,h) \in N
$$
Тогда $h_* \leq p$ на $M_*$, $h_*\big\vert_{L} = f$. Осталось доказать, что $M_* = X$. Предположим противно, то есть $\exists x_0 \in X\setminus M_*$, тогда строим
$$
M_0 = M_* \oplus \Lin\{x_0\}
$$
И строим $h_0(x + t{x_0})  = h(x) + at$, где $a = h_0(x_0)$ нам пока не известно. Тогда ясно, что 
$$
h_0 \big\vert_{M_*} = h_*
$$
Нужно определить $a$ так, чтобы $\forall x \in M_0\colon h_0(x) \leq p(x) $. Если мы найдем такое $a$, то $M_0$ будет сравнимо со всеми элементами $N$ и строго больше, что будет противоречить максимальности ЛУМА $N$. 

Поймем, что мы хотим от $a$, чтобы было выполнено $h_0(x) \leq p(x)$ Пусть $t> 0$, тогда
$$
h_0(x + tx_0) \leq p(x + tx_0) \Rightarrow a \leq p\left(\frac{x}{t} + x_0\right) - h_*\left(\frac{x}{t}\right)
$$
Перейдя к инфинуму получим
$$
a \leq \inf_{x \in M_*}\left[p(x+x_0) - h_*(x)\right]
$$
Пусть теперь $ t < 0$, тогда аналогично получим
$$
-a \leq p\left(\frac{x}{|t|} -x_0\right) - h_*\left(\frac{x}{|t|}\right)
$$
Переходя к супремуму с учетом предыдущего получим: 
$$
\sup_{z \in M_*}\left(h_*(z) - p(z - x_0)\right) \leq a \leq  \inf_{x \in M_*}\left[p(x+x_0) - h_*(x)\right]
$$
Реализуется ли эта ситуация? Оказывается да, ведь $\forall z, x \in M_*$ 
$$
h_*(z) + h_*(x) = h_*(x + z) \leq p(x+z) \leq p(x+x_0) + p(z - x_0)
$$
Значит 
$$
\forall z,x \in M_* \colon h_*(z) - p(z -x_0) \leq p(x + x_0) -h_*(x)
$$
Взяв супремум по $x$ и $z$ получим в точности необходимое. Значит такое $a$ существует и теорема доказана.
\end{proof}
\begin{claim}
	Между $f \in X^*$ и $\Rea f: X \rr \R$ существует изометрия.
\end{claim}
\begin{proof}
Заметим, что если $f \in X^*$, то 
$$
f = U + iV, U = \Rea f, \ V = \Ima f
$$
Тогда в силу линейности легко видеть, что 
$$
f(x) = U(x) - iU(ix) = \Rea f(x) - i \Rea f(ix)
$$
Причем: 
$$
|U(x)| \leq |f(x)| \leq \|f\| \Rightarrow \|U\| \leq \|f\|
$$
С другой стороны:
$$
f(x) = |f(x)|e^{i \varphi} \Rightarrow |f(x)| = f(e^{-i \varphi} x ) = U(x e^{-i\varphi}) \leq \|U\| \|x e^{-i\varphi}\| = \|U\|\|x\| 
$$
Таким образом $\|f\| = \|\Rea f\|$. Значит существует изометрический изоморфизм:
$$
X^* \ni f \mapsto \Rea f: X \rr \R 
$$
\end{proof}
\begin{next0}
	Пусть $X$ --- ЛНП, $L \subset X$ --- подпространство. Пусть $g \in L^*$, тогда существует 
	$$
	f \in X^*\colon f\big|_L = g, \quad \|f\| = \|g\|
	$$
\end{next0}

\begin{lemma}\label{lem:f}
	Пусть $X \neq 0$, $x_0 \in X, \ x_0 \neq 0$, тогда существует линейный непрерывный функционал $f \in X^*$ такой, что 
	$$
	f(x_0) = 1 
	$$
\end{lemma}
\begin{proof}
	Рассмотрим 
	$$
	L_0 = \Lin{x_0} = \{t x_0 \mid t \in \Cx \} \subset X
	$$
	Построим  $f_0: L_0 \to \Cx$ следующим образом:
	$$
	\forall t \in \Cx: \ f_0(tx_0) = t 
	$$
	Тогда 
	$$
	\|f_0\| = \sup\limits_{ t \neq 0}\frac{\|f_0(t x_0)\|}{\|tx_0\|} = \sup\limits_{ t \neq 0}\frac{|t|}{|t| \|x_0\|} = \frac{1}{\|x_0\|} \leq + \infty
	$$
	Значит $f_0 \in \L(L_0, \Cx)$. В силу предыдущего утверждения имеем:
	$$
	p(x) = \frac{\|x\|}{\|x_0\|} = \|x\| \|f_0\| \leq \Rea f_0 (x) = U_0(x)
	$$
	Тогда по теореме Хана-Банаха: сущетсвует $U: X \rr \R$ --- продолжение $U_0$ на все пространство и 
	$$
	U(x) \leq p(x)
	$$
	Кроме того $\|U\| \geq \frac{1}{\|x_0\|} = \|U_0\|$. Значит $\|U_0\| =  \|U\|$. Тогда имеем:
	$$
	f = U(x) - iU(ix)
	$$
	Который удовлетворяет условию леммы.
\end{proof}

\begin{theorem}
	\hfill
	\begin{enumerate}
		\item Пусть $Y$ --- банахово пространство, тогда $(\CL(X,Y), \tau_U)$ --- полное
		\item Если $Y$ --- не полное, а $X \neq \{0\}$, то $(\CL(X,Y), \tau_U)$, $(\CL(X,Y), \tau_S)$ --- не полны.
	\end{enumerate}
\end{theorem}
\begin{proof}
	\hfill
	\begin{enumerate}
		\item Возьмем $\tau_U$ --- фундаментальную последовательность $\{A_n\} \subset \CL(X,Y)$, то есть:
		$$
		\forall \eps > 0 : \exists N(\eps): \ \forall n,m \geq N(\eps): \ \|A_n -A_m\| \leq \eps 
		$$
		Значит
		$$
		\forall x \in X: \exists N\left(\frac{\eps}{\|x\| +1}\right) :\ \|A_n(x) - A_m(x)\| \leq \|A_n - A_m\| \|x\| \leq \eps
		$$
		Значит для любого $x$ последовательность $\{A_n(x)\} \subset Y$ --- фундаментальна, тогда в силу полноты $Y$ она сходится.
		Тогда положим:
		$$
		T: X \to Y \quad T(x) = \lim\limits_{n \to \infty} A_n(x) \in Y
		$$
		В силу линейности предела и операторов $A_n$, $T$ --- линейный оператор. Покажем, что он непрерывен. Рассмотрим произвольное $x \in X, \|x\| \leq 1$ имеем:
		$$
		\|Tx\| = \lim\limits_{n \to \infty}{\|A_n x\|} \leq \lim\limits_{n \to \infty}\|A_n\|
		$$
		Кроме того по неравенству треугольника: 
		$$
		 \forall n,m \geq N(\eps): \left|\|A_n\| - \|A_m\|\right| \leq \|A_n - A_m\| \leq \eps
		$$
		Тогда $\{\|A_n\|\} \subset \R$ --- фундаментальная числовая последовательность, и в силу полноты $\R$ имеет предел, значит: 
		$$
		\|Tx\| \leq \lim\limits_{n \to \infty} \|A_n\| \leq \infty
		$$
		Таким образом оператор $T$ --- ограничен и значит непрерывен, тогда $T \in \CL(X, Y)$
		Теперь нам надо показать сходимость к $T$ по операторной норме. Опять пусть $x \in X: \ \|x\| \leq 1$, имеем:
		$$
		\forall n, m \geq N(\eps): \ \|A_n(x) - A_m(x)\| \leq \eps
		$$
		Переходим к пределу по $m$ и в силу непрерывности нормы получаем:
		$$
		\forall n \geq N(\eps): \ \|A_n(x) - T(x)\| \leq \eps 
		$$
		Теперь беря супремум по всем $x$ из шара получаем:
		$$
		\|A_n - T\|  \leq \sup\limits_{\|x\| \leq 1}\|A(x) - T(x)\| \leq \eps
		$$
		Тогда мы получили сходимость по операторной нормы и $\|T\| = \lim\limits_{n \to \infty}\|A_n\|$
		\item Пусть $Y$ --- неполно, значит существует $\{y_n\} \subset Y$ --- фундаментальная расходящаяся в $Y$ последовательность. $X \neq 0$, значит существует ненулевой вектор $x_0 \in X$, $x_0 \neq 0$ в силу леммы \ref{lem:f} существует $f\in X^*$ такой, что $f(x_0) = 1$. Тогда рассмотрим
		$$
		 \forall x \in X: \ A_n(x) = f(x)y_n
		$$
		$A_n: X \to Y$ --- линеен в силу линейности $f$, кроме того:
		$$
		\|A_n\| = \sup\limits_{\|x\| \leq 1}|f(x)| \|y_n\| = \|f\| \|y_n\|
		$$
		\textit{Отметим, что в равенстве выше присутствуют четыре различные нормы. Читателю, который пробегается по этому тексту по диагонали, рекомендуется обратить на это внимание и понять, из каких пространств все эти нормы.} Так как $f \in X^*$, то $\|f\|$ --- конечна, тогда $A_n \in \CL(X, Y)$. Далее воспользуемся фундаментальностью $\{y_n\}$:
		$$
		\|A_n - A_m\| = \sup\limits_{\|x\| \leq 1}|f(x)| \|y_n - y_m\| = \|f\| \|y_n - y_m\| \rrr 0
		$$
		Таким образом получили $\{A_n\} \subset \CL(X, Y)$ --- $\tau_U$-фундаментальна, а значит и $\tau_S$-фундаментальная. Так как $f(x_0) = 1$, то:
		$$
		A_n(x_0) = f(x_0) y_n = y_n
		$$ 
		Но $\{y_n\}$ является расходящейся, откуда следует, что $\{A_n\}$ расходится, а значит пространства $(\CL(X,Y), \tau_S)$, $(\CL(X,Y), \tau_U)$ не являются полными.
	\end{enumerate}
\end{proof}