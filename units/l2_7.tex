\newpage
\section{Свойства сопряженного оператора, теорема о разрешимости операторных уравнений.}
\begin{claim}
	Если $X,Y$ --- банаховы и $A \in \CL(X,Y)$ и $\Ima A^*$ --- замкнут в $X^*$, тогда $\Ima A$ --- замкнут в $Y$
\end{claim}
Заметим, что в условиях этого утверждения в силу \ref{th:frspirit} получим, что $\Ima A^* = (\Ker A)^\perp$ и $\Ima A = {}^\perp(\Ker A^*)$
\begin{proof}
	Рассмотрим
	$$
	Z = [\Ima A]_{\|\|} \subset Y
	$$
	$Z$ --- замкнутое подпространство банахова пространства, значит $Z$ --- само банахово. Определим
	$$
	T \colon X \to Z, \quad T(x)  = A(x) \forall x\in X 
	$$
	Ясно, что $T \in \CL(X,Z)$, кроме того 
	$$
	 \Ima T = \Ima A \Rightarrow [\Ima T]_{Z}  = Z 
	$$
	То есть $\Ima T$--- всюду плотен в $Z$. Рассмотрим сопряженный
	$$
	T^* \colon Z^* \to X^*
	$$
	Так как $\Ima T$ --- всюду плотен, то в силу теоремы Фредгольма (\ref{th:fr}) и следствия леммы \ref{lem:densyty} $\Ker T^* = (\Ima T)^\perp = \{0\}$. Значит $T^*$ --- инъективен, значит существует обратный оператор
	$$
	\exists (T^*)^{-1}\colon \Ima T^* \to Z^*
	$$
	Текущая картина
	$$
	T^* \colon Z^* \to X^* \quad A^* \colon Y^* \to X^* 
	$$
	Причем $\Ima A^*$ --- замкнут. Нужно понять, что из себя представляет $\Ima T^*$. 
	
	Рассмотрим произвольный $h \in Z^*$ по теореме Хана-Банаха 
	$$
	\exists g \in Y^*\colon g\big|_{Z}= h
	$$
	Тогда 
	$$
	\forall x \in X\colon h(T(x)) = (T^*h)(x)
	$$
	С другой стороны 
	$$
	h(T(x))  = h(\underbrace{A(x)}_{\in \Ima A})  = g(Ax)  = (A^*g)(x)
	$$
	Таким образом
	$$
	\forall x \in X T^*h(x) = A^*g(x) \Rightarrow T^*h = A^* g
	$$
	Таким образом $ \Ima T^* \subset \Im A^*$
	Но можно рассуждать и в обратную сторону. Возьмем $g \in Y^*$, тогда $\forall x \in X$
	$$
	g(Ax) = (A^*g)(x)
	$$
	Рассматривая сужение $h = g\big|_Z$
	$$
	g(Ax) = g(Tx) = h(Tx) = (T^*h)(x)
	$$
	Значит $\forall x \in X \colon T^*g = A^*h \Rightarrow \Ima A^* \subset \Ima T^*$. Получили, что
	$$
	\Ima A^* = \Ima T^*
	$$
	Так как оператор $T$ --- сужение оператора $A$, то этот результат несколько тавтологичен. Однако теперь мы можем утверждать, что $\Ima T^*$---замкнут в банаховом $X^*$, значит по теореме Банаха об обратном операторе (\ref{th:inv_op}) 
	$$
	(T^*)^{-1} \in \CL(\Ima T^*, Z^*) \quad 0< \|(T^*)^{-1}\| < \infty
	$$
	А теперь ФОКУС. Что можно сказать про прямой оператор $T$, если он обладает непрерывным обратным сопряженным?
	Оказывается, что если $(T^*)^* \in \CL(\Ima T^*, Z^*)$, то $T \colon X \to Z$ --- открытое отображение, то есть 
	$$
	\exists r>0 \colon T(O_1^X(0)) \supset O_r^Z(0)
	$$
	Предположив это, моментально получаем, что $\Ima T = Z$ и 
	$$
	[\Ima A]_{\|\|} = Z \supset \Ima A = \Ima T = Z 
	$$
	И получаем $\Ima A = [\Ima A]_{\|\|}$
	
	Докажем, что $T\colon X \to Z$ --- открытое отображение. Рассмотрим $[TO_1(0)]_Z$ --- замкнутое и выпуклое в $Z$ множество. Рассмотрим $z \in Z \setminus [TO_1(0)]_Z$. По следствию теоремы Хана-Банаха отделим $z$ от выпуклого замкнутого множества $[TO_1(0)]_Z$, получим 
	$$
	\exists f \in Z^*, \|f\| > 0 , \ \exists\gamma \in \R \colon\forall \|x\|\leq 1  \Rea f(T(x)) \leq \gamma < \Rea f(z)
	$$
	Взяв супремум по всем $x$ из единичного шара, получим
	$$
	\sup\limits_{\|x\| \leq 1} |\Rea f(T(x))| = 	\sup\limits_{\|x\| \leq 1} |\Rea (T^* f)(x)| = \|\Rea (T^* f)\| = \|T^* f\|\leq\gamma < \Rea f(z)
	$$
	Таким образом $T^* f \in X^*$. Кроме того 
	$$
	f = (T^*)^{-1}(T^*f) \Rightarrow \|f\| \leq \|(T^*)^{-1}\| \|T^*f\| 
	$$
	Значит
	$$
	\frac{\|f\|}{\|(T^*)^{-1}\|} \leq \|T^*f\|
	$$
	Теперь можем записать цепочку неравенств
	$$
	0 < \frac{\|f\|}{\|(T^*)^{-1}\|} \leq \gamma < \Rea f(z) \leq \|\Rea f\| \|z\| = \|f\| \|z\| \Rightarrow \|z\| > \frac{1}{\|(T^*)^{-1}\|} = k > 0
	$$
	Значит $z$ не может быть очень маленьким, точнее
	$$
	z \in Z \setminus B_{k}^Z(0) 
	$$
	Но $z \in Z \setminus [TO_1(0)]_Z$, значит мы получили
	$$
	O_k^Z(0) \subset B_{k}^Z(0)  \subset [TO_1(0)]_Z 
	$$
	Мы попали в ситуацию, аналогичную ситуации в доказательстве теоремы \ref{th:banachopenmap}. Так как $X$ --- полное, то, повторяя рассуждения из того доказательства 
	$$
	 [TO_1(0)]_Z  \subset T(B_2^X(0)) \subset T(O_X^3(0))
	$$
	Таким образом 
	$$
	O_{\frac{k}{3}}^Z(0) \subset TO_1^X(0) 
	$$
	То есть $T$--- открытое отображение! Доказательство окончено.
\end{proof}
\begin{claim}
	\hfill
	\begin{enumerate}
		\item Пусть $X,Y$ --- ЛНП и $A \in \CL(X,Y)$ таков, что $\exists A^{-1 }$. Тогда $\exists (A^*)^{-1} \in L(X^*, Y^*)$ при этом $(A^*)^{-1} = (A^{-1})^*$
		\item Пусть $X,Y$ --- ЛНП и $A^* \in \CL(Y^*, X^*)$ таков, что $\exists (A^*)^{-1}\colon X^* \to Y^*$, тогда $\exists A^{-1} \in \CL(\Ima A, X)$
	\end{enumerate}	
\end{claim}
\begin{proof}
	\hfil
	\begin{enumerate}
		\item Проверяется непосредственным вычислением операторов $(A^*)(A^{-1})^*$ и $(A^{-1})^*(A^*)$ по определению. 
		\item По теореме Фредгольма (\ref{th:fr})
		$$
		\Ker A = {}^\perp(\Ima A^*) = {}^\perp (X^*) = \{0\}
		$$
		И
		$$
		[\Ima A]_{\|\|} = {}^\perp(\Ker A^*) = {}^\perp\{0\}= Y
		$$
		Значит образ всюду плотен в $Y$. Пусть $x \in X$ по следствию теоремы Хана-Банаха
		$$
		\|Ax\| = \sup\limits_{g \in Y*, \|g\| \leq 1}|g(Ax)| = \sup\limits_{g \in Y*, \|g\| \leq 1} |(A^*g)(x)|
		$$
		Так как $(A^*)^{-1} \in \CL(X^*, Y^*)$, то в силу критерия топологической непрерывности $A^* \colon Y^* \to X^*$ --- открытое отображение. Тогда 
		$$
		A^*B_1^{Y^*}(0) \supset A^*O_1^{Y^*}(0) \supset O_r^{X^*}(0) \supset  B_\frac{r}{2}^{X^*}(0)
		$$
		Значит
		$$
		\sup\limits_{g \in Y*, \|g\| \leq 1} |(A^*g)(x)| = \sup\limits_{f \in A^*B_1^{Y^*}(0)} |f(x)| \geq \sup\limits_{f \in X^*, \|f\| \leq \frac{r}{2}} |f(x)| < \frac{r}{2}\|x\|
		$$
		То есть
		$$
		\forall x \in X \Rightarrow \|Ax\| \geq \frac{r}{2}\|x\|
		$$
		Но тогда
		$$
		\|A^{-1}Ax\| = \|x\| \leq \frac{2}{r}\|Ax\| \Rightarrow \|A^{-1}\| \leq \frac{2}{r}
		$$
		Это означает, что $A^{-1} \in \CL(\Ima A, X)$
	\end{enumerate}
\end{proof}
\begin{remark}
	Если во втором пункте $X$ --- полное, то $\Ima A$ --- замкнут и $A^{-1} \in \CL(Y,X)$
\end{remark}
\begin{theorem}[Первая теорема Фредгольма]
	\hfill
	
	Пусть $X$ --- банахово $A\in \CL(X)$ --- компактный оператор. Пусть $\lambda \in \Cx \setminus \{0\}$. Рассмотрим
	$$
	A_\lambda = A - \lambda I, \quad I \colon X \to X \text{ --- тождественный, } 
	$$
	Тогда
	\begin{enumerate}
		\item $\Ker A_\lambda$ --- конечномерен. 
		\item $\Ima A_\lambda$ --- замкнут. 
	\end{enumerate}

\end{theorem}
\begin{remark}
	$A_\lambda x = y$ называется уравнением Фредгольма. 
\end{remark}
\begin{proof}
	\hfill
	\begin{enumerate}
		\item Покажем, что из любой последовательности $\{x_n\} \subset \Ker A_\lambda, \|x_n\| \leq R$ можно выделить сходящуюся подпоследовательность $x_{n_k}\to x \in \Ker A_\lambda$. Имеем
		$$
		\begin{cases}
			Ax_n = \lambda x_n  \Leftrightarrow x_n = \frac{1}{\lambda}Ax_n\\
			AB_R(0) \text{ --- вполне ограниченно}
		\end{cases} 
		$$
		Тогда $ \exists n_1< n_2 < \dots$ $Ax_{n_k}$ --- фундаментальна, значит $x_{n_k}$ --- фундаментальна, что и требовалось.
		\item В силу предыдущего пункта $\Ker A_\lambda = \Lin \{e_1, \dots ,e_N\}$. То есть
		$$
		\forall x \in \Ker A_\lambda \Rightarrow x = \sum_{k=1}^N \alpha_k(x)e_k
		$$
		Продолжая Хану-Банаху функционалы $\alpha_k$ до $f_k \in X^*$ можно рассмотреть пересечение их ядер
		$$
		M = \bigcap_{k=1}^N\Ker f_k
		$$
		Ясно, что
		$$
		M \cap \Ker A_\lambda = \{0\}
		$$
		Кроме того, любой $x \in X$ представляется как сумма из $M$ и $\Ker A_\lambda$
		$$
		x = \underbrace{\sum_{k=1}^N f_k(x) e_k}_{\in \Ker A_\lambda} + \underbrace{\left(x - \sum_{k=1}^N f_k(x) e_k\right)}_{\in M}
		$$
Значит
		$$
		\Ker A_\lambda \oplus M = X
		$$
		Заметим, что $A_\lambda \colon M \to X$ --- инъективен, так как $X = \Ker A_\lambda \oplus M$, то $A_\lambda(M) = \Ima A_\lambda$, поэтому мы можем сузиться на подпространство $M$ и анализировать образ $\Ima A_\lambda$ на нем. 
		
		
		Пусть $\exists C > 0$:
		$$
		\forall x \in M\colon \|A_\lambda x\| \geq C\|x\| 
		$$
	Покажем в этом предположении замкнутость $\Ima A_\lambda$.
		$$
		\forall y \in [A_\lambda(M)]_X = [\Ima A_\lambda]_X \Rightarrow \begin{cases}
			\exists y_n = A_\lambda(x_n) \to y \\
			x_n \in M
		\end{cases}
		$$
		Рассмотрим $\{x_n\}_{n=1}^\infty$
		$$
		\|x_n - x_m\| \leq \frac{1}{C}\|A_\lambda(x_n - x_m)\| = \|y_n - y_m\| \to 0 
		$$
		Значит $\{x_n\}_{n=1}^\infty$ --- фундаментальна в банаховом пространстве $X$, то есть
		$$
		\exists x \in X \colon x_n \to x 
		$$
		Но тогда в силу непрерывности $A_\lambda$, $y = A_\lambda(x) \Rightarrow y \in \Ima A_\lambda$, что и требовалось. 
		
		Теперь покажем, что действительно $\exists C>0$:
		$$
		\forall x \in M\colon \|A_\lambda x\| \geq C\|x\| 
		$$
		Предположим противное, то есть 
		$$
		\forall C > 0 \exists x_C \in M\colon \|A_\lambda x_C\| < C\|x_C\|
		$$
		Тогда, как минимум, $x_C \neq 0$. Рассмотрим $C_n = \displaystyle\frac{1}{n}$ и $z_n = \displaystyle\frac{x_{\frac{1}{n}}}{\|x_{\frac{1}{n}}\|} \in M, \ \|z_n\| = 1$. По предположению 
		$$
		\|A_\lambda z_n \| < \frac{1}{n} 
		$$
		Вспомним, что $A$ --- компактный оператор, так как все $z_n$ --- лежат на сфере, то образ последовательности $z_n$ --- вполне ограничен, тогда 
		$$
		\exists n_1 < n_2 < \dots \Rightarrow \{A z_{n_k} \}_{k=1}^\infty\text{ --- фундаметальна в $X$}
		$$
		Но тогда последовательность $z_{n_k}$ будет фундаментальна как сумма фундаментальной и бесконечно малой последовательностей:
		$$
		z_{n_k} = \frac{A z_{n_k} - A_\lambda z_{n_k}}{\lambda}
		$$
		$X$ --- банахово, значит $\exists x \in X, \ z_{n_k} \to x$. Поймем какими свойствами обладает $x$. Так как $\|z_{n_k}\| = 1$, то $\|x\| = 1$. Так как $M$ --- замкнуто, то $x \in M$. Кроме того 
		$$
		\begin{cases}
			A_\lambda z_{n_k} \xrightarrow{k \to \infty} 0 \\
			A_\lambda z_{n_k} \xrightarrow{k \to \infty} A_\lambda x
		\end{cases} \Rightarrow A_\lambda x = 0 \Rightarrow x \in \Ker A_\lambda
		$$
		Получается $x \in M \cap \Ker A_\lambda$, но тогда $x = 0$, противоречие с $\|x\| = 1$. Таким образом теорема доказана. 
	\end{enumerate}
\end{proof}
