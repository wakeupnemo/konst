\newpage
\section{Топологически и секвенциально непрерывные отображения топологических пространств, связь между ними. Критерий топологической непрерывности отображения.}

\begin{definition}
	Пусть $(X, \tau_1)$, $(Y, \tau_2)$ --- топологические пространства. $f: (X,\tau_1) \rr (Y, \tau_2)$ --- отображение. Тогда $f$ называется \textbf{топологически непрерывным}, если 
	$$
	\forall x \in X \ \forall U(f(x)) \in \tau_2 \ \exists V(x) \in \tau_1: \ f(V(x)) \subset U(f(x))
	$$
	Отображение $f$ называется \textbf{секвенциально непрерывным}, если
	$$
	\forall x \in X: \ \forall \{x_n\}_{n=1}^{\infty} \subset X: x_n \stackrel{\tau_1}{\rr} x \Rightarrow f(x_n) \stackrel{\tau_1}{\rr} f(x)
	$$
\end{definition}

\begin{claim}
	Пусть $(X, \tau_1), (Y, \tau_2)$ --- топологические пространства $f: (X,\tau_1) \rr (Y, \tau_2)$--- отображение, тогда:
	\begin{enumerate}
		\item Если $f$--- топологически непрерывна, то $f$ секвенкциально непрерывно.
		\item Обратное неверно
		\item Если $(X, \tau_1)$ удовлетворяет \hyperlink{fcs}{аксиоме счетности}, то из секвенциальной непрерывности следует топологическая.
	\end{enumerate}
\end{claim}
\begin{proof}
\hfill 
\begin{enumerate}
	\item 
	Пусть $f: (X, \tau_1) \rr (Y, \tau_2)$, $f$ --- топологически непрерывна и $\{x_n\} \subset X$, $x_n \rr x$. В силу топологической непрерывности $f$ имеем:
	$$
	\forall U(f(x)) \in \tau_2 \exists V(x) \in \tau_1: \ f(V(x)) \subset U(f(x))
	$$
	В силу сходимости $x_n$ к $x$ имеем:
	$$
	\exists N \in \N: \ \forall n \geq N: x_n \in V(x)
	$$
	Тогда:
	$$
	\exists N \in \N: \ \forall n \geq N: f(x_n) \in U(f(x))
	$$
	Таким образом $f(x_n) \stackrel{\tau_2}{\rr} f(x)$. То есть $f$ --- секвенциально непрерывна.
	\item Пусть $X = Y = \R$, $\tau_2 = \tau_o$ --- обычная топология на прямой. $\tau_1 = \tau_z$ --- топология Зарисского. 
	$$
	\tau_z = \{G \subset \R \mid |\R \setminus G | \leq |\N| \}\cup \{\varnothing\} 
	$$
	Доказательство того, что действительно топология оставлю читателю.
	Базой обычной топологии на прямой являются интервалы, то есть 
	$$
	G \in \tau_0  \Leftrightarrow \forall y \in G \ \exists (a,b): y \in (a,b) \subset G
	$$ 
	Для начала покажем, что в топологии Зарисского любая сходящаяся последовательность является стационарной начиная с некоторого номера. Действительно: 
	$$
	\{x_n\} \subset \R: \ x_n \stackrel{\tau_z}{\rrr} x \Leftrightarrow \forall U(x) \in \tau_z :\  \exists N \in \N: \ \forall n \geq N: \ x_n \in U(x)
	$$
	Рассмотрим окрестность точки $x$: 
	$$
	U(x) = \R \setminus \{x_n \mid x_n \neq x\}
	$$ 
	Мы выкинули из последовательности $x_n$ все элементы, не совпадающие с $x$, и взяли дополнение этого множества. Ясно, что оно не более чем счетно, тогда начиная с некоторого номера все элементы последовательности должны лежать в этой окрестности, но все элементы последовательности не совпадающие с $x$ выкинуты из нее, таким образом:
	$$
	\exists N : \ \forall n \geq N :\ x_n = x
	$$
	Теперь рассмотрим произвольное отображение $f: (\R, \tau_z) \rr (\R, \tau_o)$, покажем, что оно является секвенциально непрерывным. Действительно:
	$$
	x_n \stackrel{\tau_z}{\rrr} x \Rightarrow \exists N: \forall n \geq N: \ x_n = x \ \Rightarrow f(x_n) = f(x) \Rightarrow f(x_n) \stackrel{\tau_o}{\rrr} f(x)
	$$
	Таким образом любое отображение секвенциально непрерывно, но рассмотрим отображение:
	$$
	f: (\R, \tau_z) \rr (\R, \tau_o) :\ f(x) = x
	$$
	Имеем:
	$$
	\forall x \in \R: \exists (a,b) \subset \R :\ f(x) = x \in (a,b)
	$$
	Для того, чтобы отображение было топологически непрерывным, нам бы хотелось найти окрестность $x$ $V(x) \in \tau_z$, чтобы ее образ  попал в интервал $(a,b)$: 
	$$
	f(V(x)) = V(x) \stackrel{?}{\subset} (a,b)
	$$ 
	Поймем, что такого произойти не может, действительно, пусть такая окрестность $V(x)$ нашлась, тогда
	$$
	\R \setminus (a,b) \subset \R \setminus V(x) 
	$$
	Но слева стоит множество мощности континуум, а справа стоит не более чем счетное множество, так как $V(x)$ не пусто, получаем противоречие. Значит отображение $f(x) = x$ не является топологически непрерывным, являясь при этом секвенциально непрерывным. 
	\item Пусть для $(X, \tau_1)$ верна аксиома счетности и $f : (X, \tau_1) \rr (Y, \tau_2)$ --- секвенциально непрерывно. Предположим, что $f$ не является топологически непрерывным. То есть 
	$$
	\exists x \in X \ \exists U(f(x)) \in \tau_2: \ \forall V(x): \ f(U(x)) \nsubseteq U(f(x))	$$
	В силу аксиомы счетности для этой точки $x$ существует счетная локальная база. $\{W_n\}_{n=1}^{\infty}$ ($W_1 \supset W_2 \supset \dots $). Тогда из утверждения выше имеем:
	$$
	\forall n \geq N : f(W_n) \nsubseteq U(f(x)) \Rightarrow \exists x_n \in W_n: f(x_n) \notin U(f(x))
	$$
	Тогда построена последовательность $x_n$, которая в силу свойств локальной базы сходится к $x$ по топологии $\tau_1$, но в силу секвенциальной непрерывности $f$ это должно влечь: $f(x_n) \stackrel{\tau_2}{\rrr} f(x)$, но $f(x_n) \notin U(f(x))$, противоречие. Таким образом $f$ --- топологически непрерывна.
\end{enumerate}
\end{proof}
\begin{definition}
	Пусть $X,Y$ --- множества и $f: X \rr Y$ --- отображение. $ S \subset Y$, тогда прообразом $S$ относительно $f$ называется:
	$$
	f^{-1}(S) := \{ x \in X \mid f(x) \in S\}
	$$
\end{definition}
\begin{theorem}[Критерий топологической непрерывности]
	\label{th:ctc}
	Пусть $f: (X, \tau_1)\rr(Y,\tau_2)$. Тогда следующие утверждения эквивалентны:
	\begin{enumerate}
		\item $f$ --- топологически непрерывна. 
		\item Прообраз любого открытого множества относительно $f$ открыт.
		$$
		\forall G \in \tau_2: \ f^{-1}(G) \in \tau_1
		$$ 
		\item Прообраз любого замкнутого множества относительно $f$ замкнут. 
		$$
		\forall G \text{ --- $\tau_2$-замкнуто}: \ f^{-1}(G) \text{ --- $\tau_1$-замкнуто}
		$$
	\end{enumerate} 
	
\end{theorem}
\begin{proof}
	\hfill
	\begin{enumerate}
		\item[(2)$\Rightarrow$(3)] $S \subset Y$ --- $\tau_2$-замкнуто $\Leftrightarrow Y \setminus S \in \tau_2$ По условию имеем:
		$$\Rightarrow f^{-1}(Y\setminus S) \in \tau_1$$
		Тогда по свойствам прообраза:
		$$
		f^{-1}(Y\setminus S) = Y\setminus f^{-1}(S) \in \tau_1
		$$
		Значит прообраз замкнут. 
		\item[(3)$\Rightarrow$(2)] Доказывается аналогично.
		\item [(1)$\Rightarrow$(2)] Пусть $f$ --- топологически непрерывно, тогда: 
		$$
		\forall G \in \tau_2, \forall x \in f^{-1}(G) \text{ (считаем что непусто)} \Rightarrow f(x) \in G \in \tau_2
		$$
		Значит $G$ --- окрестность $f(x)$ в $(Y, \tau_2)$, следовательно из определения топологической неперывности $f$:
		$$
		\exists U(x) \in \tau_1: \ f(U(x)) \subset G \Rightarrow U(x) \subset f^{-1}(G)
		$$
		Значит 
		$$
		f^{-1}(G)= \bigcup_{x \in f^{-1}(G)} \{x\} \subset \bigcup_{x \in f^{-1}(G)} U(x) \subset f^{-1}(G)
		$$
		Тогда так как $U(x) \in \tau_1$
		$$
		f^{-1}(G) = \bigcup_{x \in f^{-1}(G)} U(x) \in \tau_1
		$$
		Фактически строчки выше это доказательство того, что множество открыто если любая его точка содержится с некоторой окрестностью. Таким образом прообраз любого открытого открыт. 
		\item[(2)$\Rightarrow$(1)] Пусть для любого открытого, прообраз открыт. Докажем по определению топологическую непрерывность $f$. 
		$$
		\forall U(f(x)) \in \tau_2 \ \exists V(x) = f^{-1}(U(f(x))) \ni x: \ V(x) \subset U(f(x))
		$$
	\end{enumerate}
\end{proof}

