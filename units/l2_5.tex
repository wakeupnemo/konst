\newpage
\section{Теорема Банаха о замкнутом графике, сопряженный оператор}

\begin{theorem}[Банаха о замкнутом графике]
	 Пусть $X,Y$ --- банаховы пространства. И $A\colon X \to Y$ --- линейный оператор с замкнутым графиком, то есть 
	 $$
	 \Gr A = \left\{\left(\begin{matrix}
	 	x \\
	 	Ax
	 \end{matrix}\right) \in X \times Y \mid x \in X \right\} \text{ --- замкнуто в } X \times Y
	 $$
	 Тогда $A \in \CL(X, Y)$
\end{theorem}
\begin{proof}
	Рассмотрим оператор $T : \Gr A \to X$:
	$$
	\forall x \in X \quad T \left(\begin{matrix}
		x \\ Ax
	\end{matrix}\right) = x 
	$$
	Очевидно, $T$ --- линеен. Покажем, что $\Ker T =\{0\}$. Пусть $ \left(\begin{matrix} x \\ Ax \end{matrix}\right) \in \Ker T$ тогда
	$$
	T \left(\begin{matrix}
		x \\ Ax
	\end{matrix}\right) = x = 0 \Rightarrow Ax = A(0) = 0 \Rightarrow \left(\begin{matrix}
	x \\ Ax
\end{matrix}\right) = \left(\begin{matrix}
0 \\ 0
\end{matrix}\right)
	$$ 
	Кроме того ясно, что $\Ima T = X$. Проверим, что $T \in \CL(X,Y)$ 
	$$
	\left\| T\left(\begin{matrix}
		x \\ Ax
	\end{matrix}\right) \right\| = \|x\| \leq \|x\| + \|Ax\| = \left\| \left(\begin{matrix}
	x \\ Ax
\end{matrix}\right)\right\| \Rightarrow \|T\| \leq 1
	$$ Для попадания в условие теоремы Банаха об обратном операторе (\ref{th:inv_op}), остается проверить, что $\Gr A$ --- банахово, но $\Gr A$ --- замкнуто в банаховом $X \times Y$, а значит --- банахово. Таким образом существует обратный оператор $$\exists T^{-1} : X \to \Gr A \text{ --- непрерывен}$$
	Рассмотрим 
	$$
	P \colon X \times Y \to Y \quad P \left(\begin{matrix}
		x \\ y
	\end{matrix}\right) = y
	$$
	$ P \in \CL(X \times Y, Y)$, действительно 
	$$
	\left\|P \left(\begin{matrix}
		x \\ y
	\end{matrix}\right)\right\| = \|y\| \leq \|x\| + \|y\| = \left\| \left(\begin{matrix}
	x \\ y
\end{matrix}\right)\right\| \Rightarrow \|P\| \leq 1 
	$$
	Тогда рассмотрим $ P \circ T^{-1} \colon X \to Y$, $\forall x \in X$ имеем
	$$
	PT^{-1}(x) = P \left(\begin{matrix}
		x \\ Ax
	\end{matrix}\right) = Ax
	$$
	Таким образом $A = P \circ T^{-1}$, значит $A \in \CL(X, Y)$ как композиция непрерывных. 
\end{proof}
\begin{example}
	Пусть $X$ --- банахово, и $M,N$ --- два его замкнутых подпространство такие что
	$$
	\begin{cases}
		N \cap M - \{0\} \\
		M + N = X
	\end{cases} \Leftrightarrow M \oplus N = X
	$$
	Тогда
	$$
	\forall x \in X \ \exists ! y(x) \in M , z(x) \in N: \ y(x) + z(x) = x
	$$
	Определим линейные операторы
	$$
	P_M(x) = y(x), \quad P_N(x) = z(x)
	$$
	Докажем, что эти операторы непрерывны. 
	$$
	\Gr P_M = \left\{\left(\begin{matrix}
		x \\ y(x)
	\end{matrix}\right)\mid x \in X\right\} \subset X \times X
	$$
	Покажем, что он замкнут. Пусть $ \{x_n\} \subset X$,  $y(x_n) = y_n$, $ \left(\begin{matrix}
		x_n \\ y_n
	\end{matrix}\right) \in \Gr P_M$ и $\left(\begin{matrix}
	x_n \\ y_n
\end{matrix}\right) \to \left(\begin{matrix}
x \\ y
\end{matrix}\right)$, то есть
$$
\|x_n - x\| \to 0, \quad \|y_n - y\| \to 0
$$
Но по определению 
$$
x_n = y_n + z_n
$$
Причем $x_n \to x$, $y_n \to y$, тогда $z_n = x_n - y_n \to z = x-y$ причем, разность $x_n - y_n \in N$ и так как $N$ --- замкнуто, значит $z \in N$ таким образом 
$$
x = y + z \Rightarrow y = y(x) \in M,  \quad z = z(x) \in N
$$
То есть $\left(\begin{matrix}
	x \\ y
\end{matrix}\right)\in \Gr P_M$. Значит график замкнут, и по теореме Банаха о замкнутом графике $\|P_M\| \leq \infty$.
\end{example}
\begin{definition}
	Пусть $X,Y$ --- линейные нормированные пространства. $A \in \CL(X,Y)$. Сопряженный оператор действует $A^*\colon Y^* \to X^*$
\[\begin{tikzcd}[ampersand replacement=\&]
	X \& Y \\
	{X^*} \& {Y^*}
	\arrow["A", from=1-1, to=1-2]
	\arrow[from=1-1, to=2-1]
	\arrow[from=1-2, to=2-2]
	\arrow["{A^*}", from=2-2, to=2-1]
\end{tikzcd}\]
По формуле $$\forall g \in Y^*:  \quad A^*g = gA$$
\end{definition}
\begin{remark}
	Существенно, что сопряженный оператор определяется для непрерывного оператора, так как суперпозиция $g \circ A$ должна лежать в $X^*$, это достигается именно непрерывностью $A$.
\end{remark}
$A^*$ очевидно линеен, найдем его норму.
$$
\|A^*g\| = \sup\limits_{\|x\| \leq 1} |(A^*g)(x)|= \sup\limits_{\|x\| \leq 1}|g(Ax)|  \leq \sup\limits_{\|x\| \leq 1}\|g\|\|Ax\| = \|g\| \|A\| \Rightarrow \|A^*\| \leq \|A\|
$$
Справедливо и обратное неравенство. По следствию теоремы Хана-Банаха
$$
\|A(x)\| = \sup\limits_{\substack{\|g\| \leq 1 \\ g \in Y^* }} |g(Ax)| = \sup\limits_{\substack{\|g\| \leq 1 \\ g \in Y^* }} \|(A^*g)(x)\| \leq  \sup\limits_{\substack{\|g\| \leq 1 \\ g \in Y^* }} \|A^*g\|\|x\| = \|A^*\| \|x\| \Rightarrow \|A\| \leq \|A^*\| 
$$
Таким образом $\|A\| = \|A^*\|$. 

\noindent \textbf{Конструктивное определение.}

\noindentПусть $X^*$ изометрически измоморфно $M$, $Y^*$ изометрически изоморфно $N$, то есть
$$
\varphi\colon X^* \to M, \quad \psi\colon Y^* \to N \text{ --- изоморфизмы} 
$$
И действие функционала описывается в терминах некоторого произведения
$$
f \in X^* \Rightarrow f(x) = \langle \varphi(f), x\rangle, \quad g \in Y^* \Rightarrow g(y) = \langle \psi(g), y\rangle
$$
Тогда определим $A^+ \colon N \to M$ По формуле $$
A^+ = \varphi \circ A^* \circ \psi^{-1}\colon N \to M
$$ 
Картинка для пояснения (а кому и для определения)
\[\begin{tikzcd}[ampersand replacement=\&]
	X \& {X^*} \& M \\
	Y \& {Y^*} \& N
	\arrow[from=1-1, to=1-2]
	\arrow["\varphi", from=1-2, to=1-3]
	\arrow[from=2-1, to=2-2]
	\arrow["\psi", from=2-2, to=2-3]
	\arrow["A"', from=1-1, to=2-1]
	\arrow["{A^+}"', from=2-3, to=1-3]
	\arrow["{A^*}", from=2-2, to=1-2]
\end{tikzcd}\]
\begin{example}
	Пусть $A\colon l_1 \to L_1[01]$ и 
	$$
	(Ax)(t) = \sum_{k=1}^{\infty} x(k)t^k
	$$
	Ясно, что $A$ --- непрерывен:
	$$
	\|Ax\|_1 \leq \sum_{k=1}^\infty |x(k)| = \|x\|_1 \Rightarrow \|A\| \leq 1
	$$
	Мы знаем, что $l_1^*$ изометрически изоморфно $l_\infty$, а $(L_1[01])^*$ изометрически изоморфно $L_\infty[01]$, причем функционалы действуют как 
	$$
	f \in l_1^{*},   z = \varphi(f),   f(x) = \sum_{k=1}^\infty z(k)x(k) \quad g \in  (L_1[01])^*,  w = \psi(g), g(y) = \int\limits_{0}^1 w(t)y(t)dt
	$$
	Тогда $A^+: L_\infty[01] \to l_\infty$. Найдем его:
	$
	\forall x \in L_1[01], \forall w \in L_\infty[01]
	$ имеем
	$$
	\langle w, Ax \rangle \int\limits_{0}^1 w(t)(Ax)(t)dt =  \int\limits_{0}^1 w(t) \left(
	\sum_{k=1}^\infty x(k) t^k\right)dt =  \int\limits_{0}^1 \lim\limits_{N \to \infty}\left(\sum_{k=1}^N x(k)w(t)t^k\right)dt
	$$
	Проверим условие теоремы Лебега об ограниченной сходимости
	$$
	\left|\sum_{k=1}^N x(k)w(t)t^k\right| \leq_{\text{п. в. $t\in[01]$}} \|w\|_\infty\|x\| \equiv h(t) \in L_1[01]
	$$
	Значит можно переставить местами предел и интеграл:
	$$
	\langle w, Ax \rangle = \sum_{k=1}^\infty x(k)\int\limits_{0}^1 w(t) t^k dt = \langle z, x\rangle, \ z(k) = \int\limits_{0}^1  w(t)t^k dt, \ z \in l_\infty
	$$
	Таким образом 
	$$
	(A^+ w)(k) = \int_{0}^{1}w(t)t^k dt
	$$
\end{example}
Теперь обсудим Эрмитово сопряжение для $L(H)$
$$
A \in L(H) \Rightarrow A^* \in L (H^*)
$$
\begin{theorem}[Риса, Фреше]
	Пусть $H$ --- гильбертово пространство, тогда $$\forall g \in H^* \quad  \exists! z = z(f) \in H$$ Причем $$\|z\| = \|f\|, \quad \forall x \in H\colon f(x) = (x,z) $$
	При этом отображение $z$ обладает следующими свойствами
	\begin{itemize}
		\item $\forall f,g \in H^*: \ z(f + g) = z(f) + z(g)$
		\item $\forall \lambda \in \Cx: z(\lambda f) = \bar{\lambda} z(f)$
	\end{itemize}
\end{theorem}
\begin{definition}
	Изометрия $z$ не является обычным изоморфизмом, так как скаляры выносятся с сопряжением. Это изометрия является сопряженно-линейной аддитивной изометрией, называется  \textit{изометрией Риса-Фреше} и обозначается $\Phi : H^* \to H$ и обладает следующим свойством 
	$$
	\forall f \in H^* \Rightarrow \forall x \in H \colon f(x) = (x, \Phi(f))
	$$
\end{definition}
\begin{definition}
	Пусть $A \in \CL(H)$, $H$ --- гильбертово, эрмитово сопряженный оператор $A^+ \in L(H)$ определяется как 
	$$
	A^+  =\Phi \circ A^* \circ \Phi^{-1}
	$$
\end{definition}
\begin{remark}
	Ясно что эрмитово сопряженный оператор удовлетворяет свойству:
$$(Ax, y) = (\Phi^{-1}(y))(Ax) = (A^*\Phi^{-1}y)(x) = (x, \Phi A^* \Phi^{-1}y) = (x,A^+y)$$
\end{remark}
\begin{definition}
	$A \in L(H)$ называется эрмитовым или самосопряженным по Эрмиту если $A^+ = A$
\end{definition}

\begin{proof}[Доказательство теоремы Риса-Фреше]
	Рассмотрим $f \in H^*$, если $ f = 0 \Rightarrow z(f) = 0$ подойдет. Если $f \neq 0$, тогда $\Ker f \neq H$, тогда $\Ker f$ --- замкнутое подпространстве в $H$. Тогда в силу теоремы Риса об ортогональном дополнении можно рассмотреть
	$$
	\Ker f \oplus (\Ker f)^\perp  = H
	$$
	Рассмотрим $x_0 \in (\Ker f)^\perp \setminus \{0\}$. Так как $f(x_0) \neq 0$, то 
	$$
	\forall x \in H\Rightarrow x =\frac{f(x)}{f(x_0)}x_0 + \left(x - \frac{f(x)}{f(x_0)}x_0\right)
	$$
	Тогда 
	$$
	(x,x_0) = \frac{f(x)}{f(x_0)}(x_0,x_0) \Rightarrow  \forall x \in H \colon f(x) = \left(x, \frac{\overline{f(x_0)}}{(x_0,x_0)}x_0\right)
	$$
	Значит 
	$$
	z(f) = \frac{\overline{f(x_0)}}{(x_0,x_0)}x_0
	$$
	В силу неравенства Коши-Буняковского $\|f\| \leq \|z\|$ с другой стороны
	$$
	|z(f)| \leq \frac{|f(x_0)|}{\|x_0\|} \leq \|f\|
	$$
	Таким образом $z$ --- изометрия. Аддитивность и сопряженная однородность вытекает из формул. Таким образом теорема доказана.
\end{proof}