\newpage
\section{Метризуемость слабой* топологии}
Далее рассматриваем только локально выпуклые ТВП. 


\begin{claim}[воспоминание]
	Если $X$ --- бесконечномерне банахово пространство, то в $(X, \|\|)$ нет счетного базиса Гамеля
\end{claim}
\begin{proof}
	Является простым следствием теоремы Бэра \ref{th:bear} и замкнутости конечномерного подпространства \ref{th:clfinds}
\end{proof}
\begin{claim}
	Пусть $\tau_{w^*}$ в $X^*$ является метризуемой. Тогда в $X$ есть счетный базис Гамеля.
\end{claim}
\begin{proof}
	Предположим, что $\tau_{w^*}$ --- метрическая, пусть $\rho_*$ --- метрика, тогда, существует счетная локальная база:
	$$
	\beta = \left\{O_{\frac{1}{n}}^{\rho_*}(0)\right\}_{n = 1}^\infty
	$$
	Обладая такой локальной базой, мы понимаем, что в любой такой шар ноль входит как $\tau_{w^*}$ внутренняя точка, значит 
	$$
	\forall n \in \N \ \exists x^{(n)}_1, \dots, x^{(n)}_{N_n} \in X \colon \bigcap_{k = 1}^{N_n} V_*\left(0, x_k^{(n)}, \eps_n\right)\subset O_{\frac{1}{n}}^{\rho_*}(0)
	$$
	Далее будем рассуждать как при доказательстве теоремы Шмульяна (\ref{th:shulian}). Рассматриваем 
	$$
	M = \left\{\left\{x_k^{(n)}\right\}_{k=1}^{N_n} \mid n \in \N\right\}
	$$
	Далее мы покажем, что $\Lin M = X$, тогда так как $M$ --- не более чем счетно, выделяя из $M$ максимальную систему линейно независимых векторов (Лемма Цорна (\ref{lem:zorn}) и бла-бла) получим счетный базис Гамеля. 
	
	Пусть $x \in X$. Берем окрестность нуля $U_*(0) = V_*(0,x,1)$. Эта окрестность нуля содержит элемент счетной локальной базы, который в свою очередь содержит пересечение стандартных элементов базы $\tau_{w^*}$
	$$
	\exists n: \  O_{\frac{1}{n}}^{\rho_*}(0) \supset \bigcap_{k = 1}^{N_n} V_*\left(0, x_k^{(n)}, \eps_n\right)
	$$
	По этим векторам и разложится $x$. Рассмотрим каноническое отображение 
	$$(Fx) \in (X^*, \tau_{w^*})^*\colon (Fx)(f) = f(x)$$
	Тогда если мы возьмем функционал 
	$$
	f \in \bigcap_{k=1}^{N_n} \Ker \left(Fx_k^{(n)}\right)
	$$
	то $f$ автоматически попадает в $\bigcap_{k = 1}^{N_n} V_*(0, x_k^{(n)}, \eps_n)$, откуда $f$ попадает в шар $O_{\frac{1}{n}}^{\rho_*}(0)$, но пересечение ядер является подпространством, значит туда же попадет $t f$ для любого скаляра $t \in \Cx$, значит $f \in \Ker Fx$. То есть 
	$$
	 \bigcap_{k=1}^{N_n} \Ker \left(Fx_k^{(n)}\right) \subset \Ker Fx
	$$
	Далее рассуждения дословно повторяют доказательство (\ref{th:shulian}). Получаем, что $x$ раскладывается по векторам $x_{k}^{(n)}$, что и требовалось.
\end{proof}
\begin{theorem}
	Слабая* топология в банаховых пространствах не метризуема.
\end{theorem}
\begin{proof}
	Тривиальное следствие двух предыдущих утверждений. 
\end{proof}

\begin{claim}
	Пусть $(X, \|\|)$ бесконечномерное линейное нормированное пространство, при этом сепарабельное. Пусть $ R > 0$ рассмотрим шар в сопряженном пространстве 
	$$
	B_R^*(0) = \{f \in X^* \mid \|f\| \leq R\}
	$$
	Рассмотрим $\tau_{w^*}(R) = \{G \cap B_R^*(0) \mid G \in \tau_{w^*}\}$, тогда $(B_R^*(0), \tau_{w^*}(R))$ --- метрическое пространство.
\end{claim}
\begin{proof}
	В силу сепарабельности $\exists \{x_n\}_n^\infty$ --- счетное всюду плотное множество на $1$-сфере. Тогда рассмотрим метрику 
	$$
	\rho_*(f,g) = \sum_{n=1}^{\infty} \frac{|(f-g)(x_n)|}{2^n}
	$$
	Аксиомы метрики очевидны. Рассмотрим 
	$$
	\tau_{\rho_*}(R) = \{G \cap B_R^*(0) \mid G \in \tau_{\rho_*}\}
	$$
	Покажем, что эта топология совпадает с $\tau_{w^*}$.
	\begin{enumerate}
		\item[$\tau_{w^*} \subset \tau_{\rho_*}$] Возьмем $V_*(f,x,\eps)$ --- элемент базы $\tau_{w^*}$. Нужно показать, что 
		$$
		V_*(f,x,\eps) \cap B_R^*(0)  \in \tau_{\rho_*}
		$$
		$x \neq 0$, поэтому 
		$$
		g \in V_*(f,x,\eps) \Leftrightarrow |g(x) - f(x)| < \eps \Leftrightarrow |g(y) - f(y)| < \frac{\eps}{\|x\||} = \delta \Leftrightarrow g \in V_*(f,y,\delta)
		$$
		Поэтому будем рассматривать $y$ из $1$-сферы пространства $X$. $\forall \delta > 0$. Пусть 
		$$
		g\in V_*(f,g,\delta) \cap B_R^*(0) 
		$$
		Нужно вложить окрестность выше в $O_r^{\rho_*}(g) \cap B_R^*(0)$ для некоторого $r$. Возьмем 
		$$
		\gamma = \delta - |f(y) - g(y)| > 0
		$$
		Пусть $h \in O_r^{\rho_*}(g) \cap B_R^*(0)$, тогда $\forall n$:
		$$
		|(g-h)(x_n)| < r 2^n
		$$
		Тогда 
		$$
		|h(y) - g(y) | \leq |h(y) - h(x_n)| + |h(x_n) - g(x_n)| + |g(x_n) - g(y)| \leq 2R\|y - x_n\| + r2^n 
		$$
		Так как $\{x_n\}_{n=1}^\infty$ всюду плотно на сфере, тогда выбирая $n$ можем добиться, чтобы $2R\|y - x_n\| \leq \frac{\gamma}{2}$. При выбранном $n$, выбирая $r$ добиваемся, чтобы $r2^n \leq \frac{\gamma}{2}$, тогда
		$$
		|h(y) - g(y)| \leq \gamma
		$$
		Тогда $g \in  O_r^{\rho_*}(g) \cap B_R^*(0)$, что и требовалось.
		\item[$\tau_{w^*} \supset \tau_{\rho_*}$] Пусть теперь $ g \in O_r^{\rho_*}(f) \cap B_R^*(0)$, тогда 
		$$
		O_\gamma^{\rho_*}(g) \cap B_R^*(0) \subset O_r^{\rho_*}(f) \cap B_R^*(0)
		$$
		Где $\gamma = r - \rho_*(f,g) > 0$. Попробуем впихнуть этот шар в окрестность
		$$
		\bigcap_{k = 1}^N V_*(g, x_k, \eps) \cap B_R^*(0)
		$$
		Нужно выбрать $N$ и $\eps$. Возьмем $h \in \bigcap_{k = 1}^N V_*(g, x_k, \eps) \cap B_R^*(0)$, тогда 
		$$
		\begin{cases}
			|h(x_k) - g(x_k) < \eps  \ \forall k \in \overline{1,N}\\
			\|h\|, \|g\| \leq R
		\end{cases}
		$$
		Оценим расстояние от $h$ до $g$:
		$$
		\rho_*(g,h) < \sum_{k=1}^N\frac{\eps}{2^k} + \sum_{k = N+1}^\infty \frac{2R\|x_k\|}{2^k} = [\|x_k\| = 1] =  \sum_{k=1}^N\frac{\eps}{2^k} + \frac{2R}{2^N} < \eps + \frac{2R}{2^N}
		$$
		Теперь выбирая $\eps$ и $N$ так, чтобы $ \eps + \frac{2R}{2^N} < r$, получим, что 
		$$
		g \in \bigcap_{k = 1}^N V_*(g, x_k, \eps) \cap B_R^*(0)
		$$
		Что и требовалось.
	\end{enumerate}
\end{proof}
\begin{next0}
	В силу теоремы Банаха-Алаоглу (\ref{th:banach-alaoglu}) $B_R^*(0)$ является $\tau_{w^*}$ компактом в $(X^*, \tau_{w^*})$, но в силу доказанной теоремы, эта топология метрическая, тогда $B_R^*(0)$ слабый* секвенциальный компакт.
\end{next0}
\begin{next0}
	В условиях теоремы из любой сильно ограниченной в $X^*$ можно выбрать слабо* сходящуюся подпоследовательность. 
\end{next0}
