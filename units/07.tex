\newpage
\section{Компактные подмножества топологического пространства. Теорема Александера о предбазе.}

\begin{definition}
	$K \subset X$ называется топологическим компактом в $(X, \tau)$ если для любого $\tau$- покрытия множества $K$ существует его конечное подпокрытие. Если $X = K$, то $(X, \tau)$ называют компактным топологическим пространством
\end{definition}
\begin{theorem}[Александера о предбазе]
	\label{th:asb}
	Пусть $(X, \tau)$ --- топологическое пространство, $\sigma$ --- предбаза (\ref{subbase}) и $K \subset X$ такое, что любое $\sigma$-покрытие $K$ имеет конечное подпокрытие, тогда $K$ --- топологический компакт.
\end{theorem}
\begin{proof}
	Предположим, что $K$ --- не является топологическим компактом. Тогда обозначим:
	$$
	F = \{P \subset \tau \mid P \text{ --- покрытие $K$ не имеющее конечного подпокрытия}\}
	$$
	В силу нашего предположения $F \neq \varnothing$. Упорядочим $F$ относительно вложения, получим ЧУМ $(F, \subset)$. Применим теорему Хаусдорфа о максимальности (\ref{th:Hausdorf}), тогда в данном чуме существует максимальный ЛУМ $L$. Рассмотрим (по классике):
	$$
	P_L = \bigcup_{P \in L }P
	$$
	Так как $\forall P \in L \Rightarrow P$ --- покрытие $K$, то $P_L$ --- тоже $\tau$-покрытие $K$. Причем $P_L$ не имеет конечного подпокрытия. Действительно. Предположив противное рассмотрим полученное конечное подпокрытие:
	$$
	V_1, \dots V_N P_L\
	$$
	Так как $P_L$ --- лум, то найдется $P_o \in P_L$, которое содержит все $V_1, \dots V_N$, тогда получаем противоречие с тем, что $P_o$ не имеет конечного подпокрытия. Таким образом $P_K \in F$. Заметим что $P_L$ обладает необычным свойством:
	$$
	\forall V \in \tau \setminus P_L \Rightarrow P_L \cup \{V\} \text{ --- $\tau$-покрытие $K$ имеющее конечное подпокрытие.}
	$$
	Так как, если $P_L \cup \{V\} \in F$, то $L$ --- не максимальный лум, действительно рассмотрим $L_V = L \cup \{P_L \cup \{V\}\}$ --- это будет лум в $F$ и он больше $L$, противоречие. 
	Таким образом 
	$$
	\forall V \in \tau \setminus P_L \ \exists V_1, \dots, V_N \in P_L: \ V_1 \cup \dots \cup V_N \cup V \supset K
	$$
	Теперь когда мы уже покушали говна, можно приступать к использованию предбазы. Рассмотрим 
	$$
	P_\sigma = P_L \cap \sigma
	$$
	То есть мы выбираем из $P_L$ элементы предбазы. По условию любое $\sigma$-покрытие имеет конечное подпокрытие, тогда $P_\sigma$ --- не является покрытием $K$. Тогда в зазоре найдется точка:
	$$
	\exists x \in K: x \notin \bigcup_{V \in P_\sigma} V
	$$
	C другой стороны $P_L$ --- покрытие $K$, значит $\exists V \in P_L: \ x \in V$. $V$ --- открытое множество, вспоминая определение предбазы имеем:
	$$
	\exists W_1, \dots W_N \in \sigma:  \  x \in \bigcap_{k=1}^N W_k \subset V
	$$
	Заметим, что $W_k$ не лежат в $P_L$: 
	$$
	W_k \in \sigma, W_k \notin P_\sigma \Rightarrow W_k \notin P_L
	$$
	Тогда вспоминаем удивительное свойство $P_L$: 
	$$
	\forall k \in 1,N : P_L\cup W_k \text{ --- имеет конечное подпокрытие.}
	$$
	То есть для каждого $k$ найдутся множества $V_i^{(k)} \in P_L$:
	$$
	V_1^{(k)}, \dots V_{M_k}^{(k)}: K \subset \bigcup_{i=1}^{M_k}V_i^{(k)} \cup W_k 
	$$
	Есть два \sout{гендера} вида точек из $K$: покрывающиеся объединением $V_{i}$ или покрывающиеся $W_k$, тогда соорудим покрытие $K$:
	$$
	K \subset \left(\bigcup_{k = 1}^N \bigcup_{i =1}^{M_k} V_i^{k}\right) \cup \bigcap_{k = 1}^N W_k
	$$
	Но $\bigcap_{k = 1}^N W_k \subset V \in P_L$, таким образом мы только что выделили конечное подпокрытие из покрытия $P_L$:
	$$
	K \subset \left(\bigcup_{k = 1}^N \bigcup_{i =1}^{M_k} V_i^{k}\right) \cup V
	$$
	Что противоречит тому, что $P_L \in F$ из этого следует, что предположение о непустоте $F$ неверно и теорема доказана.
\end{proof}

