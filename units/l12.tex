\newpage
\section{Компактные множества в метрических пространствах и линейных нормированных. Теоремы рисса}
\begin{definition}
	Метрическое пространство $(X, \rho)$ называется полным, если любая $\rho$-фундаментальная последовательность сходится.
\end{definition}

\begin{claim}
	Пусть $(K, \rho)$ --- метрическое пространство и $K$ --- секвенциальный компакт, тогда пространство $(K, \rho)$ --- полно
\end{claim}
\begin{proof}
	Рассмотрим фундаментальную последовательность $\{x\}_{n = 1}^\infty \subset K$:
	$$
	\forall \eps> 0 \exists N \in \N: \ \forall n,m > N: \ \rho(x_n,x_m) < \eps
	$$
	Так как $K$---секвенциальный компакт, то последовательность $\{x_n\}$ имеет сходящуюся подпоследовательность $\{x_{n_k}\}_{k=1}^\infty$, по определению сходимости существует $x \in K$:
	$$
	\forall \eps > 0 \ \exists M \in \N \ \forall s > M \ \rho(x, x_s) < \eps
	$$
	Тогда покажем, что вся последовательность $\{x_n\}$ сходится к $x$, действительно:
	$$
	\forall \eps > 0 \exists \max{N,M}: \ \forall s \geq n > \max{N,M}: \rho(x, x_n) \leq \rho(x, x_{n_s}) + \rho(x_{n_s}, x_n) < 2\eps
	$$
	Таким образом последовательность сходится, значит пространство полно
\end{proof}

\begin{definition}
	$(X, \rho)$ --- метрическое пространство, тогда $S \subset X$ называется вполне ограниченным, если существует конечная $\eps$-сеть, то есть:
	$$
	\forall \eps > 0 \ \exists N \in \N, x_1, \dots x_N \in S: \ \bigcup_{n=1}^N O_\eps(x_n) \supset S
	$$
\end{definition}
\begin{remark}
	Ясно, что данное определение не изменится если заменить открытые шары на замкнутые, а также брать элементы сети не из множества $S$, а из всего пространства $X$. 
\end{remark}

\begin{claim}[Критерий вполне ограниченности множества в метрическом пространстве]
	\label{claim:tbc}
	Пусть $(X, \rho)$ --- метрическое пространство, $S \subset X$, тогда $S$ --- вполне ограниченно тогда и только тогда когда любая последовательность $\{x_n\} \subset S$ содержит фундаментальную подпоследовательность.
\end{claim}
\begin{proof}
	\hfil
	\begin{enumerate}
		\item[$\Leftarrow$] Пусть любая последовательность $\{x_n\} \subset S$ имеет фундаментальную подпоследовательность, предположим, что множество не вполне ограниченно, то есть: 
		$$
		\exists \eps_0 > 0 \forall z_1, \dots z_N \in S: \ S \nsubseteq \bigcup_{n=1}^NO_{\eps_0}(z_n)
		$$
		Возьмем сеть состоящую из одной точки $x_1 \in S$ из условия выше получаем, что $S \nsubseteq O_{\eps_0}(x_1)$, тогда $\exists x_2 \in S \setminus O_{\eps_0}(x_1)$. Получили, что 
		$$
		\rho(x_1, x_2) > \eps_0
		$$
		Теперь возьмем сеть из двух точек $x_1, x_2$ аналогично получим точку $x_3$ которая удалена от каждой двух на $\eps_0$, таким образом получена последовательность $\{x_n\} \subset S$ обладающая следующим свойством:
		$$
		\forall n,m \in N \ \rho(x_n, x_m) > \eps_0
		$$
		Но тогда такая последовательность не может иметь фундаментальной подпоследовательности, противоречие с условием, значит $S$ --- вполне ограниченно.
		\item[$\Rightarrow$] Пусть $S$ --- вполне ограниченно, рассмотрим произвольную последовательность $\{x_n\} \subset S$. Пусть $\eps_1 = 1$, тогда, воспользуемся ослабленным определением:
		$$
		\exists z_1 \in X: \ B_1(z_1) \text{ содержит бесконечно много элементов $x_n$}
		$$
		Шар содержит бесконечное число элементов последовательности, так как всего шаров конечно, а последовательность бесконечна. Рассмотрим подпоследовательность, лежащую в этом шаре. 
		$$
		\{x_{n_{m(1)}}\}_{m=1}^\infty \subset B_1(z_1)
		$$ 
		Теперь берем $\eps_2  = \left(\frac{1}{2}\right)^2$. Опять найдется $\eps_2$-сеть, которая покроет все множество, значит
		$$
		\exists z_2 \in X: \ B_{\eps_2}(z_2)\cap B_1(z_1) \neq \varnothing 
		$$
		 И $B_{\eps_2}(z_2)$ содержит бесконечное число элементов подпоследовательности $\{x_{n_m(1)}\}$
		Продолжая таким образом получим счетное число последовательностей $\{\{x_{n_m(s)}\}_{m=1}^\infty\}_{s=1}^\infty$. Как приказал Кантор в таких ситуациях нужно садиться на диагональный элемент. Получим последовательность 
		$$
		x_{n_k} : = x_{n_k(k)}, \ \{x_{n_k}\}_{k=1}^\infty 
		$$
		По построению она является подпоследовательностью исходной последовательности и обладает приятным свойством фундаментальности. Так как все элементы начиная с $k$ лежат в шаре $B_{\eps_k}(z_k), \ \eps_k = \left(\frac{1}{2}\right)^k$. Получили фундаментальную подпоследовательность, что и требовалось.
	\end{enumerate}
\end{proof}
\begin{next0}
	\label{next:nontb}
	$S$ --- не вполне ограниченно тогда и только тогда когда существует дырявая последовательность: 
	$$
	\exists \eps_0 \ \exists \{x_n\} \subset S, \ \forall n \neq m: \  \rho(x_n,x_m) > \eps_0
	$$
\end{next0}

\begin{theorem}[Фреше]
	Пусть $(X, \rho)$ --- метрическое пространство. $K \subset X$, $K$ --- полное и вполне ограниченное, тогда $K$ --- топологический компакт.
\end{theorem}
\begin{proof}
	Предположим, что $K$ --- не является топологическим компактом, то $\exists P \subset \tau_\rho$ не имеет конечного подпокрытия. Рассмотрим $\eps_n = \left(\frac{1}{2}\right)^n$, тогда 
	$$
	\forall n \in \N \exists \ \eps_n\text{- сеть для $K$}: \ x_1(n), \dots x_{N_n}(n) \in K, K \subset \bigcup_{k=1}^{N_n} O_{\eps_n}(x_k(n))
	$$
	При этом для $K$ не выделяется конечного подпокрытия, значит 
	$$
	\forall n \in \N, \exists k_n: O_{\eps_n}(x_{k_n}(n)) \text{не покрывается конечным набором из P}
	$$
	Таким образом К НАМ В РУКИ ПРИПЛЫЛА последовательность $z_n = x_{k_n}(n) \in K$ такая что $O_{\eps_n}(z_n)$ не покрывается конечным набором множеств из $P$. Но $K$ --- вполне ограниченно, тогда по утверждению (\ref{claim:tbc}) из $z_n$ можно выбрать фундаментальную подпоследовательность. $\{z_{n_m}\} \subset K$ --- $\rho$-фундаментальная, но $(K, \rho)$ полно по условию, значит 
	$$
	\exists z \in K, \ z_{n_m} \stackrel{\rho}{\rrr} z \ (m \rr \infty)
	$$
	Так как $P$ --- покрытие $K$, то $\exists V_z \in P$, в силу открытости 
	$$
	\exists r > 0: \ O_r(z) \subset V_z
	$$
	Тогда в силу сходимости ясно, что 
	$$\exists M: \ \forall m \geq M: O_{\eps_m}(z_{n_m}) \subset O_r(z)$$
	Но $z_n$ строилась так что $O_{\eps_m}(z_{n_m})$ нельзя покрыть конечным набором из $P$, а мы покрыли одним $V_z$, противоречие. Таким образом из $P$ можно выбрать конечное подпокрытие, значит $K$ --- топологический компакт.
\end{proof}
\begin{next0}[Критерий компактности в метрических пространств]
	$K \subset X$, $(X, \rho)$ --- метрическое пространство, тогда следующие утверждения эквивалентны
	\begin{itemize}
		\item $K$ --- топологический компакт
		\item $K$ --- счетный компакт
		\item $K$ --- секвенциальный компакт
		\item $K$ --- ПиВО
	\end{itemize}
\end{next0}
