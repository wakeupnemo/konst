\newpage
\section{Слабая* топология в сопряженном пространстве к топологическому векторному пространству. Теорема о представлении слабо* непрерывного линейного функционала.}

По определению $X^*$ --- пространство непрерывных линейных функционалов в $\Cx$. Поэтому можно погрузить его в пространство всех функций $\Cx^X = \{g \colon X \to \Cx\}$. В пространстве функций можно ввести топологию Тихонова и тогда 
\begin{definition}
	Слабой* топологией $\tau_{w^*}$ в пространстве $X^*$ сопряженном топологическому векторному пространству $(X,\tau)$ называется индуцированная с пространства $\Cx^X$ топология Тихонова. 
\end{definition}
Тогда ясно, что предбаза $\tau_{w^*}$:
$$
\sigma_{w^*} = \{V_*(f,x,\eps) \mid f \in X^*, x\in X, \eps > 0\}, \quad V_*(f,x,\eps) = \pi_x^{-1}(B_\eps(f(x)) \cap X^*)
$$
Множество $V_*(f,x,\eps)$ можно так же представить как $V_*(f,x,\eps) = \{g \in X^* \mid |g(x) -f(x)| < \eps\}$. Топология $\tau_{w^*}$ векторная, доказательство этого является очень простым и при этом техническим фактом, которое мне лень полностью техать. Кроме того, ясно что $\tau_{w^*}$ имеет выпуклую локальную базу
$$
\beta_{0w^*} = \left\{\bigcap_{k=1}^N V_*(0,x_k, \eps) \mid x_1,\dots, x_N \in X, \eps > 0\right\}
$$ Значит пространство является локально выпуклым. Следующая теорема проясняет структуру сопряженного к этом пространству.

\begin{theorem}[Шмульян]\label{th:shulian}
	Пусть $(X,\tau)$ --- топологическое векторное и $\Phi \in (X^*, \tau_{w^*})^*$, тогда $\exists x \in X \colon\forall f \in X^* \Rightarrow   \Phi(f) = f(x)$. 
\end{theorem}
\begin{proof}
	В силу линейности $\Phi(0) = 0$. Тогда в силу векторности топологии 
	$$
	\exists U_*(0) \in \tau_{w^*}\colon \forall f \in U_*(0) \Rightarrow |\Phi(f)| < 1
	$$
	Эта окрестность содержит элемент базы
	$$
	U_*(0) \supset \bigcap_{k = 1}^N V_*(0, x_k, \eps)
	$$
	Тогда если мы рассмотрим произвольный $f \in X^*$ такой что $f(x_1) = \dots = f(x_N) = 0$. То
	$$
	\forall n \in \N \Rightarrow nf \in \bigcap_{k = 1}^N\{V_*(0, x_k, \eps)\} \subset U_*(0)
	$$
	Но тогда 
	$$
	|\Phi(nf)| < 1 \Rightarrow |\Phi(f)| < \frac{1}{n} \forall n \in \N
	$$
	Следовательно $\Phi(f) = 0$. Значит мы показали, что 
	$$
	\bigcap_{k=1}^N \Ker F_{x_k} \subset \Ker \Phi
	$$
	Где $F_{x_k}$ --- функционал порожденный каноническим вложением $x_k$ в $(X^*,\tau_{w^*})^*$ ($F_{x_k}(f) = f(x_k)$). Теперь рассмотрим
	$$
	M = \left\{\begin{pmatrix}
		f(x_1) \\
		\vdots \\
		f(x_N)
	\end{pmatrix} \mid f \in X^*\right\} \subset \Cx^N
	$$
	$M$ --- конечномерное подпространство в $\Cx^N$. Заведем на нем функционал $\Lambda \colon M \to \Cx$:
	$$
	\forall f \in X^* \quad	\Lambda(f(x_1), \dots, f(x_N)) = \Phi(f)
	$$
	Проверим что $\Lambda$ определен корректно, действительно, пусть $f(x_k) = g(x_k)$, тогда $f - g \in \bigcap_{k=1}^N \Ker F_{x_k} \subset \Ker \Phi$, тогда 
	$$
	\Phi(f -g)  = 0 \Leftrightarrow \Phi(f) = \Phi(g)
	$$
	Таким образом $\Lambda$ --- функционал над конечномерным пространством, тогда из линейной алгебры известно, что 
	$$
	\exists a_1, \dots, a_N \in \Cx\colon \Phi(f) = \Lambda(f(x_1), \dots, f(x_N)) = \sum_{k=1}^N a_k f(x_k) = f\left(\sum_{k=1}^N a_k x_k\right)
	$$
	Таким образом искомый $x = \sum_{k=1}^N a_k x_k$. Что и требовалось.
\end{proof}
\begin{remark}
	Если исходное пространство $(X,\tau)$ является локально выпуклым, то по следствию теоремы Хана-Банаха, все точки отделяются функционалами, и $x$ из теоремы выше будет единственным.
\end{remark}