\newpage
\section{Теорема о спектре самосопряженного оператора $A \in \CL(\CH)$ в гильбертовом пространстве $\CH$.}
\begin{definition}
	Линейный оператор $A \in \CL(\CH)$ называется самосопряженным если 
	$$
	\forall x,y \in \CH \Rightarrow (Ax, y) = (x, Ay)
	$$
\end{definition}
\begin{claim}
	Пусть $A \in \CL(\CH)$ --- самосопряженный, тогда 
	\begin{enumerate}
		\item $(A(x), x)  \in \R$ для любого $x \in \CH$ 
		\item $\sigma_p(A) \subset \R$
	\end{enumerate}
\end{claim}
\begin{proof}
	\begin{enumerate}
		\item В силу самосопряженности имеем 
		$$
		(Ax,x) = (x,Ax) = \overline{(Ax,x)}
		$$
		Значит $(Ax,x) \in \R$
		\item Пусть $\lambda \in \sigma_p(A)$, тогда $\exists x \neq 0:$ $Ax = \lambda x$. Тогда в силу первого утверждения 
		$$
		(Ax,x) = \lambda(x,x) = \lambda \|x\|^2 \in \R \Rightarrow \lambda \in \R
		$$
	\end{enumerate}
\end{proof}
\begin{claim}
	Пусть $A \in \CL(\CH)$ --- самосопряженный, тогда для любого $\lambda \in \Cx$:
	$$
	\Ker A_\lambda \oplus [\Ima A_\lambda] = \CH
	$$
\end{claim}
\begin{proof}
	Имеем 
	$$
	(A_\lambda)^* = (A - \lambda I)^* =  A^* - \overline{\lambda}I = A_{\overline{\lambda}}
	$$
	В силу теоремы Фредгольма $[\Ima A_\lambda] = \Ker A_{\overline{\lambda}}^\perp$. Тогда по теореме Рисса о дополнении
	$$
	\Ker A_{\overline{\lambda}} \oplus [\Ima A_\lambda] = \CH
	$$
	Если $\lambda \in \R$ то все доказано, в противном случае $\overline{\lambda }\notin \sigma_p(A)$, тогда $\Ker A_{\overline{\lambda}} = \{0\}$ И тогда $[\Ima A_\lambda] = \Ker A_{\overline{\lambda}}^\perp = \{0\}^\perp = \CH$ и утверждение верно. 
\end{proof}
\begin{claim}
	 Пусть $A \in \CL(\CH)$ --- самосопряженный и $\lambda \in \Cx$ такое что $\Ima \lambda \neq 0$, тогда $\lambda \in \rho(A)$ и 
	 $$
	 \|R_A(\lambda)\| \leq \frac{1}{|\Ima \lambda|}
	 $$
\end{claim}
\begin{proof}
	Пусть $\lambda = \mu + i \nu$, тогда 
	$$
	\|A_\lambda(x)\| = (A_\mu x - i\nu x, A_\mu x - i\nu x) = \|A_\mu x\|^2 - i\nu(A_\mu x, x) + i\nu (x, A_\mu) + \nu^2 \|x\|^2
	$$
	Так как $\mu \in \R$ и $A$ --- самосопряженный, то $(A_\mu x, x) = (x, A_\mu x)$, тогда 
	$$
	\|A_\lambda x\|^2 = \|A_\mu x\|^2 + \nu^2 \|x\|^2\leq \nu^2 \|x\|^2 \Rightarrow \|A_\lambda x\| \geq \nu \|x\|
	$$
	Получаем, что оператор $A_\lambda$ --- ограничен снизу, тогда он непрерывно обратим на образе, но так как $\lambda \notin \R$, то $\Ker A_\lambda = \{0\}$. И образ замкнут, тогда $A_\lambda \in \CL(\CH)$, значит $\lambda \in \rho(A)$, кроме того 
	$$
	\|R_A(x)\| \leq \frac{R_A(A_\lambda(x))}{|\nu|} = \frac{\|x\|}{\nu}
	$$ 
	То есть 
	$$
	\|R_A\| \leq \frac{1}{|\nu|}
	$$
	Что и требовалось.
\end{proof}