\newpage
\section{Эквивалентность замкнутости $\Ima A$ и $\Ima A^*$ для оператора $A \in \CL(X,Y)$, где $X$ и $Y$ банаховы пространства. Равенство $(\Ker A)^\perp  = \Ima A^*$ при условии замкнутости $\Ima A$.}

\begin{theorem}[в духе Фредгольма]\label{th:frspirit}
	Пусть $X,Y$ --- банаховы пространства, $A \in \CL(X,Y)$ и $\Ima A$ ---- замкнут. Тогда 
	$$
	(\Ker A)^\perp = \Ima A^*
	$$
\end{theorem}
\begin{proof}
	Пусть $f\in (\Ker A)^\perp$ то есть $\forall x \in X, Ax = 0 \Rightarrow f(x) = 0$
	Хотим доказать, что $f \in \Ima A^*$, тогда будет верно
	$$
	[\Ima A^*]_{\|\|} \subset (\Ker A)^\perp \subset \Ima A^* \subset [\Ima A^*]_{\|\|}
	$$
	Откуда сразу $(\Ker A)^\perp = \Ima A^*$.
	
	Рассмотрим $h \colon \Ima A \to \Cx$ по формуле
	$$
	\forall x \in X  \colon h(Ax) = f(x) 
	$$
	Если $y = Ax_1 = Ax_2 \in \Ima A$, тогда 
	$$
	x_1 - x_2 \in \Ker A \Rightarrow f(x_1 - x_2)  = 0 \Rightarrow h(y) = f(x_1) = f(x_2)
	$$
	Поэтому $h$ определен корректно. Ясно, что $h$ --- линейный функционал. Так как $\Ima A$ --- замкнуто в банаховом пространстве $Y$, то $\Ima A$ --- само банахово, с другой стороны $X$ --- банахово, тогда по теореме банаха об открытом отображении (\ref{th:banachopenmap}) $A : X \to \Ima A$ --- открытое отображение. Значит 
	$$
	\exists r > 0 \colon O_r^Y(0) \cap \Ima A \subset A(O_1^X(0))
	$$
	Тогда $\forall y \in \Ima A \setminus \{0\} \Rightarrow  \frac{r}{2}\frac{y}{\|y\|} \in AO_1^X(0)$. Значит
	$$
	\exists x \in X \colon \|x\| \leq 1 \Rightarrow Ax = \frac{r}{2}\frac{y}{\|y\|}
	$$
	В силу линейности $A$ получаем
	$$
	y = A\left(\frac{2\|y\|}{r}x\right)
	$$
	Теперь вспоминая про $h$ 
	$$
	h(y) = h\left(A\left(\frac{2\|y\|}{r}x\right)\right) = f\left(\frac{2\|y\|}{r}x\right)
	$$
	Равенство выше верно для любого $y$, тогда
	$$
	|h(y)| \leq \|f\|\frac{2}{r}\|y\| \Rightarrow \|h\| \leq \frac{2}{r}\|f\|
	$$
	Значит $h$ --- непрерывный, то есть $h \in (\Ima A)^*$, по теореме Хана-Банаха продолжим его на весь Y:
	$$
	\exists g \in Y^* \colon g\big|_{\Ima A} = h \quad \|g\| = \|h\| 
	$$
	Тогда $\forall x \in X$:
	$$
	f(x) = h(Ax) = g(Ax) = (A^*g)(x) \Rightarrow f = A^*g \in \Ima A^*
	$$
	Что и требовалось!
\end{proof}

\begin{claim} \label{cl:frspirit}
	Если $X,Y$ --- банаховы и $A \in \CL(X,Y)$ и $\Ima A^*$ --- замкнут в $X^*$, тогда $\Ima A$ --- замкнут в $Y$
\end{claim}
Заметим, что в условиях этого утверждения в силу \ref{th:frspirit} получим, что $\Ima A^* = (\Ker A)^\perp$ и $\Ima A = {}^\perp(\Ker A^*)$
\begin{proof}
	Рассмотрим
	$$
	Z = [\Ima A]_{\|\|} \subset Y
	$$
	$Z$ --- замкнутое подпространство банахова пространства, значит $Z$ --- само банахово. Определим
	$$
	T \colon X \to Z, \quad T(x)  = A(x) \forall x\in X 
	$$
	Ясно, что $T \in \CL(X,Z)$, кроме того 
	$$
	\Ima T = \Ima A \Rightarrow [\Ima T]_{Z}  = Z 
	$$
	То есть $\Ima T$--- всюду плотен в $Z$. Рассмотрим сопряженный
	$$
	T^* \colon Z^* \to X^*
	$$
	Так как $\Ima T$ --- всюду плотен, то в силу теоремы Фредгольма (\ref{th:fr}) и следствия леммы \ref{lem:densyty} $\Ker T^* = (\Ima T)^\perp = \{0\}$. Значит $T^*$ --- инъективен, значит существует обратный оператор
	$$
	\exists (T^*)^{-1}\colon \Ima T^* \to Z^*
	$$
	Текущая картина
	$$
	T^* \colon Z^* \to X^* \quad A^* \colon Y^* \to X^* 
	$$
	Причем $\Ima A^*$ --- замкнут. Нужно понять, что из себя представляет $\Ima T^*$. 
	
	Рассмотрим произвольный $h \in Z^*$ по теореме Хана-Банаха 
	$$
	\exists g \in Y^*\colon g\big|_{Z}= h
	$$
	Тогда 
	$$
	\forall x \in X\colon h(T(x)) = (T^*h)(x)
	$$
	С другой стороны 
	$$
	h(T(x))  = h(\underbrace{A(x)}_{\in \Ima A})  = g(Ax)  = (A^*g)(x)
	$$
	Таким образом
	$$
	\forall x \in X\  T^*h(x) = A^*g(x) \Rightarrow T^*h = A^* g
	$$
	Таким образом $ \Ima T^* \subset \Ima A^*$
	Но можно рассуждать и в обратную сторону. Возьмем $g \in Y^*$, тогда $\forall x \in X$
	$$
	g(Ax) = (A^*g)(x)
	$$
	Рассматривая сужение $h = g\big|_Z$
	$$
	g(Ax) = g(Tx) = h(Tx) = (T^*h)(x)
	$$
	Значит $\forall x \in X \colon T^*g = A^*h \Rightarrow \Ima A^* \subset \Ima T^*$. Получили, что
	$$
	\Ima A^* = \Ima T^*
	$$
	Так как оператор $T$ --- сужение оператора $A$, то этот результат несколько тавтологичен. Однако теперь мы можем утверждать, что $\Ima T^*$---замкнут в банаховом $X^*$, значит по теореме Банаха об обратном операторе (\ref{th:inv_op}) 
	$$
	(T^*)^{-1} \in \CL(\Ima T^*, Z^*) \quad 0< \|(T^*)^{-1}\| < \infty
	$$
	А теперь ФОКУС. Что можно сказать про прямой оператор $T$, если он обладает непрерывным обратным сопряженным?
	Оказывается, что если $(T^*)^{-1} \in \CL(\Ima T^*, Z^*)$, то $T \colon X \to Z$ --- открытое отображение, то есть 
	$$
	\exists r>0 \colon T(O_1^X(0)) \supset O_r^Z(0)
	$$
	Предположив это, моментально получаем, что $\Ima T = Z$ и 
	$$
	[\Ima A]_{\|\|} = Z \supset \Ima A = \Ima T = Z 
	$$
	И получаем $\Ima A = [\Ima A]_{\|\|}$
	
	Докажем, что $T\colon X \to Z$ --- открытое отображение. Рассмотрим $[TO_1(0)]_Z$ --- замкнутое и выпуклое в $Z$ множество. Рассмотрим $z \in Z \setminus [TO_1(0)]_Z$. По следствию теоремы Хана-Банаха отделим $z$ от выпуклого замкнутого множества $[TO_1(0)]_Z$, получим 
	$$
	\exists f \in Z^*, \|f\| > 0 , \ \exists\gamma \in \R \colon\forall \|x\|\leq 1  \Rea f(T(x)) \leq \gamma < \Rea f(z)
	$$
	Взяв супремум по всем $x$ из единичного шара, получим
	$$
	\sup\limits_{\|x\| \leq 1} |\Rea f(T(x))| = 	\sup\limits_{\|x\| \leq 1} |\Rea (T^* f)(x)| = \|\Rea (T^* f)\| = \|T^* f\|\leq\gamma < \Rea f(z)
	$$
	Таким образом $T^* f \in X^*$. Кроме того 
	$$
	f = (T^*)^{-1}(T^*f) \Rightarrow \|f\| \leq \|(T^*)^{-1}\| \|T^*f\| 
	$$
	Значит
	$$
	\frac{\|f\|}{\|(T^*)^{-1}\|} \leq \|T^*f\|
	$$
	Теперь можем записать цепочку неравенств
	$$
	0 < \frac{\|f\|}{\|(T^*)^{-1}\|} \leq \gamma < \Rea f(z) \leq \|\Rea f\| \|z\| = \|f\| \|z\| \Rightarrow \|z\| > \frac{1}{\|(T^*)^{-1}\|} = k > 0
	$$
	Значит $z$ не может быть очень маленьким, точнее
	$$
	z \in Z \setminus B_{k}^Z(0) 
	$$
	Но $z \in Z \setminus [TO_1(0)]_Z$, значит мы получили
	$$
	O_k^Z(0) \subset B_{k}^Z(0)  \subset [TO_1(0)]_Z 
	$$
	Мы попали в ситуацию, аналогичную ситуации в доказательстве теоремы \ref{th:banachopenmap}. Так как $X$ --- полное, то, повторяя рассуждения из того доказательства 
	$$
	[TO_1(0)]_Z  \subset T(B_2^X(0)) \subset T(O_3^X(0))
	$$
	Таким образом 
	$$
	O_{\frac{k}{3}}^Z(0) \subset TO_1^X(0) 
	$$
	То есть $T$--- открытое отображение! Доказательство окончено.
\end{proof}
