\newpage
\section{Компактные операторы в пространстве $\CL(X,Y)$. Замкнутость подпространства компактных операторов $\mathcal{K}(X,Y)$ в пространстве $\CL(X,Y)$ с равномерной операторной топологией.}
\begin{definition}
	Линейный оператор $A \colon X \to Y$, где $X,Y$ --- линейные нормированные пространства, называется компактным, если образ любого ограниченного множества является предкомпактом (то есть его замыкание является компактом). Пространство компактных операторов обозначается $\CK(X,Y)$.
\end{definition}
\begin{definition}
	В случае банахового $Y$ предкомпактность равносильна вполне ограниченности. 
\end{definition}
Так как вполне ограниченное множество всегда ограниченно, то $\CK(X,Y) \subset \CL(X,Y)$.
\begin{claim}
	Пусть последовательность компактных операторов 
	$$
	\{A_m\}_{m=1}^\infty \subset \CK(X,Y)
	$$
	является сходящейся к оператору $A$ по операторной норме, т.е. 
	$$
	\|A_m - A\| \xrightarrow{m \to \infty} 0 
	$$
	Тогда $A$ является компактным оператором. Иными словами подпространство $\CK(X,Y)$ замкнуто в $\CL(X,Y)$.
\end{claim}
\begin{proof}
	В силу сходимости по операторной норме
	$$
	\forall \eps > 0 \ \exists N(\eps) \colon \forall m \geq N(\eps) \Rightarrow \|A_m - A\| \leq \eps
	$$
	Тогда для любого $x \in B_1(0)$ получаем 
	$$
	\|A_m(x) - A(x)\| \leq \|A_m - A\| \leq \eps 
	$$
	Зафиксируем произвольное $m \geq N(\eps)$. Так как множество $A_m(B_1(0))$ вполне ограниченно в $Y$, то существуют векторы 
	$$
	x_1, \dots, x_N \in B_1(0)
	$$
	Такие что множество 
	$$
	A_m(x_1), \dots, A_m(x_N) \in A_m(B_1(0))
	$$
	является конечной $\eps$-сетью для множества $A_m(B_1(0))$. Тогда $\forall x \in B_1(0)$ $\exists x_k$:
	$$
	\|A_m(x) - A_m(x_k)\| \leq \eps
	$$ 
	Тогда получаем 
	$$
	\|A(x) - A(x_k)\| \leq \|A(x) - A_m(x)\| + \|A_m(x) - A_m(x_k)\| + \|A_m(x_k) - A(x_k)\| \leq 3\eps 
	$$
	что и требовалось.
\end{proof}