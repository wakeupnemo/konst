\newpage
\section{Счeтно компактные и секвенциально компактные подмножества топологического пространства, связь между ними.}

\begin{definition}
	$K$ называется счетным компактом в $(X, \tau)$, если из любого счетного открытого покрытия $K$ можно выделить конечное подпокрытие.
\end{definition}

\begin{definition}
	$K$ называется секвенциальным компактом в $(X, \tau)$, если любая последовательность из $K$ имеет сходящуюся подпоследовательность. Формально: $\forall \{x_n\}_{n=1}^{\infty} \subset K$ $\exists n_1 < n_2 < \dots < n_k < \dots$ и $\exists x_0 \in K$: $x_{n_k} \stackrel{\tau}{\rrr}x_0 \ (k \rr \infty)$
\end{definition}

\begin{claim}
	Из секвенциальной компактности следует счетная компактность. 
\end{claim}
\begin{proof}
	Пусть $K$ --- секвенциальный компакт, предположим, что $K$ --- не является счетным компактом, тогда найдется покрытие $P = \{V_n\}_{n=1}^\infty$ --- $\tau$-покрытие $K$, не имеющее конечного подпокрытия, то есть:
	$$
	\forall n \in \N: \ K \setminus \bigcup_{m=1}^n V_m \neq \varnothing 
	$$
	Тогда К НАМ В РУКИ ПЛЫВЕТ последовательность $x_n \in K \setminus \bigcup_{m=1}^nV_m$. В силу секвенциальной компактности $K$ данная последовательность имеет сходящуюся подпоследовательность $x_{n_k} \rr x_0 \in K \ (k \rr \infty)$. Но $x_0$ лежит в $K$, тогда в силу того, что $V_n$ покрывают все $K$ найдется номер $n_0$, такой что $x_0 \in V_{n_0}$. Тогда в силу сходимости подпоследовательности к $x_0$:
	$$
	\exists k_0: \ \forall k \geq k_0: \ x_{n_k} \in V_{n_0}
	$$
	Но тогда взяв достаточно большой номер $k^*$, а конкретно 
	$$
	k^* \geq \max\{n_0, k_0\} \Rightarrow \ n_{k^*} \geq k^* \geq n_0
	$$Получим:
	$$
	V_{n_0} \ni x_{n_{k^*}} \in K \setminus \bigcup_{m=1}^{n_{k^*}}V_m \subset K \setminus V_{n_0}
	$$
	Получаем противоречие. 
\end{proof}
\begin{claim}
	Пусть $K$ --- счетный компакт и выполнена \hyperlink{fcs}{аксиома счетности}, тогда $K$ --- секвенциальный компакт.
\end{claim}
\begin{proof}
	 Возьмем произвольную последовательность $\{x_n\} \subset K$, тогда покажем, что найдется такая точка $z\in K$, которая имеет в любой своей окрестности бесконечное количество элементов $\{x_n\}$.
	 $$
	 \exists z \in K: \ \forall U(z) \Rightarrow \{n \in \N \mid x_n \in U(z)\} \text{ --- бесконечно}
	 $$
	 Предположим противное, тогда 
	 $$
	 \forall z \in K: \exists U(z) \in \tau: \ \{n \in \N \mid x_n \in U(z)\} \text{ --- конечно или пусто}
	 $$
	 Рассмотрим конечные или пустые подмножества натурального ряда $I \subset \N$ и для каждого такого $I$ определим:
	 $$
	 M_I = \{z \in K \mid \{n \in \N \mid x_n \in U(z)\} =I\}
	 $$
	 Где $U(z)$ --- существующая по предположению окрестность $z$, которая содержит конечное или пустое множество элементов последовательности.
	 Заметим, что количество таких $I$ --- счетно, тогда $M_I$ --- тоже счетно. Теперь определим 
	 $$
	 V_I = \bigcup_{z \in M_I}U(z) \in \tau
	 $$
	 Получили счетное покрытие: $P = \{V_I \mid I \subset \N \text{ --- конечно или пусто}\}$ Это покрытие, так как $\forall z \in K: I (z) := \{n \mid x_n \in U(z)\}$ и $z \in U(z) \subset V_{I(z)}$. Но $K$ --- счетный компакт, тогда существуют $I_1, \dots, I_N$ такие что 
	 $$
	 K \subset \bigcup_{m=1}^NV_{I_m}
	 $$
	 Но $\cup I_n$ --- конечное множество индексов, тогда 
	 $$
	 \forall n \in \N \setminus \bigcup_{n=1}^N I_n \Rightarrow K \ni x_n \notin \bigcup_{m=1}^N V_{I_m} \supset K
	 $$
	 Получили противоречие, таким образом:
	 $$
	 \exists z \in K: \ \forall U(z) \Rightarrow \{n \in \N \mid x_n \in U(z)\} \text{ --- бесконечно}
	 $$
	 В силу аксиомы счетности для этой точки имеется счетная локальная база:
	 $$ \exists  \{W_n(z)\}_n^{\infty} \subset \tau: \ \forall U(z) \ \exists n: U(z) \supset W_n(z)$$ 
	 Причем, как было оговорено, считаем, что $W_n$ упорядоченны по вложению. В силу свойств точки $z$:
	 \begin{itemize}
	 	\item $W_1$ содержит бесконечно много элементов последовательности $\{x_n\}$, значит $\exists n_1: x_{n_1} \in W_1$
	 	\item $W_2$ содержит бесконечно много элементов $\{x_n\}$, значит $\exists n_2 > n_1$ (так как бесконечно много) $x_{n_2} \in W_2$
	 \end{itemize} 
 	Догадливый читатель уже понял, что, продолжая таким образом, мы получим подпоследовательность $\{x_{n_k}\}_{k=1}^\infty$, которая сходится к $z$ в силу свойств локальной базы:
 	$$
 	\forall U(z) \ \exists N \in \N: \forall k \geq N: x_{n_k} \in W_N(z) \subset U(z)
 	$$
	Таким образом $K$ --- секвенциальный компакт.
\end{proof}