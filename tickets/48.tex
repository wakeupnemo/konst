\newpage

\section{Теорема Гельфанда-Наймарка для коммутативной подалгебры банаховой алгебры $\CL(H)$, где $H$--- гильбертово пространство.}

\begin{claim}
	Множество комплексных гомоморфизмов $Z(\CA)$ банахововой алгебры $\CA$ является слабым* компактом в пространстве $\CA^*$
\end{claim}
\begin{definition}
	Преобразованием Гельфанда элемента $x \in \CA$ банаховой алгебры $\CA$ называется отображение $\hat{x}: Z(\CA) \to \Cx$ вида
	$$
	\hat{x}(h) = h(x) \ \forall h \in Z(\CA)
	$$
\end{definition}
Видим, что $\hat{x}$ является слабо* топологически непрерывным отображением слабого* компакта $Z(\CA) \subset \CA^*$ в $\Cx$.

Пусть $C(Z(\CA)) = \{\varphi \colon Z(\CA) \to \Cx \mid \text{$\varphi$ слабо* непрерывно}\}$. Так как $Z(\CA)$ --- слабый* компакт, то любое отображение $\varphi \in C(Z(\CA))$ ограниченно на $Z(\CA)$ и 
$$
\|\varphi\|_{\infty} = \max_{h \in Z(\CA)}|\varphi(h)|
$$
достигается. Получается, что для любого $x \in \CA$ $\hat{x}$ лежит в $C(Z(\CA))$ и
$$
\|\hat{x}\| = \max_{h \in Z(\CA)}|\hat{x}(h)| = \max_{h \in Z(\CA)}|h(x)|  \leq \max_{\lambda \in \sigma(x)} |\lambda| = r(x)
$$
где неравество выполняется, так как $\{h(x) \mid h \in Z(\CA)\} \subset \sigma(x)$. Если $\CA$ --- коммутативная банахова алгебра, то по теореме Гельфанда имеем
$$
\hat{x}(Z(\CA)) = \{h(x) \mid h \in Z(\CA)\} = \sigma(x)
$$
откуда $\|\hat{x}\|_{\infty} = r(x)$. 

Заметим, что пространство $(C(Z(\CA)), \|\|_{\infty})$ само является коммутативной банаховой алгеброй с произведением 
$$
(\varphi \psi) (h)= \varphi(h) \psi(h)
$$
(проверьте все аксиомы и докажите это). Кроме того, множество 
$$
\hat{A} = \{\hat{x} \mid x \in \CA \} \subset C(Z(\CA))
$$
является подалгеброй алгебры $C(Z(\CA))$ (этот факт тоже проверьте в качестве упражнения, чтобы проверить --  не выпали ли вы еще из понимания, потому что на данном этапе познания курса функционального анализа, выпасть из понимания --- естественная защитная реакция организма). 

Главная теорема из билета дает понимание, когда подалгебра $\hat{A}$ совпадает со всей алгеброй $C(Z(\CA))$. Чтобы ее сформулировать понадобится ввести новые ограничения на алгебру. 

\begin{definition}
	Говорят, что в банаховой алгебре $\CA$ задана инволюция элемента, если для любого $x \in \CA$ определена операция $x^* \in \CA$, обладающая следующими свойствами (для всех $x,y \in \CA, \alpha \in \Cx$):
	\begin{enumerate}
		\item $x^{**} = x$;
		\item $(x + y)^* = x^* + y^*$;
		\item $(xy)^* = y^*x^*$.
	\end{enumerate}
\end{definition}
Например для гильбертого пространства $\CH$ банахова алгебра $\CA=\CL(\CH)$ является банаховой алгеброй с инволюцией, где инволюцией является эрмитово сопряжение. Cвойства проверяются тривиально. При этом выполнено дополнительно свойство: для любого $T \in \CL(\CH)$ выполено
$$
\|T^{+} T\| = \|T\|^2.
$$
В самом деле, для любого $x \in \CH$ имеем
$$
\|Tx\|^2 = (Tx, Tx) = (x, T^{+}Tx) \leq \|T^{+}T\| \|x\|^2
$$
откуда
$$
\|T\|^2 = \sup_{\|x\|\leq 1}\|Tx\|^2 \leq \|T^{+}T\| 
$$
и также верно 
$$
\|T^{+}T\| \leq \|T^{+}\|\|T\| = \|T\|^2
$$
так как $\|T^{+}\| = \|T\|$. Данное свойство прерващается в следующее 
\begin{definition}
	Банахова алгебра с инволюцией называется $B^*$-алгеброй если для любого $x \in \CA$ выполнено 
	$$
	\|x^*x\| = \|x\|^2
	$$
\end{definition}
Таким образом $\CL(\CH)$ --- $B^*$-алгебра.
\begin{definition}
	Пусть $\CA$ --- банахова алгебра с инволюцией. Элемент $x \in \CA$ называется эрмитовым, если $x^* = x$. 
\end{definition}