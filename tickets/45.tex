\newpage
\section{Теорема о спектре компактного оператора $A \in \CL(X)$ в банаховом пространстве $X$.}
\begin{theorem}
	Пусть $A \in \CK(X)$ --- компактный оператор. Тогда 
	\begin{enumerate}
		\item $\forall \lambda \in \sigma(A) \setminus \{0\} \Rightarrow \lambda \in \sigma_p(A)$
		\item Множество $\sigma(A)$ не более чем счетно и не имеет предельных точек кроме, быть может $0$.
	\end{enumerate}
\end{theorem}
\begin{proof}
	\begin{enumerate}
		\item Пусть $\lambda \in \sigma(A) \setminus \{0\}$. Предположим, что $\lambda \notin \sigma_p(A)$, тогда $\Ker A_\lambda = \{0\}$. Тогда в силу 
		$$
		\dim \Ker A_\lambda = \dim \Ker A^*_\lambda 
		$$
		теоремы Фредгольма и замкнутости образа $\Ima A_\lambda$ получаем
		$$
		\Ima A_\lambda = {}^\perp (\Ker A^*_\lambda) = {}^\perp\{0\} = X
		$$
		Но тогда по теореме Банаха об обратном операторе $A_\lambda$ обратим, противоречие с тем, что $\lambda \in \sigma(A)$. 
		\item Для $\delta > 0$ рассмотрим множества 
		$$
		\Lambda_\delta = \{ \lambda \in \sigma_p(A) \mid |\lambda| \geq \delta\}
		$$
		Покажем, что все они конечны или пусты. Предположим противное, то есть, что существует $\delta > 0$ для которого $\Lambda_\delta$ бесконечно, тогда оно содержит счетную последовательность различных собственных значений 
		$$
		\{\lambda_m\}_{m=1}^\infty \subset \Lambda_\delta
		$$
			Каждому собственному значению соответствует собственный вектор $x_m$. 
			Так как все $x_m$ отвечают различным собственным значениям, то $x_m$ линейно независимы. Определим 
		$$
		M_n = \Lin \{x_m\}_{m=1}^n
		$$
		Тогда имеем 
		$$
		A(M_n) \subset M_n \quad A_{\lambda_{n+1}} \subset M_n
		$$
		и
		$$
		M_n \subset M_{n+1}, \quad M_n \neq M_{n+1}
		$$
		Применим теорему Рисса о почти перпендикуляре для линейного нормированного пространства $M_{n+1}$ и его подпространства $M_n$. Тогда получим
		$$
		\exists z_n \in M_{n+1}: \ \|z_n\| = 1 \text{ и } \rho(z_n, M_n) > \frac{1}{2}
		$$
		Тогда покажем, что последовательность $\{A(z_n)\}_{n=1}^\infty$ не содержит фундаментальной подпоследовательности, что будет противоречить компактности оператора $A$. Действительно, если $m < n$, то 
		$$
		A(z_m) \in A(M_{m+1}) \subset M_{m+1} \subset M_n
		$$
		и
		$$
		A_{\lambda_{n+1}}(z_n) \in A_{\lambda_{n+1}}(M_{n+1}) \subset M_n
		$$
		Тогда 
		$$
		w = A(z_m) - A_{\lambda_{n+1}}(z_n) \in M_n
		$$
		Тогда 
		$$
		\|A(z_n) - A(z_m)\| = \|A(z_n) + A(z_m) - A_{\lambda_{n+1}}(z_n)  - w - A(z_m)\| = \|\lambda_{n+1} z_m - w\| \leq |\lambda_{n+1}|\rho(z_n, M_n) \leq \frac{\delta}{2}
		$$
		Что и требовалось. 
	\end{enumerate}
\end{proof}