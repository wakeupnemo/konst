\newpage
\section{Факторпространство замкнутого подпространства линейного нормированного пространства, факторнорма. Полнота факторпространства замкнутого подпространства банахова пространства}

Нормированное пространство является частным случаем топологического векторного, поэтому определение для топологического векторного (\ref{def:fac}) остается прежним, однако теперь у нас есть норма, поэтому хочется ввести норму и в фактор пространстве.
\begin{definition}
	Пусть $(X, \|\|)$ --- нормированное пространство, $N\subset X$ --- некоторое его замкнутое подпространство. Факторнормой в факторпространстве $X / N$ называется 
	$$
	\|\xi\| = \inf_{y \in \xi} \|y\|
	$$ 
\end{definition}
\begin{claim}
	Введенная норма действительно является нормой в линейном пространстве $X / N$.
\end{claim}
\begin{claim}
	Неотрицательность, положительная однородность и неравенство треугольника очевидно наследуются из пространства $X$. Вопрос возникает со свойством $$\forall \xi \in X / N \Rightarrow \|\xi\| = 0 \Leftrightarrow \xi = 0.$$
	Покажем его. Нулевой элемент фактор пространства это класс равный $N$, но $N$ --- подпространство, значит $0 \in N$, тогда $\|N\| = \inf\limits_{y \in N}\|y\| = 0$, с другой стороны, пусть $\xi \in X / N, \ \|\xi\| = 0$, тогда по определению инфинума существует последовательность элементов $\xi$ сходящихся к нулю, но $\xi = y +N$ для некоторого $y \in X$, сумма в топологическом векторном пространстве не портит замкнутости, значит $\xi$ замкнуто в $X$, тогда $0 \in \xi$, значит $\xi = N$.
\end{claim}

\begin{claim}\label{cl:factorbanach}
	Пусть $(X, \|\|)$ --- банахово, $N \subset X$ --- замкнутое подпространство, тогда $X / N$ с факторнормой банахово.
\end{claim}
\begin{proof}
	В силу определения факторнормы $\forall \xi \in X / N$ найдется элемент $x \in X$: 
	$$
	\|\xi\| \geq \frac{\|x\|}{2}
	$$
	Пусть $\{\xi_n\}\subset X / N$ --- фундаментальная последовательность в $X / N$, тогда можно считать (если что перейдем к подпоследовательности), что ряд 
	$$
	\sum_{n=1}^\infty \|\xi_n - \xi_{n-1}\| 
	$$
	сходится. Добавив к $\{\xi_n\}$ нулевой элемент $\xi_0 = N$. Выберем $x_n \in \xi_{n+1} - \xi_n$ так, что 
	$$
	\|\xi_n - \xi_{n-1}\| \geq \frac{\|x_n\|}{2}
	$$
	Тогда ряд 
	$$
	\sum_{n=0}^\infty \|x_n\|
	$$
	сходится и, в силу банаховости пространства $X$, $\sum_{n=1}^\infty x_n$ сходится к некоторому элементу $x \in X$, рассмотрим $\xi = x + N$, имеем (так как $\sum_{n=0}^k x_n \in \xi_k$ при каждом $k$)
	$$
	\|\xi_n - \xi\| \leq \left\| x - \sum_{n=1}^k x_n\right\| \xrightarrow{k \to \infty}0
	$$
	Значит $\{\xi_n\}$ --- сходится, что и требовалось.
\end{proof}
	