\newpage
\section{Эквивалентные нормы в линейном пространстве. Эквивалентность норм в конечномерном линейном пространстве.}

\begin{definition}
	Пусть $Z$ --- линейное пространство и $\|\|_1, \|\|_2$ --- нормы в этом пространстве, тогда говорят, что эти нормы эквивалентны, если
	$$
	\exists C_1, C_2 \in \R \ \forall x \in Z:	\begin{cases}
		\|x\|_1 \leq C_2 \|x\|_2 \\
		\|x\|_2 \leq C_1 \|x\|_1
	\end{cases}
	$$
\end{definition}

Следующая лемма, в силу ее простоты, не формулировалась на лекции, но я решил вынести ее отдельно, потому что она носит общий характер.
\begin{lemma}
	В локально компактном нормированном пространстве $(X, \|\|)$ замкнутый шар $B_1(0)$ является компактом. 
\end{lemma}
\begin{proof}
	Так как $X$ --- локально компактно, то существует окрестность нуля $	U(0)$ такая, что $[U(0)]_{\|\|}$ является компактом. Так как в нормированном пространстве шары образуют базу нормированной топологии, то существует $r>0$ такое что 
	$$
	O_r(0) \subset U(0)
	$$
	Но тогда его замыкание $B_r(0)$ содержится в компакте $[U(0)]_{\|\|}$, а значит, как замкнутое подмножество компакта, является компактом. Но умножение на константу в топологическом векторном пространстве является непрерывным отображением, тогда 
	$$
	\frac{1}{r} B_r(0) = B_1(0)
	$$ 
	тоже является компактом, что и требовалось.
	
\end{proof}

\begin{claim}
	Пусть $L$ --- конечномерное линейное пространство, тогда любые нормы  на $L$ эквивалентны. 
\end{claim}
\begin{proof}
	Для определенности будем считать, что $L$ --- линейное пространство над полем $\R$ (можно взять $\Cx$, это ничего не изменит). Пусть $\dim L = n$ и $\{e_1, \dots, e_n\}$ --- базис. Тогда 
	$$
	\forall x \in L \Rightarrow \exists \alpha_1, \dots, \alpha_n \in \R: \ x = \sum_{k=1}^n \alpha_k e_k
	$$
	Пусть на $L$ введена некоторая норма, тогда 
	$$
	\|x\| \leq \sum_{k=1}^n|\alpha_k| \|e_k\| \leq M\sum_{k=1}^n |\alpha_k|
	$$
	где $M = \max\limits_{k} \|e_k\|$. Введем новую норму:
	$$
	\|x\|_e = \sum_{k=1}^n |\alpha_k|
	$$
	Легко проверить, что все аксиомы нормы выполнены, тогда мы получили оценку:
	$$
	\|x \| \leq M \|x\|_e
	$$
	Теперь покажем, что существует $C > 0$ такое, что $ \forall x \in L \colon \|x\|_e \leq C \|x\|$. Это будет означать, что нормы $\|\|$ и $\|\|_e$ эквивалентны, а значит и все. 
	
	Так как $L$ --- конечномерное линейное пространство размерности $n$, то оно изоморфно линейному пространству $\R^n$. Вспоминаем, что $(L, \|\|)$ --- нормированное пространство, а значит топологическое векторное, тогда в силу утверждения (\ref{claim:hom}) $L$ гомеоморфно $\R^n$ со стандартной топологией. Но $\R^n$ --- локально компактное пространство, значит в силу предыдущей леммы $B_1(0)$ является компактом. Теперь построим конкретный изоморфизм между $L$ и $\R^n$. Будем смотреть на $\alpha_k$ как на функции от $x$, тогда отображение 
	$$
	\Lambda \colon L \to \R^n \quad \Lambda(x) = (\alpha_1(x), \dots, \alpha_n(x))^T
	$$
	является изоморфизмом. Опять, в силу утверждения (\ref{claim:hom}), этот изоморфизм является гомеоморфизмом, тогда 	$\Lambda(B_1(0))$ является компактом в $\R^n$. Компакт в $\R^n$ является замкнутыми и ограниченными множеством, значит все координаты компактного множества $\Lambda(B_1(0))$ ограничены некоторой константой $R > 0$, тогда 
	$$
	\forall x \in L\colon \|x\| \leq 1 \Rightarrow |\alpha_k(x)| \leq R
	$$
	Тогда 
	$$
	\|x\|_e = \sum_{k=1}^n |\alpha_k| \leq n R
	$$
	Но тогда $\forall x \in L$ имеем
	$$
	\left\|\frac{x}{\|x\|}\right\|_e \leq nR \Rightarrow \|x\|_e \leq nR \|x\|
	$$
	Что завершает доказательство. 
\end{proof}
