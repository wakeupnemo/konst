\newpage
\section{Лемма Рисса о почти перпендикуляре. Теорема Рисса об отсутствии вполне ограниченности сферы в бесконечномерном линейном нормированном пространстве. }
\begin{lemma}[О почти перпендикуляре]
	\label{lemm:perp}
	Если $(X, \|\|)$ --- линейное нормированное пространство, $L \subsetneq X$ --- замкнутое собственное подпространство $X$, тогда
	$$
	\forall \eps \in (0,1) \ \exists x_\eps \in S: \ \rho(x_\eps, L) \geq 1 - \eps
	$$
	Где $S = \{y \in X \mid \|y \| = 1\}$ --- единичная сфера.
\end{lemma}
\begin{proof}
	По определению $\rho(x, L) = \inf_{y \in L} \| x - y\|$. $L$ --- замкнутое собственное, тогда 
	$$
	\exists x_0 \in X \setminus L: \ \rho(x_0, L) = \inf_{y \in L}\|x_0 - y\| > 0
	$$
	Тогда $\forall \eps \in (0,1)$ рассмотрим 
	$$
	\frac{\rho(x_0, L)}{1 - \eps} > \rho(x_0, L)
	$$
	Значит 
	$$\exists y_\eps \in L: \|x_0 - y_\eps\| < \frac{\rho(x_0, L)}{1 - \eps}$$
	Тогда положим $x_\eps = \frac{x_0 - y_\eps}{\|x_0 - y_\eps\|} \in S$. Тогда $\forall y \in L$:
	$$
	\|y - x_\eps\| = \left\| y - \frac{x_0 - y_\eps}{\|x_0 - y_\eps\|}\right\| = \frac{\|\overbrace{\|x_0-y_\eps\|y + y_\eps}^{\in L}  - x_0 \|}{\|x_0 - y_\eps\|} \geq \frac{\rho(x_0, L)}{\|x_0 - y_\eps\|} > 1 - \eps
	$$
	То есть $\rho(x_\eps, L) = \inf_{y \in L}\|y - x_\eps\| \geq 1 - \eps$ что и требовалось.
\end{proof}
\begin{theorem}[Рисс]
	Пусть $(X, \| \|)$ --- бесконечномерное линейное нормированное пространство. Тогда сфера $S = \{x \in X \mid \|x\| = 1\}$ --- не вполне ограниченна
\end{theorem}
\begin{proof}
	$\forall x_1 \in S$ рассмотрим $L_1 = Lin \{x_1\}$ --- одномерно в $X$. Тогда так как  линейное нормированное пространство частный случай топологического векторного пространства в силу \ref{th:clfinds} $L_1$ --- замкнуто. При этом так как $X$ --- бесконечномерно, то $L_1 \neq X$, значит $L_1$ --- собственное замкнутое подпространство, тогда по лемме \ref{lemm:perp} для $\eps = \frac{1}{2}$:
	$$
	\exists x_2 \in S: \ \|x_2 - x_1\| \geq \rho(x_2, L_1) \geq 1 - \eps = \frac{1}{2}
	$$
	Далее рассматриваем $L_2 = Lin\{x_1, x_2\}$ аналогично $L_2$ конечномерно и замкнуто, тогда опять применяя лемму найдем $x_3$: 
	$$
	\|x_3 - x_{1,2}\| \geq \rho(x_3, L_2) \geq \frac{1}{2}
	$$
	Продолжая этот процесс получим дырявую последовательность $\{x_n\} \subset S$ (мы всегда найдем новое конечномерное собственное подпространство в силу бесконечномерности $X$), значит по следствию теоремы \ref{claim:tbc} $S$ --- не вполне ограниченно.
\end{proof}