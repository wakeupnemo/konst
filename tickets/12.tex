\newpage
\section{Метрические пространства и метрическая топология. Теорема Бэра о категории.}
\begin{definition}
	$(X, \rho)$ называется метрическим пространством, если функция $\rho \colon X \times X \to [0, \infty)$ такая, что 
	\begin{itemize}
		\item $\forall x,y \in X\colon \ \rho(x,y) = 0 \Leftrightarrow x = y$
		\item $\forall x,y \in X \colon \ \rho(x,y) = \rho(y,x)$
		\item $\forall x,y,z \in X \colon \ \rho(x,z) \leq \rho(x,y) + \rho(y,z)$
	\end{itemize}
\end{definition}

\begin{definition}
	В метрическом пространстве $(X,\rho)$ открытым шаром в точке $x$ радиуса $R$ называется
	$$
	O_R(x) = \{y \in X \mid \rho(x,y) < R\}
	$$
\end{definition}
\begin{definition}
	Замкнутый шаром будем обозначать $B_R(x) = [O_R(x)]$.
\end{definition}
\begin{definition}
	Пусть $(X, \rho)$ --- метрическое пространство, топология в $X$, порожденная базой 
	$$
	\beta_\rho = \{O_R(x) \mid R > 0, \ x \in X\}
	$$
	Называется метрической топологией (мы будем обозначать ее $\tau_\rho$).
\end{definition}
\begin{claim}
	$\beta_\rho$ действительно является базой некоторой топологии.
\end{claim}
\begin{proof}
	Проверим критерий базы (\ref{th:base_criterian}). Для любого $x \in X$, $x \in O_R(x)$, поэтому первое условие выполнено. Пусть $x \in O_{R_1}(x_1) \cap O_{R_2}(x_2)$. Покажем что существует такое $R > 0$, что 	
	$$
	O_R(x) \subset O_{R_1}(x_1) \cap O_{R_2}(x_2)
	$$
	Возьмем $R = \min\{R_1 - \rho(x_1, x), R_2 - \rho(x_2, x)\}$, тогда 
	$$
	\forall y \in O_R(x) \Rightarrow \rho(x_1, y) < R_1
	$$
	Значит, можем записать 
	$$
	\rho(x_1, y) \leq \rho(x_1, x) +  \rho(x, y) < \rho(x_1,x) + R \leq R_1
	$$
	Аналогично 
	$$
\rho(x_2, y) \leq \rho(x_2, x) +  \rho(x, y) < \rho(x_2,x) + R \leq  R_2
	$$
	Таким образом 
	$$
	O_R(x) \subset O_{R_1}(x_1) \cap O_{R_2}(x_2)
	$$
	И значит, $\beta_\rho$ --- база.
\end{proof}

\begin{definition}
	Последовательность $\{x_n\} \subset X$, называется фундаментальной в метрическом пространстве $(X,\rho)$, если
	$$
	\forall \eps > 0 \ \exists N \in \N \colon \  \forall n, m \geq N \Rightarrow \rho(x_n,x_m) < \eps 
	$$
\end{definition}
\begin{definition}
	Метрическое пространство называется полным, если любая фундаментальная последовательность является сходящейся. 
\end{definition}

\begin{definition}
	Подмножество метрического пространства $S \subset X$, называется всюду плотным в $(X, \rho)$, если 
	$$
	\forall x \in X, \ \forall U(x) \in \tau_\rho, \ \exists s \in S \Rightarrow s \in U(x)
	$$
\end{definition}

\begin{definition}
	Пусть $(Z, \rho)$ --- метрическое пространство. Множество $S \subset Z$ называется нигде не плотным, если 
	$$
	\Int [S] = \varnothing
	$$
\end{definition}
\begin{remark}
	Очевидно, что это определение равносильно:
	$$
	\forall r > 0, \ \forall z \in Z \Rightarrow B_r(z) \nsubseteq [S]
	$$
	Я использую замкнутый шар, потому что если нельзя впихнуть открытый, то взяв радиус поменьше, не впихнется и замкнутый (это следствие аксиом отделимости для метрических пространств)
\end{remark}

\begin{definition}
	\hfill 
	\begin{itemize}
		\item Метрическое пространство $(Z, \rho)$ --- называется первой категории по Бэру, если 
		$$
		Z = \bigcup_{n=1}^\infty S_n
		$$
		Где $S_n$ --- нигде не плотные множества
		\item Если $(Z, \rho)$ не является первой категории по Бэру, то оно называется второй категории по Бэру.
	\end{itemize} 
\end{definition}

\begin{theorem}[Бэр]\label{th:bear}
	Пусть $(Z, \rho)$ --- полное метрическое пространство (в частности $(Z, \|\|)$ --- банахово), тогда $(Z, \rho)$ --- второй категории.
\end{theorem}
\begin{proof}
	Пусть $Z$ представлено в виде счетного объединения некоторых множеств.
	$$
	Z = \bigcup_{n=1}^\infty S_n , \ S_n \subset Z
	$$
	Предположим, что все $S_n$ --- нигде не плотные. То есть $\Int [S_n] = \varnothing$. 
	Построим фундаментальную последовательность, сходящуюся к точке не лежащей в $Z$, что будет являться противоречием. Возьмем $z_0 \in Z$, тогда
	$$
	B_1(z_0) \nsubseteq [S_1] \Rightarrow B_1(z_0)\setminus [S_1] \text{ --- открыто и имеет не пустую внутренность	}
	$$
	Тогда $\exists r_1 \leq \frac{1}{2}$, $\exists z_1 \in Z$:
	$$
	B_{r_1}(z_1) \subset B_1(z_0)\setminus [S_1]
	$$
	Тогда посмотрим на $B_{r_1}(z_1)$ в силу того, что $S_2$ --- нигде не плотно:
	$$
	B_{r_1}(z_1) \setminus [S_2] \neq \varnothing
	$$
	Значит $\exists r_2 \leq \displaystyle \frac{1}{2^2}$, $\exists z_2 \in Z$:
	$$
	B_{r_2}(z_2) \subset B_{r_2}(z_1) \setminus [S_2]
	$$
	Кроме того, по построению:
	$$
	B_{r_2}(z_2) \subset B_{r_1}(z_1) \subset B_{r_0}(z_0)
	$$
	Продолжая процесс получим последовательность $\{z_n\} \subset Z$, она фундаментальна:
	$$
	\forall \eps > 0: \exists N: \frac{1}{2^N} < \eps: \forall n, m \geq N \ z_n,z_m \in B_{r_N}(z_N) \Rightarrow \rho(z_n,z_m) < \eps
	$$ 
	Так как, $Z$ --- полно, то $\exists z^* \in Z: z_n \xrightarrow{\rho} z^*$. Но по построению для любого $S_N$ верно $\forall n \geq N + 1: \ z_n \notin [S_n]$. Таким образом
	$$
	z^* \notin \bigcup_{n=1}^\infty S_n = Z
	$$
	Полученное противоречие доказывает теорему.
\end{proof}