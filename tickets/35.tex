\newpage
\section{Слабая компактность замкнутого шара в рефлексивном пространстве. Существование проекции точки на замкнутое подпространство рефлексивного пространства.}
\begin{theorem}
	Пусть $X$ --- рефлексивно, тогда $B_1(0)$ является слабым компактом.
\end{theorem}
\begin{proof}
	Запишем равенство множеств функционалов
	$$
	(X^*, \tau_{w^*})^* = F(X) = (X^*, \|\|)^* = X^{**}
	$$
	Первое равенство обеспечено теоремой Шмульяна, второе равенство --- определение рефлексивности. Теперь рассмотрим слабую* топологию в $X^{**}$ (второе сопряженное относительно нормы)
	$$
	V^{**}(\Phi, f, \eps) = \{\Psi \in X^{**} \mid |\Psi(f) - \Phi(f)| < \eps\}
	$$
	В силу рефлексивности все такие функционалы $\Phi$ порождаются элементом $x \in X$, то есть 
	$$
	\{\Psi \in X^{**} \mid |\Psi(f) - \Phi(f)| < \eps\} = V_{**}(F(x), f, \eps)
	$$
	Таким образом слабая* топология (относительно нормируемой топологии в $X^*$) и $\tau_{w^{**}}$ в $X^{**}$ совпадают, значит это одно и тоже пространство. 
	
	Но в $(X^*, \|\|)^*$ шар $B^{**}_1(0)$, являясь полярой $B^{*}_1(0)$, по теореме Банаха-Алаоглу (\ref{th:banach-alaoglu} ) слабо* компактен. Значит он является и $\tau_{w^{**}}$-компактом в $X^{**}$, тогда его прообаз под действием гомоморфизма $F$ является $\tau_w$-компактом, что и требовалось.
\end{proof}

\begin{claim}
	Пусть $X$ --- рефлексивно. Пусть множество $S \subset X$ является выпуклым и замкнутым. Тогда для любого вектора $x \in X$ в множестве $S$ существует ближайший элемент, т.е. вектор $y = y(x) \in S$, такой, что 
	$$
	\|x-y\| = \rho(x,S) = \inf\limits_{z \in S} \|x - z\|
	$$
\end{claim}
\begin{proof}
	По определению инфинума существует минимизирующая последовательность $\{z_n\} \subset S$: 
	$$
	\rho(x,S) = \lim\limits_{n \to \infty} \|x - z_n\|
	$$
	Так как 
	$$
	\|z_n\| \leq \|x\| + \|x - z_n\|
	$$
	а сходящаяся числовая последовательность $\|x-z_n\|$ является ограниченной, то последовательность $z_n$ является ограниченной в пространстве $X$, а значит вкладывается в шар $B_R(0)$. Так как $X$ --- рефлексивно, то $B_R(0)$ является слабым компактом. Тогда по теореме Эберлейна-Шмульяна, $B_R(0)$ является слабым секвенциальным компактом, а значит ${z_n}$ содержит слабо сходящуюся подпоследовательность $\{z_{n_k}\}$ к вектору $y \in X$. По теореме Мазура выпуклое замкнутое множество является слабо замкнутым, значит $y \in S$. По следствию теоремы Хана-Банаха существует функционал $f \in X^*$ вида 
	$$
	\|f\| = 1 \text{ и } |f(x-y)| = \|x-y\|
	$$
	Тогда получаем
	$$
	\rho(x,S) \leq \|x-y\| = |f(x-y)| = \lim\limits_{k\to \infty} |f(x - z_k)| \leq \lim\limits_{k\to\infty} \|x-z_{n_k}\| = \rho(x,S)
	$$
	Откуда моментально получаем равенство $\rho(x,S)  = \|x - y\|$. Что и требовалось.
\end{proof}

