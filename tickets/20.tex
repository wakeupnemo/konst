\newpage
\section{Равномерная операторная топология $\tau_u$ в пространстве $\CL(X, Y)$ линейных ограниченных операторов, действующих в нормированных пространствах $X$ и $Y$. Теорема о полноте пространства $(\CL(X,Y), \tau_u)$.}

\begin{definition}
	Пусть $X,Y$ --- линейные нормированные пространства относительно поля $\Cx$. Тогда множество всех непрерывных линейных отображений обозначается:
	$$
	\mathcal{{L}}(X, Y) = \{A: X \to Y \mid A \text{ --- линеен на $X$}, \ A \text{ --- непрерывен на $X$}\}
	$$
\end{definition}

\begin{claim}
	$\CL(X,Y)$ --- линейное пространство.
\end{claim}
\begin{proof}
	очевидно. 
\end{proof}

\begin{claim}
	В пространстве $\CL(X,Y)$ можно ввести норму:
	\begin{equation}
		\|A\|_{op} = \sup\limits_{\|x\|_X \leq 1} \|A(x)\|_Y = \sup\limits_{\|x\|_X = 1} \|A(x)\|_Y = \sup\limits_{x \neq 0}\frac{\|A(x)\|_Y}{\|x\|_X}
	\end{equation}
\end{claim}
\begin{remark}
	Далее я не буду писать индексы у норм. Чтобы понять какая из норм имеется в виду в том или ином случае, необходимо посмотреть на аргумент. 
\end{remark}
\begin{proof}
	Проверим все аксиомы нормы
	\begin{itemize}
		\item$A \in \CL(X,Y) \Rightarrow 0 \leq \|A\| < \infty$
		\item Из последнего равенства формулы (1) имеем
		$$
		\|A\| = 0 \Leftrightarrow \forall x \in X: \ A(x) = 0 \Leftrightarrow A = 0
		$$
		\item $ \lambda \in \Cx$ из первого равенства из формулы (1):
		$$
		\|\lambda A\| = |\lambda| \|A\|
		$$
		\item Неравенство треугольника следует из неравенства треугольника для соответствующей нормы:
		$$
		\|Ax + Tx\| \leq \|Ax\| + \|Tx\|
		$$
		Переходя к супремуму по единичному шару получаем требуемое.
	\end{itemize}
\end{proof}
{ \footnotesize \color{violet}
\begin{claim} \label{cl:linfuncon}
	Линейный функционал $A : X \to Y$ непрерывен тогда и только тогда, когда его норма конечна
\end{claim}
\begin{proof}
	\begin{enumerate}
		\item[$\Rightarrow$] Пусть $A \in \CL(X, Y)$. Тогда для $\eps =1$ воспользуемся непрерывностью $A$ в нуле, тогда $\exists \delta > 0$: 
		$$
		\forall \|x\| \leq \delta: \ \|A(x)\| \leq 1 
		$$
		Тогда $\forall x \in X: \  \|x\| \leq 1$:
		$$
		\|A(x)\| = \frac{1}{\delta}\|A(\delta x)\| \leq  \frac{1}{\delta}
		$$
		
		Значит норма $\|A\| \leq \frac{1}{\delta}$, то есть норма конечна.
		\item[$\Leftarrow$] Если норма оператора конечна, то 
		$$
		\forall x \in X: \ \|A(x)\| \leq \|A\| \|x\|
		$$
		Тогда 
		$$
		\|A(x_1) - A(x_2)\| \leq \|A\| \|x_1 - x_2\|
		$$
		То есть оператор является липшецевым с константой $\|A\|$ откуда сразу следует его непрерывность.
	\end{enumerate}
\end{proof}
}
\begin{definition}
	$\tau_U$ --- топология в $\CL(X,Y)$ порожденная операторной нормой. Называется равномерной операторной топологией.
\end{definition}
\begin{remark}
	Индекс $U$ подчеркивает, что эта топология обеспечивает равномерную сходимость операторов на единичном шаре.
\end{remark}

\begin{theorem}
	Пусть $Y$ --- банахово пространство, тогда $(\CL(X,Y), \tau_U)$ --- полное.
\end{theorem}
\begin{proof}
	Возьмем $\tau_U$-фундаментальную последовательность $\{A_n\} \subset \CL(X,Y)$, то есть:
	$$
	\forall \eps > 0 : \exists N(\eps): \ \forall n,m \geq N(\eps): \ \|A_n -A_m\| \leq \eps 
	$$
	Значит
	$$
	\forall x \in X: \exists N\left(\frac{\eps}{\|x\| +1}\right) :\ \|A_n(x) - A_m(x)\| \leq \|A_n - A_m\| \|x\| \leq \eps
	$$
	Значит для любого $x$ последовательность $\{A_n(x)\} \subset Y$ --- фундаментальна, тогда в силу полноты $Y$ она сходится.
	Тогда положим:
	$$
	T: X \to Y \quad T(x) = \lim\limits_{n \to \infty} A_n(x) \in Y
	$$
	В силу линейности предела и операторов $A_n$, $T$ --- линейный оператор. Покажем, что он непрерывен. Рассмотрим произвольное $x \in X, \|x\| \leq 1$ имеем:
	$$
	\|Tx\| = \lim\limits_{n \to \infty}{\|A_n x\|} \leq \lim\limits_{n \to \infty}\|A_n\|
	$$
	Кроме того по неравенству треугольника: 
	$$
	\forall n,m \geq N(\eps): \left|\|A_n\| - \|A_m\|\right| \leq \|A_n - A_m\| \leq \eps
	$$
	Тогда $\{\|A_n\|\} \subset \R$ --- фундаментальная числовая последовательность, и в силу полноты $\R$ имеет предел, значит: 
	$$
	\|Tx\| \leq \lim\limits_{n \to \infty} \|A_n\| < \infty
	$$
	Таким образом оператор $T$ --- ограничен и значит непрерывен, тогда $T \in \CL(X, Y)$
	Теперь нам надо показать сходимость к $T$ по операторной норме. Опять пусть $x \in X: \ \|x\| \leq 1$, имеем:
	$$
	\forall n, m \geq N(\eps): \ \|A_n(x) - A_m(x)\| \leq \eps
	$$
	Переходим к пределу по $m$ и в силу непрерывности нормы получаем:
	$$
	\forall n \geq N(\eps): \ \|A_n(x) - T(x)\| \leq \eps 
	$$
	Теперь беря супремум по всем $x$ из шара получаем:
	$$
	\|A_n - T\|  \leq \sup\limits_{\|x\| \leq 1}\|A_n(x) - T(x)\| \leq \eps
	$$
	Тогда мы получили сходимость по операторной нормы и $\|T\| = \lim\limits_{n \to \infty}\|A_n\|$
\end{proof}