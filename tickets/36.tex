\newpage
\section{Теорема Рисса-Фреше о представлении сопряженного гильбертова пространства. Рефлексивность гильбертова пространства.}
\begin{claim}
	Пусть $(X, \|\|)$ --- комплексно линейное нормированное пространство. Тогда линейное отображение $F \colon X \to X^{**}$ вида 
	$$
	(Fx)f = f(x) \quad \forall x \in X, \ \forall f \in X^*
	$$
	осуществляет изометрический изоморфизм из $X$ на подпространство $\Ima F \subset X^{**}$, то есть взаимно однозначно и сохраняет норму. 
\end{claim}
\begin{proof}
	Покажем, что отображение инъективно, пусть $x,y \in X$ и $Fx = Fy$, тогда для любого $f \in X^*$ выполнено 
	$$
	f(x) = f(y)
	$$
	По следствию теоремы Хана-Банаха $x = y$. Таким образом отображение взаимно однозначно отображет $X$ на $\Ima F$. Далее для норм в силу следствия теоремы Хана-Банаха имеем
	$$
	\|Fx\| = \sup\limits_{\substack{f \in X^* \\ \|f\| =1 }} |(Fx)f| = \sup\limits_{\substack{f \in X^* \\ \|f\| =1 }} |f(x)| = \|x\|
	$$
	что и требовалось.
\end{proof}
\begin{definition}
	Комплесно линейное нормированное пространство $(X, \|\|)$ называется рефлексивным, если
	$$
	\Ima F = X^{**}
	$$
\end{definition}
\begin{remark}
	Обратите внимание, что в общем случае равенство множеств $X$ и $X^{**}$ не означает рефлексивность. Важно, что образ конкретного отображения совпадает со всем $X^{**}$.
\end{remark}
\begin{theorem}[Риса, Фреше]
	Пусть $H$ --- гильбертово пространство, тогда $$\forall f \in H^* \quad  \exists! z = z(f) \in H$$ Причем $$\|z\| = \|f\|, \quad \forall x \in H\colon f(x) = (x,z(f)) $$
	При этом отображение $z$ обладает следующими свойствами
	\begin{itemize}
		\item $\forall f,g \in H^*: \ z(f + g) = z(f) + z(g)$
		\item $\forall \lambda \in \Cx: z(\lambda f) = \bar{\lambda} z(f)$
	\end{itemize}
\end{theorem}
\begin{definition}
	Изометрия $z$ не является обычным изоморфизмом, так как скаляры выносятся с сопряжением. Это изометрия является сопряженно-линейной аддитивной изометрией, называется  \textit{изометрией Риса-Фреше} и обозначается $\Phi : H^* \to H$ и обладает следующим свойством 
	$$
	\forall f \in H^* \Rightarrow \forall x \in H \colon f(x) = (x, \Phi(f))
	$$
\end{definition}
\begin{definition}
	Пусть $A \in \CL(H)$, $H$ --- гильбертово, эрмитово сопряженный оператор $A^+ \in L(H)$ определяется как 
	$$
	A^+  =\Phi \circ A^* \circ \Phi^{-1}
	$$
\end{definition}
\begin{remark}
	Ясно что эрмитово сопряженный оператор удовлетворяет свойству:
	$$(Ax, y) = (\Phi^{-1}(y))(Ax) = (A^*\Phi^{-1}y)(x) = (x, \Phi A^* \Phi^{-1}y) = (x,A^+y)$$
\end{remark}
\begin{definition}
	$A \in L(H)$ называется эрмитовым или самосопряженным по Эрмиту если $A^+ = A$
\end{definition}

\begin{proof}[Доказательство теоремы Риса-Фреше]
	Рассмотрим $f \in H^*$, если $ f = 0 \Rightarrow z(f) = 0$ подойдет. Если $f \neq 0$, тогда $\Ker f \neq H$, тогда $\Ker f$ --- замкнутое подпространстве в $H$. Тогда в силу теоремы Риса об ортогональном дополнении можно рассмотреть
	$$
	\Ker f \oplus (\Ker f)^\perp  = H
	$$
	Рассмотрим $x_0 \in (\Ker f)^\perp \setminus \{0\}$. Так как $f(x_0) \neq 0$, то 
	$$
	\forall x \in H\Rightarrow x =\frac{f(x)}{f(x_0)}x_0 + \left(x - \frac{f(x)}{f(x_0)}x_0\right)
	$$
	Тогда так как $f\left(x - \frac{f(x)}{f(x_0)}x_0\right) = f(x) - f(x) = 0$, то $x - \frac{f(x)}{f(x_0)}x_0  \in \Ker f$ а значит ортогональна $x_0$, тогда
	$$
	(x,x_0) = \frac{f(x)}{f(x_0)}(x_0,x_0) \Rightarrow  \forall x \in H \colon f(x) = \left(x, \frac{\overline{f(x_0)}}{(x_0,x_0)}x_0\right)
	$$
	Определим отображение $z\colon H^* \to H$ 
	$$
	z(f) = \frac{\overline{f(x_0)}}{(x_0,x_0)}x_0
	$$
	В силу неравенства Коши-Буняковского $\|f\| \leq \|z\|$ с другой стороны
	$$
	|z(f)| \leq \frac{|f(x_0)|}{\|x_0\|} \leq \|f\|
	$$
	Таким образом $z$ --- изометрия. Аддитивность и сопряженная однородность вытекает из формул. Таким образом теорема доказана.
\end{proof}
\begin{next0}
	Гильбертово пространство рефлексивно
\end{next0}
\begin{proof}
	Нужно доказать, что образ отображения $F \colon H \to H^{**}$ 
	$$
	(Fx)(f) = f(x) \quad \forall x \in H, \ \forall f \in H^{*}
	$$
	совпадает со всем $H^{**}$. Пусть $\Phi \in H^{**}$, пусть $z\colon H^{*} \to H$ изометрия Риса-Фреше, существование которой доказано в предыдущей теореме. Для каждого $\Phi$ определим функционал  
	$$
	f = \overline{\Phi \circ z^{-1}}
	$$
	тогда для любого $x \in H$ имеем равенство 
	$$
	|f(x)| \leq \|\Phi\| \|z^{-1}(x)\|  = \|\Phi\| \|x\| 
	$$
	то есть $\|f\| \leq \|\Phi\|$. Cледовательно функционал $f$ --- непрерывен, то есть $f \in H^*$. Теперь определим вектор 
	$$
	y = z(f)
	$$
	Тогда для любого функционала $g \in H^*$ получим
	$$
	(Fy)(g) = g(y) = (z(f), z(g)) = \overline{f(z(g))} = \Phi(z^{-1}(z(g))) = \Phi(g)
	$$
	То есть $Fy = \Phi$. Что и требовалось.
\end{proof}