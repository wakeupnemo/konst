\newpage
\section{Критерий метризуемости слабой* топологии в сопряженном пространстве локально выпуклого топологического пространства. Неметризуемость слабой* топологии в сопряженном пространстве бесконечномерного пространства.}


\begin{claim}
	Если $X$ --- бесконечномерное банахово пространство, то в $(X, \|\|)$ нет счетного базиса Гамеля
\end{claim}
\begin{proof}
	Является простым следствием теоремы Бэра \ref{th:bear} и замкнутости конечномерного подпространства \ref{th:clfinds}
\end{proof}
\begin{claim}
	Пусть $\tau_{w^*}$ в $X^*$ является метризуемой. Тогда в $X$ есть счетный базис Гамеля.
\end{claim}
\begin{proof}
	Предположим, что $\tau_{w^*}$ --- метрическая, пусть $\rho_*$ --- метрика, тогда, существует счетная локальная база:
	$$
	\beta = \left\{O_{\frac{1}{n}}^{\rho_*}(0)\right\}_{n = 1}^\infty
	$$
	Обладая такой локальной базой, мы понимаем, что в любой такой шар ноль входит как $\tau_{w^*}$~-~внутренняя точка, значит 
	$$
	\forall n \in \N \ \exists x^{(n)}_1, \dots, x^{(n)}_{N_n} \in X \colon \bigcap_{k = 1}^{N_n} V_*\left(0, x_k^{(n)}, \eps_n\right)\subset O_{\frac{1}{n}}^{\rho_*}(0)
	$$
	Далее будем рассуждать как при доказательстве теоремы Шмульяна (\ref{th:shulian}). Рассматриваем 
	$$
	M = \left\{\left\{x_k^{(n)}\right\}_{k=1}^{N_n} \mid n \in \N\right\}
	$$
	Далее мы покажем, что $\Lin M = X$, тогда так как $M$ --- не более чем счетно, выделяя из $M$ максимальную систему линейно независимых векторов (Лемма Цорна (\ref{lem:zorn}) и бла-бла) получим счетный базис Гамеля. 
	
	Пусть $x \in X$. Берем окрестность нуля $U_*(0) = V_*(0,x,1)$. Эта окрестность нуля содержит элемент счетной локальной базы, который в свою очередь содержит пересечение стандартных элементов базы $\tau_{w^*}$
	$$
	\exists n: \  O_{\frac{1}{n}}^{\rho_*}(0) \supset \bigcap_{k = 1}^{N_n} V_*\left(0, x_k^{(n)}, \eps_n\right)
	$$
	По этим векторам и разложится $x$. Рассмотрим каноническое отображение 
	$$(Fx) \in (X^*, \tau_{w^*})^*\colon (Fx)(f) = f(x)$$
	Тогда если мы возьмем функционал 
	$$
	f \in \bigcap_{k=1}^{N_n} \Ker \left(Fx_k^{(n)}\right)
	$$
	то $f$ автоматически попадает в $\bigcap_{k = 1}^{N_n} V_*(0, x_k^{(n)}, \eps_n)$, откуда $f$ попадает в шар $O_{\frac{1}{n}}^{\rho_*}(0)$, но пересечение ядер является подпространством, значит туда же попадет $t f$ для любого скаляра $t \in \Cx$, значит $f \in \Ker Fx$. То есть 
	$$
	\bigcap_{k=1}^{N_n} \Ker \left(Fx_k^{(n)}\right) \subset \Ker Fx
	$$
	Далее рассуждения дословно повторяют доказательство \ref{th:shulian}. Получаем, что $x$ раскладывается по векторам $x_{k}^{(n)}$, что и требовалось.
\end{proof}
\begin{theorem}
	Слабая* топология в бесконечномерных сбанаховых пространствах не метризуема.
\end{theorem}
\begin{proof}
	Тривиальное следствие двух предыдущих утверждений. 
\end{proof}
