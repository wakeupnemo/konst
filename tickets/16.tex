\newpage
\section{Гильбертово пространство. Теоремы Рисса о проекции и об ортогональном разложении в гильбертовом пространстве.}
\begin{definition}
	Пусть $X$ --- комплексное линейное пространство. Скалярным произведением в $X$ называется отображение 
	$$
	(\cdot, \cdot) \colon X \times X \to \Cx
	$$
	удовлетворяющее свойствам 
	\begin{enumerate}
		\item для любого $x \in X$ число $(x,x) \in \R$ и выполнено $(x,x) \geq 0$;
		\item $(x,x) = 0 \Leftrightarrow x = 0$;
		\item для любых $x,y \in X$ выполнено $(x,y) = \overline{(y,x)}$;
		\item для любых $x,y,z \in X$ и $\alpha, \beta \in \Cx$ выполнено $(\alpha x + \beta y, z) = \alpha (x,z) + \beta (y,z)$.
	\end{enumerate}
\end{definition}
\begin{definition}
	Линейное пространство с фиксированным в нем скалярным произведением называется евклидовым. 
\end{definition}
\begin{claim}
	Пусть $X$ --- евклидово пространство. Тогда величина 
	$$
	\|x\| = \sqrt{(x,x)}
	$$
	удовлетворяет определению нормы на $X$.
\end{claim}

\begin{definition}
	Евклидово пространство, полное относительно нормы, порожденной скалярным произведением, будем называть гильбертовым пространством.
\end{definition}
\begin{definition}
	Пусть $(X,\|\|)$ --- линейное нормированное пространство, множество $S \subset X$, вектор $x \in X$. Вектор $y \in S$ называется метрической проекцией вектора $x$ на множество $S$, если справедливо равенство
	$$
	\|x-y\| = \rho(x,S) = \inf_{z \in S} \|x - z\|
	$$
\end{definition}
\begin{claim}
	В гильбертовом пространстве $\CH$ справедливы неравенство Коши-Буняковского и равенство параллелограммов
	$$
	|(x,y)| \leq \|x\|\|y\| \quad \|x - y\|^2 + \|x + y\|^2 = 2\|x\|^2 + 2\|y\|^2
	$$
\end{claim}
\begin{theorem}[Рисс, о проекции]
	Пусть $\CH$ --- гильбертово пространство, $S \subset \CH$ --- выпуклое замкнутое множество. Тогда для любого $x \in \CH$ существует единственный вектор $y \in S$, который является метрической проекцией вектора $x$ на множество $S$.
\end{theorem}
\begin{proof}
	По определению инфинума 
	$$
	\exists \{z_m\}_{m=1}^\infty \subset S \Rightarrow \rho(x,S) = \lim\limits_{m \to \infty} \|x - z_m\|
	$$
	Покажем, что $\{z_m\}$ --- фундаментальна. В силу равенства параллелограммов:
	$$
	\|z_m - z_n\|^2 = \|(z_m - x)  - (z_n - x)\|^2 = 2\|z_m - x\|^2 + 2\|z_n - x\|^2 - \|z_m + z_n - 2x\|^2 
	$$
	В силу выпуклости множества $S$:
	$$
	\frac{z_m + z_n}{2} \in S
	$$
	Поэтому 
	$$
	\|z_m + z_n -2x\|^2 = 4\left\| \frac{z_n+z_m}{2} - x\right\|^2 \geq 4\rho^2(x,S)
	$$
	Тогда получаем
	$$
	\|z_m - z_n\|^2 \leq 2\|z_m - x\|^2 + 2\|z_n - x\|^2 - 4\rho^2(x,S) \to 0 \text{ при } n,m \to \infty
	$$
	Тогда $\{z_m\}$ --- фундаментальна и в силу полноты $\exists z = \lim\limits_{m \to \infty} z_m$. В силу замкнутости $z \in S$ при этом в силу неравенства треугольника 
	$$
	|\|x-z\| - \|x - z_m\|| \leq \|z - z_m\| \to 0
	$$
	То есть $\|x - z\| = \rho(x,S)$. Покажем единственность. Пусть $y \neq z$ и 
	$$
	\|x - y\| = \|x - z\| = \rho(x,S)
	$$
	Тогда в силу равенства параллелограммов 
	$$
	\|y - z\|^2 = \|(y-x) - (z - x)\|^2 = 2\|y -x\|^2 + 2 \|z - x\|^2 - \|y + z - 2x\|^2
	$$
	В силу выпуклости $\frac{y + z}{2} \in S$, тогда 
	$$
	\|y + z - 2x\| = 4\left\|\frac{y + z}{2} - x\right\|^2 \geq 4 \rho^2(x,S)
	$$
	Тогда 
	$$
	\|y -z\|^2 \leq 2\|y - x\|^2 + 2\|z-x\|^2 - 4\rho^2(x,S) = 0
	$$
	То есть $y = z$, что и требовалось. 
\end{proof}

\begin{definition}
	Пусть $\CH$ --- гильбертово пространство, $L \subset \CH$ --- подпространство. Ортогональным дополнением $L$ называется 
	$$
	L^\perp = \{x \in \CH \mid (x,y) = 0 \ \forall y	 \in L\}
	$$
\end{definition}
\begin{theorem}[Рисс, об ортогональном дополнении]
	Пусть $\CH$ --- гильбертово пространство, $L \subset \CH$ --- замкнутое подпространство. Тогда справедливо равенство 
	$$
	\CH = L \oplus L^\perp
	$$
\end{theorem}
\begin{proof}
	Так как $L$ --- подпространство, то оно является выпуклым, кроме того оно является замкнутым, значит в силу предыдущей теоремы $\forall x \in \CH$ существует и единственна метрическая проекция $y \in L$ такая что 
	$$
	\|x - y\| = \rho(x,L)
	$$
	Покажем, что $ z = x - y \in L^\perp$. Для любого вектора $a \in L$ и $t \in \R$ выполнено 
	$$
	\|x - y\| \leq \|x - y - ta\|
	$$
	так как $ y + ta \in L$. Следовательно 
	$$
	\|x-y\|^2 \leq \|x-y -ta\|^2 = \|x - y\|^2 + t^2 \|a\|^2 -2t\Rea(x-y,a)
	$$
	Тогда при $t>0$:
	$$
	\Rea(x-y,a) \leq \frac{t}{2}\|a\|^2 \to 0 \text{ при } t \to +0
	$$
	А при $t < 0$
	$$
	\Rea(x-y,a) \geq \frac{t}{2}\|a\|^2 \to 0  \text{ при } t \to -0
	$$
	Таким образом $\Rea(x-y,a) = 0$. Теперь рассматривая 
	$$
	\|x-y\| \leq \|x - y - ita\| 
	$$
	Получим $\Ima(x-y,a) = 0$, таким образом $(x-y,a) = 0$ для любого $a \in L$. То есть справедливо вложение
	$$
	x-y = z \in L^\perp
	$$
	Таким образом $\CH = L + L^\perp$. Но если $x \in L \cap L^\perp$, то $(x,x) = 0 \Leftrightarrow x = 0$, значит 
	$$
	\CH = L \oplus L^\perp
	$$
	Что и требовалось.
\end{proof}