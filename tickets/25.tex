\newpage
\section{Теорема о приближении компактного оператора в пространстве $\CL(X,H)$ с равномерной операторной топологией конечномерным оператором для гильбертова пространства $H$. }
\begin{definition}
	Оператор $A$ называется конечномерным, если он является пределом последовательности линейных непрерывных операторов с конечномерными образами. 
\end{definition}
Очевидно, что любой линейный непрерывный оператор с конечномерным образом является компактным, тогда в силу замкнутости пространства компактных операторов, любой конечномерный оператор компактен. Оказывается для гильбертого пространства верно и обратное. 

\begin{theorem}
	Пусть $Y = \CH$ --- гильбертово пространство, а линейный оператор $A \in \CK(X,Y)$, тогда $A$ --- конечномерный.
\end{theorem}
\begin{proof}
	Так как $A$ --- компактный, то множество $AB_1(0)$ является вполне ограниченным в $\CH$. Следовательно $\forall \eps > 0$ существует конечная $\eps$-сеть 
	$$
	\{y_m\}_{m=1}^M \subset A(B_1(0))
	$$
	для множества $AB_1(0)$. Определим 
	$$
	L_\eps = \Lin \{y_1, \dots, y_M\} \subset \CH
	$$
	Так как подпространство $L_\eps$ --- конечномерно, то оно замкнуто в $\CH$. Тогда по теореме Риса об ортогональном дополнении 
	$$
	\CH = L_\eps \oplus (L_\eps)^\perp
	$$
	Поэтому для любого вектора $y \in \CH$ существуют единственные векторы 
	$$
	y_{||} \in L_\eps, \quad y_\perp \in (L_\eps)^\perp
	$$
	Такие что $y = y_{||} + y_\perp$. Определим оператор проекции 
	$$
	P_\eps \colon \CH \to \CH
	$$
	по формуле 
	$$
	P_\eps(y) = y_{||}
	$$
	Так как
	$$
	\|y\| = \sqrt{(y_{||} + y_\perp,y_{||} + y_\perp)}= \sqrt{\|y_{||}\|^2 + \|y_\perp\|^2} \geq \|y_{||}\| = \|P_\eps(y)\|
	$$
	то $\|P_\eps\| \leq 1$ и $P_\eps \in \CL(\CH,\CH)$. Определим оператор 
	$$
	A_\eps = P_\eps A \in \CL(X, \CH)
	$$
	Так как $\Ima A_\eps \subset L_\eps$, то $A_\eps$ имеет конечномерный образ. Рассмотрим произвольный $x \in B_1(0)$. Для него существует $m \in \overline{1, M}$, такой, что 
	$$
	\|A(x) - y_m\| \leq \eps
	$$
	При этом по определению $P_\eps$ справедливо 
	$$
	P_\eps(y_m) = y_m
	$$
	Тогда 
	$$
	\|A(x) - A_\eps(x)\| \leq \|A(x) - y_m\| + \|P_\eps(y_m - A_x)\| \leq (1 + \|P_\eps\|)\|A(x) - y_m\| \leq 2\eps
	$$
	В силу произвольности $x \in B_1(0)$ получаем 
	$$
	\|A - A_\eps\| \leq 2\eps
	$$
	Взяв $\eps_n = \frac{1}{n}$, получим  $A_{\eps_n} \xrightarrow{\|\|} A$ при $n \to \infty$. Где $A_{\eps_n}$ --- операторы с конечномерными образами. Что и требовалось.
\end{proof}
