\newpage
\section{Топологическое и секвенциальное определение замкнутости и замыкания множества топологического пространства, связь между ними. Аксиома счетности}
\begin{definition}
	$(X,\tau)$ --- топологическое пространство, тогда если $S \subset X: \ X \setminus S \in \tau$, то $S$ называется топологически замкнутым в $(X,\tau)$.
\end{definition}
\begin{definition}
	$(X, \tau)$ --- топологическое пространство, $S \subset X$. Тогда $x \in X$ называется топологической точкой прикосновения $S$, если 
	$$
	\forall U(x) \in \tau \Rightarrow U(x) \cap S \neq \varnothing
	$$
\end{definition}
\begin{definition}
	$(X,\tau)$ --- топологическое пространство, $S \subset X$. Тогда множество
	$$
	[S]_\tau \stackrel{def}= \{x \in X \mid x \text{ --- топологическая точка прикосновения } S \}
	$$
	Называется топологическим замыканием $S$
\end{definition}
\begin{definition}
	$(X, \tau)$ --- топологическое пространство. $\{x_n\}_{n=1}^\infty \subset X$, $x \in X$. Говорят, что последовательность $\{x_n\}$ сходится по топологии к $x$. Пишут:
	$$
	x_n \xrightarrow[n \to \infty]{\tau} x \  \deff \forall U(x) \in \tau \ \exists N \in \N : \ \forall n \geq N \Rightarrow x_n \in U(x)
	$$
\end{definition}
\begin{definition}
	$(X,\tau)$	 --- топологическое пространство, $S \subset X$. Тогда $x \in X$ называется секвенциальной точкой прикосновения $S$ если
	$$
	\exists \{x_n\}_{n=1}^\infty \subset S : \ x_n \stackrel{\tau}{\longrightarrow} x 
	$$
\end{definition}
\begin{claim}
	Всякая секвенциальная точка прикосновения является топологической. Обратное неверно.
\end{claim}
\begin{proof}
	$x \in X$ --- секвенциальная точка прикосновения $S$, тогда существует $\{x_n\}_{n=1}^\infty \subset S$:
	$$
	\forall U(x) \in \tau \ \exists N \in \N \ \forall n \geq N \Rightarrow x_n \in U(x)  \Rightarrow U(x) \cap  S \neq \varnothing 
	$$
	Тогда $x$ --- топологическая точка прикосновения. 
	
	Рассмотрим топологию Зарисского на оси $(\bb{R}, \tau_z)$, где:
	$$
	\tau_z = \{G \subset \R \mid G \neq \varnothing, \R \setminus G \text{ не более чем счетно}\} \cup \{\varnothing\}
	$$
	Пусть $x_n \stackrel{\tau_z}{\longrightarrow} x$ Тогда 
	$$
	U(x) = \R \setminus \{x_n \neq x\} \in \tau \Rightarrow \ \exists N: \ \forall n \geq N : \ x_n \in U(x) \Rightarrow  x_n = x 
	$$
	Тогда $\forall S \subset \R, S \neq \varnothing$ $x \in X$ --- секвенциальная точка прикосновения $S$ iff $x \in S$. То есть секвенциальными точками прикосновения множества $S$ могут быть только точки этого множества. С другой стороны пусть $S \subset \R$, $|S| = |\R|$. Тогда $\forall x \in \R \setminus S$ $\forall U(x) \ U(x) \cap S \neq \varnothing$, так как $|S| > |\N|$. Таким образом, $x$ --- топологическая точка прикосновения.
\end{proof}
\begin{definition}
	$(X, \tau)$ --- топологическое пространство. $S \subset X$. Тогда $S$ --- называется секвенциально замкнутым, если 
	$$S = \{x \in X \mid x \text{ --- секвенциальная точка прикосновения } S\}$$ 
	При этом множество 
	$$
	[S]_{\text{секв}} = \{ \text{все секвенциальные точки прикосновения } S\}
	$$
	Называется секвенциальным замыканием $S$.
\end{definition}
\begin{remark}
	Из утверждения выше следует что $[S]_\text{секв} \subset [S]_\tau$
\end{remark}
\begin{definition}
	$(X,\tau)$ --- ТП. $x \in X$, тогда $B(x)$ --- некоторое подмножество окрестностей $x$ называется локальной базой $x$, если:
	$$
	\forall U(x) \in \tau \ \exists V \in B(x): \ V \subset U(x)
	$$
\end{definition}
\noindent \hypertarget{fcs}{\textbf{Аксиома cчетности.}} Если $	\forall x \in X \ \exists B(x) = \{V_n\}_{n=1}^\infty$ --- cчетная локальная база $x$, то говорят, что $(X, \tau)$ удовлетворяет аксиоме счетности.
\begin{remark}
	При выполнении аксиомы счетности часто удобно считать, что элементы локальной базы упорядоченны по вложению. Заметим, что это всегда можно осуществить положив 
	$$
	W_n = \bigcap_{k=1}^n V_k
	$$
	Где $V_k$ --- элементы исходной локальной базы.
\end{remark}
\begin{theorem}
	Пусть в $(X, \tau)$ --- топологическом пространстве выполнена аксиома счетности, $S \subset X$, тогда $[S]_\tau = [S]_\text{секв}$. То есть любая топологическая точка прикосновения является секвенциальной.
\end{theorem}

\begin{proof}
	Имеем $B(x) = \{W_n\}_{n=1}^\infty$ --- счетную локальную база, такую что 
	$$W_1 \supset W_2 \supset W_3 \supset \dots$$ 
	Тогда 
	$$
	\forall x \in [S]_\tau \deff \forall U(x) \in \tau \Rightarrow U(x)\cap S \neq \varnothing  
	$$
	Тогда это выполнено и для элементов локальной базы $B(x) = \{W_n\}_{n=1}^\infty$: 
	$$
	\forall n \in \N \Rightarrow W_n \cap S \neq \varnothing \Rightarrow \ \exists x_n \in W_n \cap S 
	$$
	Значит существует $\{x_n\}_{n=1}^\infty$. Тогда:
	$$
	\forall U(x) \exists N \in \N \ W_N \subset U(x) (\text{определение локальной базы}) \Rightarrow \forall n \geq N: \ x_n \in W_n \subset W_N \subset U(x)
	$$
	Значит $x_n \stackrel{\tau}{\rightarrow} x$, то есть $x$ --- секвенциальная точка прикосновения.
\end{proof}
