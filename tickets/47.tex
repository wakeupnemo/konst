\newpage
\section{Комплексные гомоморфизмы и спектр элемента банаховой алгебры. Теорема Гельфанда о спектре элемента коммутативной банаховой алгебры}


\begin{definition}
	Нетривиальный линейный функционал $h \colon \CA \to \Cx$, такой, что для любых $x,y \in \CA$
	$$
	h(xy) = h(x)h(y)
	$$
	назовем комплексным гомоморфизмом на $\CA$.
\end{definition}
Обозначим 
$$
Z(\CA) = \{h \colon \CA \to \Cx \mid h \text{ --- комплексный гомоморфизм}\}
$$
Следующая серия лемм очень проста в доказательстве. Читателю рекомендуеются попытаться сделать это самому.
\begin{lemma}
	Для комплкесного гомоморфизма $h(e) = 1$, где $e$ --- единица по умножению банаховой алгебры.
\end{lemma}
\begin{proof}
	Так как функционал $h \in Z(\CA)$ нетривиальный, по следствию теоремы Хана-Банаха, для любого $h$ найдется $x_0$, что $h(x_0) \neq 0$. Тогда 
	$$
	h(x_0) = h(x_0 e) = h(x_0) h(e) \Rightarrow h(e) = 1
	$$
\end{proof}
\begin{next0}
	Для любого обратимого элемента $x \in G(\CA)$ $h(x)\neq 0$.
\end{next0}
\begin{proof}
	В самом деле,
	$$
	1 = h(e) = h(x x^{-1}) = h(x) h(x^{-1}).
	$$
\end{proof}
\begin{lemma}
	Любой комплексный гомоморфизм $h$ непрерывен, причем $\|h\| = 1$.
\end{lemma}
\begin{proof}
	Мы знаем, что спектор элемента бахановой алгебры ограничен его нормой, значит для любого $x \in \CA$, для любого $\lambda \in \Cx$, $|\lambda|  > \|x\|$ и для любого комплесного гомоморфизма $h \in Z(\CA)$ в силу предыдущего следствия имеем
	$$
	h(x_\lambda) \neq 0.
	$$
	В силу линейности
	$$
	h(x) \neq h(e\lambda) = \lambda, \ |\lambda| > \|x\|
	$$
	в частности $\forall x \in \CA$, $\|x\| \leq 1$ и $\forall \lambda \in \Cx$: $|\lambda| > 1 \geq \|x\|$ имеем
	$$
	h(x) \neq \lambda \Rightarrow |h(x)| \leq 1
	$$
	Откуда $\|h\| \leq 1$. С учетом $h(e) = 1$, получаем требуемое.
\end{proof}
\begin{lemma}
	Для банаховой алгебры $\CA$ комплесного гомоморфизма $h \in Z(\CA)$ и элемента $x \in \CA$, справедливо 
	$$
	\{h(x) \mid h \in Z(\CA)\} \subset \sigma(x)
	$$
\end{lemma}
\begin{proof}
	Пусть $h(x) = \lambda$, тогда в силу линейности
	$$
	h(x_\lambda) = 0
	$$
	откуда в силу следсвия имеем $x_\lambda \notin G(\CA)$, откуда $\lambda \in \sigma(x)$. 
\end{proof}
Оказывается вложение в обратную сторону верно в случае коммутативной алгебры.
\begin{theorem}[Гельфанд]
	Пусть $\CA$ --- коммутативная банохава алгебра, то есть 
	$$
	\forall x,y \in \CA  \Rightarrow  xy = yx
	$$
	тогда $\forall x \in \CA$ выполнено 
	$$
	\sigma(x) = \{h(x) \mid h \in Z(\CA)\}
	$$
\end{theorem}
Чтобы доказать эту теорему понадобится дополнительная подготовка. 
\begin{definition}
	Подалгебра $I$ коммутативной банаховой алгебры $\CA$ называется идеалом, если 
	$$
	I\CA \subset I \ \CA I \subset I
	$$
	то есть $\forall i \in I, \ x \in \CA$: 
	$$
	xi, ix \in I
	$$
\end{definition}
\begin{definition}
	Идеал $I$ коммутативной баноховой алгебры $\CA$ называется максимальным, если $I \neq \CA$ и для любого идеала $J$, такого что $I\subsetneq J$, следует $J= \CA$.
\end{definition}
\begin{lemma}
	Если $h \in Z(\CA)$ в коммутативной банаховой алгебре $\CA$, то $\Ker h$ --- максимальный идеал в $\CA$.
\end{lemma}
\begin{proof}
	$h \neq 0$ поэтому $\Ker h \neq \CA$. Далее пусть $x \in \Ker h$, $y \in \CA$, тогда 
	$$
	h(xy) = h(x) h(y) = 0 \cdot h(y) = 0
	$$
	значит элемент $xy$ принадлежит $\Ker h$. Значит $\Ker h$--- идел. Теперь предположим, что существует элемент $x_0 \notin \Ker h$, тогда $h(x_0) \neq 0$. Для произвольного $x$ рассмотрим 
	$$
	y = x - \frac{h(x)}{x_0} x_0 \in \Ker h.
	$$
	Тогда 
	$$
	x = y + \frac{h(x)}{h(x_0)},
	$$
	в силу произвольности $x$ получаем 
	$$
	\CA = \Ker h \oplus \Lin \{x_0\}
	$$
	Таким образом, любое объемлющее подпространство совпадает с $\CA$, так как идеал это подпространство, то $\Ker h$ --- максимальный. 
\end{proof}
\begin{lemma}
	Любой максимальный идеал явялется замкнутым в $\CA$. 
\end{lemma}
\begin{proof}
	Пусть $M$ --- максимальный идеал. Рассмотрим
	$$
	N = [M]
	$$
	Тогда $N$ --- тоже идеал в $\CA$. Дейсвтительно любой $x \in N$ приближается последовательностью $x_n \in M$, то есть
	$$
	\|x - x_n\| \to 0.
	$$
	Пусть $y \in \CA$, тогда 
	$$
	\|yx - yx_n\| \leq \|y\| \|x - x_n\| \to 0
	$$
	но $y x_n$ лежат в $M$ так как $M$ --- идеал, а значит предельный элемент $yx$ лежит в $N$, что и требовалось. Теперь, так как $M$ --- максимальный идеал и
	$$
	M \subset [M] = N
	$$
	то либо $N = M$, либо $N = \CA$. Однако последнее произойти не может, так как любой собственный идеал не пересекается с открытым множеством $G(\CA)$, действительно если $ x \in G(\CA) \cap I$, то 
	$$
	e = x^{-1} x \in I
	$$
	откуда $I = \CA$. Таким образом $M = [M]$, что и требовалось.
\end{proof}
\begin{lemma}
	Любой собственный идеал содержится в некотором максимальном иделе.
\end{lemma}
\begin{proof}
	Получается стандартным рассуждением с рассмотрением множества всех идеалов содержащих данный и примерением теоремы Хаудорфа о максимальности (\ref{th:Hausdorf}). (проведите его!)
\end{proof}
\begin{lemma}
	Пусть $\CA$ --- коммутативная банахова алгебра и $N \subset \CA$ --- замкнутый идеал в $\CA$, тогда факторпространство $ \CA / N$ --- коммутативная банахова алгебра относительно произведения
	$$
	\pi(x) \pi(y) = xy + N 
	$$
	где $\pi: \CA \to \CA / N$ --- фактор-отображение.
\end{lemma}
\begin{proof}
	Фактор пространство банахова пространства по замкнутому подпространству относительно фактор нормы является банаховым пространством (утверждение \ref{cl:factorbanach}). Коммутативность очевидна, остается показать 
	$$
	\|\pi(x) \pi(y)\| \leq \|\pi(x)\| \|\pi(y)\|.
	$$
	Пусть $\xi, \eta \in N$, тогда заметим, что для любых $x,y \in \CA$ в силу того, что $N$ --- идеал иммем
	$$
	(x  + \xi) (y + \eta) = xy +\xi y +x \eta + \xi \eta \subset xy + N.
	$$
	Далее 
	\begin{align*}
		\|\pi(x)\| \|\pi(y)\| = \inf_{\xi \in N} \|x + \xi\| \inf_{\eta \in N} \|y + \eta\| = \inf\limits_{\substack{\xi \in N\\ \eta \in N }} \|x + \xi\|\|y + \eta\| \geq \inf\limits_{\substack{\xi \in N\\ \eta \in N }} \|(x + \xi)(y + \eta)\|  \geq \\ \geq \inf_{z \in N} \|xy + z\| = \|\pi(xy)\| = \|\pi(x)\pi(y)\|
	\end{align*}
\end{proof}

\begin{proof}[Доказательтво теоремы Гельфанда]
	Пусть $\CA$ --- коммутативная банахова алгебра. Покажем, что всякий максимальный идеал $M \subset \CA$ является ядром некоторого комплексного гомоморфизма $h \in Z(\CA)$. Для произвольного элемента $ x \in \CA\setminus M$ рассмотрим 
	$$
	N = \{ax + y \mid a \in \CA, \ y \in M\}
	$$ 
	тогда $N$ --- идеал в $\CA$. В самом деле, для любого $z \in \CA$, $a \in \CA$, $y \in M$ верно
	$$
	z(ax + y) = (za)x + zy \subset (za)x + M \subset N
	$$
	Далее, так как $x = ex + 0 \in N$, но $x \notin M$, то $x \in N \setminus M$. Поэтому $M \subsetneq N$. Но $M$ --- максимальный идеал, значит $N = \CA$. Поэтому $e \in N$. То есть для любого $x\in \CA\setminus M$ найдется $a_x \in \CA$ и $y_x \in M$, что 
	$$
	a_x x + y_x = e
	$$
	Рассмотрев факторалгебру $\CA / M$. Получаем, что $x \in \CA\setminus M$ тогда и только тогда когда $\pi(x) \neq M$. $\pi$ --- гомоморфизм, поэтому $\pi(e) = e_{\CA / M}$. Тогда 
	$$
	\pi(a_x)\pi(x) = \pi(a_x x) = \pi(e)
	$$
	Получили, что элемент $\pi(x)$ --- обратим в фактор алгбере. В силу произвольности $x$, получили что все ненулевые элементы в $\CA / M$ обратимы, тогда по теореме Гельфанда $\CA / M \cong \Cx$. Пусть $\varphi$ --- изомофизм, тогда легко понять, что искомый гомоморфизм $h$ это $h = \varphi \circ \pi$. 
	
	Теперь берем произвольный элемент $x \in \CA$ и рассматриваем 
	$$
	L = \{a x \mid a \in \CA\}
	$$
	Очевидно, что $L$ --- идеал. Если $x \in G(\CA)$, то $L = \CA$. Пусть теперь $x \notin G(\CA)$, тогда $L$ --- собственный идеал и потому сущесвтует максимальный идеал $M$ содержащий $L$. Тогда существует $h \in Z(\CA)$. Такой что $\Ker h = M$. В частности 
	$$
	h|_L = \{0\}
	$$
	И тогда $h(x) = 0$. Итак, в коммутативной баноховой алгебре для любого элемента $x \in \CA \setminus G(\CA)$ существует комплексный гомоморфизм $h$, такой что $x \in \Ker h$. Значит для любого элемента $x \in \CA$ и для любого $\lambda \in \sigma(x)$ так как по определению спектра $x_\lambda \in A \setminus G(\CA)$, то существует $h \in Z(\CA)$, что $h(x_\lambda) = 0$. То есть 
	$$
	h(x) = \lambda h(e) = \lambda 
	$$
	Значит $\sigma(x) \subset \{h(x) \mid h \in Z(\CA)\}$, и обратное вложение доказано. 
\end{proof}
