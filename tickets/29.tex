\newpage
\section{Теорема Банаха-Алаоглу о слабой* компактности поляры окрестности нуля топологического векторного пространства.}


\begin{theorem}[Банах-Алаоглу]\label{th:banach-alaoglu}
	Пусть $(X,\tau)$ --- топологическое векторное пространство и $ V\in \tau$ --- окрестность нуля. Рассмотрим
	$$
	\Gamma(V) = \{ f\in X^* \mid |f(x)| \leq 1  \ \forall x \in V\}
	$$
	$\Gamma(V)$ --- называется поляром окрестности. Тогда $\Gamma(V)$ --- $\tau_{w^*}$-компакт.
\end{theorem}
\begin{proof}
	Рассмотрим произвольный $x \in X$, $x\cdot 0 = 0 \in V$, тогда 
	$$
	\exists \delta_x > 0 \ \forall \lambda \in \Cx : \ |\lambda| < \delta_x \Rightarrow \lambda x \in V
	$$
	Возьмем $t_x = \frac{\delta_x}{2}$, тогда 
	$$
	t_x \cdot x \in V \Rightarrow \forall f \in \Gamma(V) \Rightarrow |f(t_x x)| \leq 1 \Rightarrow |f(x)| \leq \frac{1}{t_x}
	$$
	Обозначим $r_x = \frac{1}{t_x}$. Таким образом любой функционал на конкретном $x$ ограничен по модулю числом $r_x$. Теперь рассмотрим замкнутые круги в $\Cx$:
	$$
	K_r = \{\lambda \in \Cx \mid |\lambda| \leq r\}
	$$
	Для каждого $x$ рассмотрим такой круг и построим декартово произведение
	$$
	\bigtimes_{x\in X} K_{r_x} = \{g \colon X \to \Cx \mid \forall x \in X\colon  g(x) \in K_{r_x}\}
	$$
	Тогда в силу ограничений выше получаем 
	$$
	\Gamma(V) \subset 	\bigtimes_{x\in X} K_{r_x} 
	$$
	Наделим это декартово произведение топологией Тихонова. Так как для каждого $x$, $K_{r_x}$ является компактом, то по теореме Тихонова $	\bigtimes_{x\in X} K_{r_x} $ --- компакт в топологии Тихонова, но сужение топологии Тихонова на непрерывные линейные функции и есть $\tau_{w^*}$, тогда $	\bigtimes_{x\in X} K_{r_x}  \cap X^*$ является $\tau_{w^*}$-компактом, при этом $\Gamma(V)$ является его подмножеством, значит нам остается показать, что $\Gamma(V)$ --- замкнуто. 
	
	Пусть $ g \in [\Gamma(V)]_{\tau_T} \subset \Cx^X$. Берем 
	$$
	x_{1,2} \in X,  \quad V_T(g,x,\eps) = \{h \in \Cx^X \mid |g(x) - h(x)| < \eps\}
	$$
	Тогда рассмотрим 
	$$
	U(g) = V_T(g, x_1 + x_2, \eps) \cap V_T(g,x_1,\eps) \cap V_T(g,x_2,\eps) \in \tau_T
	$$
	Тогда 
	$$
	\exists f \in U(g) \cap \Gamma(V)
	$$
	Такой что 
	$$
	|f(x_1 + x_2) - g(x_1 + x_2)| = |f(x_1) + f(x_2) - g(x_1 + x_2)| < \eps, \quad|g(x_{1,2}) - f(x_{1,2})| < \eps
	$$
	Тогда пользуясь умным нулем получаем
	$$
	|g(x_1) + g(x_2) - g(x_1 + x_2)| < 3\eps
	$$
	Устремляя $\eps$ к нулю, получаем, что $g$ --- аддитивен. 
	
	Теперь пусть $x \in X, \lambda \in \Cx$, теперь возьмем окрестность 
	$$
	U(g) = V_T(g,\lambda x, \eps) \cap V_T(g,x,\eps)
	$$
	Аналогично аддитивности получаем $g(\lambda x) = \lambda g(x)$. 
	
	Теперь $g$ --- линеен, осталось показать непрерывность. Но $\forall x$ $g(x) \in K_{r_x}$, тогда 
	$$
	\forall x \in V \Rightarrow t_x = r_x = 1 \Rightarrow |g(x)| \leq 1
	$$
	Значит линейный функционал ограничен на окрестности нуля, что равносильно его непрерывности, значит $g \in X^*$, тогда $g \in \Gamma(V)$, то есть 
	$$
	[\Gamma(V)] = \Gamma(V)
	$$
	Таким образом замкнутость доказана, значит $\Gamma(V)$ --- компакт, что и требовалось.
\end{proof} 
\begin{remark}
	В случае нормированного пространства $\Gamma(O_1(0))$ является единичным шаром в сопряженном пространстве. 
\end{remark}