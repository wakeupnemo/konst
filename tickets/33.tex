\newpage
\section{Неметризуемость слабой топологии в бесконечномерном локально выпуклом топологическом векторном пространстве. Теорема о метризуемости слабой топологии на шаре линейного нормированного пространства.}
\begin{theorem}
	Слабая топология в бесконечномерном топологическом векторном пространстве неметризуема.
\end{theorem}
\begin{proof}
	Предположим, что $\tau_w$ на бесконечномерном $X$ метризуема и $\rho_*$ --- метрика. Тогда рассмотрим систему вложенных шаров по данной метрике
	$$
	\left\{O_{\frac{1}{n}}^{\rho_*}(0)\right\}_{n=1}^\infty
	$$ 
	Это семейство является счетной локальной базой точки $0$. По предположению $\tau_w  = \tau_{\rho_*}$, тогда для любого $n$ найдется элемент базы, который содержится в $O_{\frac{1}{n}}^{\rho_*}$, то есть найдутся такие $f_1^{(n)}, \dots, f_N^{(n)} \in X^*$ и $\eps^{(n)}> 0$, что 
	$$
	\bigcap_{k=1}^N V(0,f_k^{(n)},\eps) \subset O_{\frac{1}{n}^{\rho_*}}(0)
	$$
	Для каждого $n$ рассмотрим
	$$
	\Phi_n = \Lin \{f_1^{(n)}, \dots, f_N^{(n)}\}
	$$
	Покажем, что объединение этих конечномерных пространств даст все пространство $X^*$, что приведет нас к противоречию. Пусть $f \in X^*$, произволен, в силу непрерывности существует окрестность нуля $U(0) \in \tau_w$, на который функционал $f$ ограничен, можно считать, что 
	$$
	\forall x \in U(0)  \ |f(x)| < 1
	$$
	Но так как система вложенных шаров является локальной базой нуля, найдется номер $n$, что выполнено вложение
	$$
	U(0) \supset O_{\frac{1}{n}}^{\rho_*} \supset \bigcap_{k=1}^{N^{(n)}} V(0, f_k^{(n)}, \eps^{(n)})
	$$
	Рассмотрим произвольный $x$ из пересечения ядер $\bigcap_{k=1}^{N^{(n)}} \Ker f_k^{(n)}$, тогда 
	$$
	\forall k \in \overline{1, N} \ |f_k^{(n)}(x)| = 0 < \eps^{(n)} 
	$$
	что означает вложение в $\bigcap_{k=1}^{N^{(n)}} V(0, f_k^{(n)}, \eps^{(n)})$ а значит и в  $U(0)$, значит 
	$$
	\bigcap_{k=1}^{N^{(n)}} \Ker f_k^{(n)}	\subset U(0)
	$$ 
	Пересечение ядер функционалов имеет конечную коразмерность, значит в силу бесконечномерности пространства $X$, подпространство $	\bigcap_{k=1}^{N^{(n)}} \Ker f_k^{(n)}$ не пусто, тогда $\forall x \in 	\bigcap_{k=1}^{N^{(n)}} \Ker f_k^{(n)}$ $\forall p \in \N$, $px \in 	\bigcap_{k=1}^{N^{(n)}} \Ker f_k^{(n)}$, значит в силу свойства окрестности $U(0)$: 
	$$
	|f(px)| = p|f(x)| < 1 \Rightarrow |f(x)| < \frac{1}{p}
	$$
	что верно для любого $p \in \N$. Таким образом $f(x) = 0$, то есть выполнено вложение ядер 
	$$
		\bigcap_{k=1}^{N^{(n)}} \Ker f_k^{(n)} \subset \Ker f
	$$
	Рассмотрим конечномерное линейное пространство 
	$$
		L = \left\{\begin{pmatrix}
		f_1^{(n)}(x) \\
		\vdots \\
		f_N^{(n)}(x)
	\end{pmatrix} \mid x \in X\right\} \subset \Cx^{N^{(n)}}
	$$
	И рассмотрим линейный функционал действующий из этого пространства 
	$$
	\Lambda \colon L \to \Cx \quad \Lambda(x) = f(x)
	$$
	Если $f_k^{(n)}(x) = f_k^{(n)}(z)$, то 
	$$
	f_k^{(n)}(x -z) = 0 \Rightarrow x - z \in \Ker f_k^{(n)} \subset \Ker f 
	$$
	Значит функционал $\Lambda$ определен корректно. Но $L$ --- конечномерно, тогда из линейной алгебры 
	$$
	\forall x \in X \ \Lambda = \sum_{k=1}^{N^{(n)}} \alpha_k f_k^{(n)}(x) = f(x)
	$$
	Значит $f \in \Phi_n$.
	Таким образом для произвольного функционала $f \in X^*$ найдется номер $n$, что $f$ будет лежать в $\Phi_n$. То есть 
	$$
	X^* = \bigcup_{n=1}^\infty \Phi_n
	$$
	Но $\Phi_n$ --- конечномерно, а значит замкнуто в бесконечномерном векторном топологическом пространстве $X$, тогда внутренность пуста и
	$$
	\Int [\Phi_n]_{\tau_w} = \varnothing
	$$
	Однако пространство $X^*$ полно как пространство линейных непрерывных функционалов в полное пространство $\Cx$, противоречие с теоремой Бэра. 
\end{proof}
\begin{theorem}
	Пусть $X$ --- линейное нормированное пространство и $X^*$ --- сепарабельно. Тогда топологическое пространство $(B_R(0), \tau_w(R))$, где $\tau_w(R)$ --- слабая топология индуцированная на шар $B_R(0)$ --- метрическое.
\end{theorem}
\begin{proof}
	утверждение 5.4.29. в Lec-Funkan 
	
	
	Пусть $X^*$ --- сепарабельно, тогда рассмотрим счетное всюду плотное на 1-сфере множество $\{f_k\}_{k=1}^\infty$. Рассмотрим метрику 
	$$
	\rho(x,y) = \sum_{k=1}^{\infty} \frac{|f_k(x) - f_k(y)|}{2^k}
	$$
	Можно проверить, что это действительно метрика (проверьте!). Обозначим предбазу индуциорованной топологии 
	$$
	\sigma_w(R) = \{V(x,f,\eps)  \cap B_R(0) \mid x \in X, f \in X^*, \eps > 0\}
	$$
	Покажем, что $V(x,f,\eps) \cap B_R(0) \subset O_r(0) \cap B_R(0)$ для какого-то $r > 0$. Рассмотрим произвольный ненулевой функционал $f \in X^*$, произвольный вектор $x \in X$ и число $\eps > 0$. Если $f = 0$, то все тривиально, если же $f \neq 0$, то определим $g = \frac{f}{\|f\|}$ и число $\delta = \frac{\eps}{\|f\|}$. Тогда получим
	$$
	V(x,f,\eps) = V(x,g,\delta) 
	$$
	Рассмотрим произвольный вектор 
	$$
	y \in V(x,g,\delta) \cap B_R(0), \ \text{ т.е. } |g(y-x)| < \delta \text{ и } \|y\| \leq R
	$$
	Найдется $m$, такой, что
	$$
	\|g - f_m\| < \frac{\delta - |g(y-x)|}{4R}
	$$
	(выбор этого числа --- чистой воды подгон под дальнейшие неравенства). Пусть число 
	$$
	r = \frac{\delta - |g(y-x)|}{2^{m+1}} > 0
	$$
	Рассмотрим произвольный вектор $z \in B_R(0)$ вида $\rho(y,z) < r$. Тогда
	$$
	|f_m(z-y)| < 2^m r = \frac{\delta - |g(y-x)|}{2}.
	$$

	Следовательно, получаем 
	\begin{gather*}
	|g(z- x)| \leq |g(z - y)| + |g(y -x )| \leq \\ \leq  \|(g - f_m)(z - y)| + |f_m(z - y)| + |g(y-x)| <\\ <  \|g - f_m\| 2R + \frac{\delta + |g(y - x)|}{2} < \frac{\delta - |g(y-x)|}{2} + \frac{\delta + |g(y-x)|}{2} = \delta
	\end{gather*}
	т.е. выполнено вложение 
	$$
	z \in V(x,g,\delta) \cap B_R(0) = V(x,f,\eps) \cap B_R(0)
	$$
	Следовательно, любой вектор $y$ множества $V(x,f,\eps) \cap B_R(0)$ является его $\rho$-внутренней точкой, то есть $V(x,f,\eps) \cap B_R(0) \in \tau_\rho(R)$.
	
	Покажем обратное вложение. Так как базой $\beta_\rho(R)$ метрической топологии служат шары вида
	$$
	O_r^\rho(x) = \{y \in B_R(0) \mid \rho(x,y) < r\}
	$$
	то достаточно вложить $$\beta_\rho(R) \subset \tau_w(R).$$ Зафиксируем вектор $x \in B_R(0)$ и число $r > 0$. Рассмотрим вектор 
	$$
	y\in O_r^\rho(x)
	$$
	Существует $N$, такой, что
	$$
	2^{-N} < \frac{r - \rho(x,y)}{4R}
	$$
	Возьмем $\delta = \frac{r - \rho(x,y)}{2} > 0$. Рассмотрим произвольный вектор  
	$$
	z \in \left(\bigcap_{n=1}^N V(y,f_n,\delta)\right) \cap B_R(0) = U(u) \in \tau_w(R)
	$$
	Тогда получаем
	\begin{gather*}
		\rho(x,z) \leq \rho(y,z) + \rho(x,y) \leq \\ \leq \sum_{n=1}^N 2^{-n} |f_n(y-z)| + \sum_{n = N+1}^\infty 2^{-n} 2R + \rho(x,y) < \\ < \delta + 2^{-N} 2R + \rho(x,y) < \frac{r - \rho(x,y)}{2} + \frac{r - \rho(x,y)}{2} + \rho(x,y) = r
	\end{gather*}
	Таким обрзаом справделиво вложение $U(y) \subset O_r^\rho(x)$. Что и требовалось.
\end{proof}
