\newpage
\section{Теорема Эберлейна-Шмульяна о слабой секвенциальной компактности слабого компакта в нормированном пространстве.}


\begin{theorem}[Эберлейн-Шмульян]
	Если $(X,\|\|)$ --- ЛНП и $K \subset X$--- слабый компакт, тогда $K$ --- сильно ограничен и слабый секвенциальный компакт.
\end{theorem}
\begin{proof}
	\begin{itemize}
		\item $\forall f \in X^* \Rightarrow f(K)$ --- компакт в $\Cx$ тогда $f(K)$ ограниченно в $\Cx$, тогда рассмотрев 
		$$
		M = \{Fx \mid x \in K\} \subset X^{**}
		$$
		Тогда $\forall f \in X^*\colon \exists R_f > 0 \Rightarrow |\Phi(f)| = |f(x)| \leq R_f$
		Тогда по теореме Банаха-Штейнгауза, $ \exists R > 0 \colon \forall x \in K: \ \|\Phi\| = \|Fx\| = \|x\| \leq R$, значит $K$ --- сильно ограничен.
		\item $\forall \{x_n\} \subset K$. Без ограничения общности $x_n \neq x_m$. Тогда рассматриваем 
		$$
		L = [\Lin \{x_n\}_{n=1}^\infty]_{\|\|}
		$$
		По теореме Мазура $L$ --- слабо замкнуто. Кроме того рассмотрим 
		$$
		M = \left\{\sum_{k=1}^N \alpha_k x_k \mid \alpha_k \in \mathbb{Q}^2 \subset \Cx^2, N \in \N \ \right\}
		$$
		Ясно, что $M$ --- счетное, всюду плотное в $L$ множество, значит $L$ --- сепарабельное пространство. Тогда слабая* топология метризуема на $B_1^*(0) \subset L^*$. То есть $(B_1^*(0), \rho_*)$ --- метрическое пространство, а значит сепарабельное. Обозначим 
		$$
		\{f_s\}_{s=1}^\infty \subset B_1^*(0)
		$$
		Счетное всюду плотное в $B_1^*(0)$ множество. Заметим, что 
		$$
		|f_s(x_n)| \leq \|f_s\| \|x_n\| \leq R
		$$
		То есть $\forall s \in \N$ последовательность $\{f_s(x_n)\}_{n=1}^\infty$ ограниченна в $\Cx$. Тогда применяя канторов диагональный процесс выделим подпоследовательность $\{x_{n_k}\}_{k=1}^\infty$. Такую что 
		$$
		\forall s \in \N \Rightarrow \exists \lim\limits_{k\to\infty} f_s(x_{n_k}) \in \Cx
		$$
		Вспомним, что в топологическом компакте любое бесконечное множество имеет предельную точку, то есть такую точку, в любой окрестности которой лежит бесконечное число элементов множества. В частности в слабом компакте $K$ множество $\{x_{n_k}\}_{k=1}^\infty$ имеет предельную точку $x$ и 
		$$
		\forall U(x) \in \tau_w \Rightarrow \exists k \in \N: \ x_{n_k} \in U(x) \text{ и } x_{n_k} \neq x
		$$
		Так как в полном $\Cx$ для каждого $s \in \N$ существует предел $\lim\limits_{k\to\infty}f_s(x_{n_k})$, то 
		$$
		\forall \eps > 0 \ \exists N_s(\eps): \ \forall k, r \geq N_s(\eps) \Rightarrow |f_s(x_{n_k} - f_s(x_{n_r}))| \leq \eps 
		$$
		Выберем слабую окрестность 
		$$
		U(y) = V(y,f_s,\eps) \setminus \{x_{n_k}\}_{k=1}^{N_s(\eps)}
		$$
		$U(y)$ открыта, так как одноточечные множества в векторной топологии $\tau_w$ замкнуты. Так как $y$ --- предельная точка, то 
		$$
		\exists k > N_s(\eps) \Rightarrow x_{n_k} \in U(y)
		$$ 
		То есть 
		$$
		|f_s(y) - f_s(x_{n_k})| \leq \eps
		$$
		Тогда $\forall r > N_s(\eps)$
		$$
		|f_s(x) - f_s(x_{n_r}| \leq |f_s(y) - f_s(x_{n_s})| + |f_s(x_{n_s}) - f_s(x_{n_r})| \leq 2\eps 
		$$
		То есть мы получили $\forall s \in \N \Rightarrow \exists \lim\limits_{k\to\infty} f_s(x_{n_k}) = f_s(x)$
		\item  Пусть теперь $g \in X^*$. Покажем, что $g(x_{n_k}) \to g(x)$. Заметим, что для $h = g\big |_ L\colon L \to \Cx $ выполнено $h \in L^*$. Предположим, что 
		$$
		h(x_{n_k}) \nrightarrow h(x)
		$$
		Тогда существует подпоследовательность $\{x_{n_{k_p}}\}_{p=1}^\infty$ и $\eps_0 > 0$, что  
		$$
		\forall s \in \N: \ |h(x_{n_{k_p}}) - h(x)| \geq \eps_0 
		$$
		Но $\{x_{n_{k_p}}\}_{p=1}^\infty$ бесконечное множество, а значит имеет предельную точку $z$ в $K$, тогда, повторяя рассуждения выше, получим
		$$
		\forall s \in \N \Rightarrow \exists \lim\limits_{p \to \infty} f_s(x_{n_{k_p}}) = f_s(z)
		$$
		Но $\lim\limits_{p \to \infty} f_s(x_{n_{k_p}}) = f_s(x)$, значит для любого $s \in \N$ $f_s(z) = f_s(x)$. Вспоминаем, что $\{f_s\}_{s=1}^\infty$ --- всюду плотное множество в $B_1^*(0)$, значит 
		$$
		h(y) = h(z)
		$$ 
		Но $z$ --- слабая предельная точка для $\{x_{n_{k_p}}\}_{p=1}^\infty$. В частности для окрестности 
		$$
		U(z) = V(z,h,\eps_0)
		$$
		Тогда 
		$$
		\exists x_{n_{k_{p_0}}}\Rightarrow |h(x_{n_{k_{p_0}}}) - h(z)| = |h(x_{n_{k_{p_0}}}) - h(y)| < \eps_0
		$$
		Но 
		$$
		|h(x_{n_{k_{p_0}}}) - h(y)| \geq \eps_0
		$$
		противоречие, значит $h(x_{n_k}) \to h(x)$. Тогда $\forall g \in X^*$: 
		$$
		g(x_{n_k}) \to g(x) \text{ т.к. } h = g\big |_L  \in L^*
		$$
		Окончательно, получили подпоследовательность $\{x_{n_k}\}$ слабо сходящуюся к $x \in K$, что и требовалось.  
	\end{itemize}
	
\end{proof}