\newpage
\section{Топологические пространства, база и предбаза топологии. Критерий базы и предбазы топологии}

\begin{definition}
	Пусть $X$ --- непустое множество, $\tau \subset 2^X$ --- семейство его подмножеств. Тогда пара $(X, \tau)$ называется топологическим пространством, если:
	\begin{itemize}
		\item $X, \varnothing \in \tau$
		\item $\forall \{U_\alpha\}_{\alpha \in \mathbb{A}} \subset \tau \Rightarrow \bigcup\limits_{\alpha \in \bb{A}}U_\alpha \in \tau$
		\item $\forall U, V \in \tau \Rightarrow U \cap V \in \tau$
	\end{itemize}
При этом $\tau$ --- называется топологией в $X$, а элементы $\tau$ --- $\tau$-открытыми множествами.
\end{definition} 

\begin{definition}
	Пусть $X \neq \varnothing$ и $\beta \subset 2^X$ --- семейство подмножеств $X$. Тогда говорят, что $\beta$ --- база топологии в $X$, если 
	$$
	\tau = \left\{\bigcup\limits_{G \in M }G \mid M \subset \beta \right\} \text{ --- топология в } X
	$$
\end{definition}

\begin{theorem}[Критерий базы]\label{th:base_criterian}
	Пусть $X \neq \varnothing$ и $\beta \subset 2^X$, тогда $\beta$ --- база топологии в $X$ iff 
	\begin{itemize}
		\item $\beta$ --- покрытие $X$ 
		\item $\forall G_1, G_2 \in \beta \Rightarrow \forall x \in G_1 \cap G_2 \ \exists G \in \beta \Rightarrow x \in G \subset  G_1 \cap G_2$
	\end{itemize}
\end{theorem}

\begin{proof}
Пусть $\beta$ --- база, тогда 
		$$
		\tau = \left\{\bigcup\limits_{G \in M }G \mid M \subset \beta \right\} \text{ --- топология в } X
		$$
		$X \in \tau$, тогда $X = \bigcup\limits_{G \in \beta }G$, то есть $\beta$ --- покрытие $X$. Пусть $G_1, G_2 \in \beta$, тогда $G_1, G_2 \in \tau$, тогда 
		$$
		G_1 \cap G_2 \in \tau \Rightarrow \exists M \subset \beta \ G_1 \cap G_2 = \bigcup\limits_{G \in M } G
		$$
		Тогда $\forall x \in G_1 \cap G_2 \ \exists G \in M: \ x \in G \subset  G_1 \cap G_2$
		
		Обратно. Пусть выполнены условия критерия, покажем, что 
		$$
		\tau = \left\{\bigcup\limits_{G \in M }G \mid M \subset \beta \right\} 
		$$
		Является топологией в $X$.
		\begin{itemize}
			\item $X \in \tau$ так как $\beta$ --- покрытие $X$.
			\item $\varnothing \in \tau$ как пустое объединение.
			\item $\{U_\alpha\}_{\alpha \in \bb{A}} \subset \tau$, тогда для каждого $U_\alpha$ существует $M_\alpha \subset \beta$:
			$$
			\bigcup\limits_{\alpha \in \bb{A}}U_\alpha = \bigcup\limits_{\alpha \in \bb{A}} \bigcup\limits_{G \in M_\alpha}G = \bigcup\limits_{G \in M^* }G \in \tau
			$$
			Где $M^* = \bigcup\limits_{\alpha \in \bb{A}}M_\alpha \subset \beta$
			\item $\forall U,V \in \tau$, тогда используя второе условие критерия для каждого $x$ из пересечения имеем элемент базы из пересечения: $G_x \in \beta$. Тогда:
			$$
			U \cap V = \bigcup\limits_{x \in U \cap V}G_x \in \tau
			$$
		\end{itemize}
\end{proof}

\begin{definition}
	\label{subbase}
	Пусть $X \neq \varnothing$ и $\sigma \subset 2^X$. $\sigma$ --- называется предбазой топологии в $X$, если:
	$$
	\beta = \left\{\bigcap\limits_{k=1}^NV_k \mid N \in \bb{N}, V_k \in \sigma \right\} \text{ --- база топологии}
	$$
\end{definition}

\begin{theorem}[Критерий предбазы]
	\label{th:csb}
	Пусть $X \neq \varnothing$ и $\sigma \subset 2^X$, тогда $\sigma$ --- предбаза топологии в $X$ iff $\sigma$ --- покрытие $X$.
\end{theorem}
\begin{proof}
	Если $\beta= \left\{\bigcap\limits_{k=1}^NV_k \mid N \in \bb{N}, V_k \in \sigma \right\}$ --- база, то $\sigma$ --- покрытие $X$, так как 
	$$
	X = \bigcup\limits_{G \in \beta } G \subset \bigcup\limits_{G \in \sigma}G \subset X
	$$
	Обратно, воспользуемся критерием базы. $\beta$ --- покрытие так как $\sigma \subset \beta$ и $\sigma$ --- покрытие. Второе условие выполнено автоматически, так как:
	$$
	\forall U,V \in \beta \Rightarrow U \cap V = \bigcap\limits_{k=1}^{N_1}V_k \cap \bigcap\limits_{k=1}^{N_2} U_k = \bigcap\limits_{k=1}^{N^*}V_k \in \beta
	$$
	Таким образом $\beta$ --- база.
\end{proof}