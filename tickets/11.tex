\newpage
\section{Факторпространство и фактортопология. Теорема о замкнутости суммы замкнутого и конечномерного подпространств топологического векторного пространства.}

Пусть $(X, \tau)$ --- топологическое векторное пространство, $N \subset X$ --- подпространство в $X$, тогда можно ввести отношение: 
$$
x, y \in X: \ x \stackrel{N}{\sim} y \Leftrightarrow x - y \in N
$$
Это отношение является отношением эквивалентности. 
$\forall x \in X$ рассмотрим классы эквивалентности
$$
\pi_N(x) = \{ y \in X \mid y \stackrel{N}{\sim} x\} = x + N
$$
\begin{definition}\label{def:fac}
	Совокупность всех классов эквивалентности $X / \{\pi_N(x) \mid x \in X\}$ --- называется фактор пространством.
\end{definition}
В этом пространстве вводятся естественные линейные операции:
$$
\pi_N(x) + \pi_N(y) := \pi_N(x+y), \ \alpha \pi_N(x) = \pi_N(\alpha x)
$$
Аксиомы линейного пространства выполнены. Таким образом $X / N$ --- линейное пространство. Теперь введем топологию.
\begin{definition}
	$\pi_N: X \rr X / N$ --- называется фактор-отображением. Фактор-топологией назовем
	$$
	\tau_{X / N} = \{ W \subset X / N \mid \pi_N^{-1}(W) \in \tau\}
	$$
\end{definition}
Можно проверить, что это действительно топология. Заметим, что определение фактор-топологии можно интерпретировать как слабейшую топологию обеспечивающую непрерывность фактор-отображению. Более интересным представляется вопрос, является ли эта топология векторной?

Заметим, что не только прообраз открытых множеств под действием $\pi_N$ будет открыт, но и образ, действительно:
$$
\forall G \in \tau  \Rightarrow \pi_N^{-1}(\pi_N(G)) = G + N = \bigcup_{y \in N }(G + y) \in \tau
$$
Это свойство называется открытостью отображения $\pi_N$

\begin{claim}
	$(X / N , \tau_{X / N})$ удовлетворяет первой аксиоме отделимости если и только если $N \subset X$ --- $\tau$-замкнуто 
\end{claim}
\begin{proof}
	Зафиксируем $x \in X$, точкой в фактор пространстве является $\{\pi_N(x)\}$, она является замкнутым множеством тогда и только тогда, когда
	$$
	(X/ N) \setminus \{\pi_N(x)\} \in \tau_{X / N}
	$$
	По определению векторной топологии нам нужно проверить является ли открытым прообраз этого множества в $(X, \tau)$:
	$$
	\pi_N^{-1}\left((X/ N) \setminus \{\pi_N(x)\}\right) = X \setminus \pi_N^{-1}(\pi_N(x)) = X \setminus (x + N)
	$$
	Ясно, что $X \setminus (x + N) \in \tau \Leftrightarrow x + N$ --- $\tau$-замкнуто, что, в силу непрерывности трансляции равносильно $N$ --- $\tau$-замкнуто.
\end{proof}
\begin{claim}
	Если $(X, \tau)$ --- топологическое векторное пространство, $N \subset X$ --- $\tau$-замкнуто, то $(X / N , \tau_{X / N})$ является топологическим векторным пространством.
\end{claim}
\begin{proof}
	\begin{itemize}
		\item Первая аксиома отделимости выполнена в силу предыдущего утверждения.
		\item  Проверим непрерывность сложения.$$
		\forall x,y \in X \ \forall U(\pi_N(x) + \pi_N(y)) \in \tau_{X / N}
		$$
		В силу линейности $U(\pi_N(x) + \pi_N(y)) = U(\pi_N(x+y))$. По определению окрестности в $\tau_{X / N}$:
		$$
		V : = \pi_N^{-1}(U(\pi_N(x+y))) \in \tau, \ x + y \in V
		$$ 
		Тогда так как $(X,\tau)$ --- векторная топология, то $\exists V_1(x), V_2(y) \in \tau$: 
		$$
		V_1(x) + V_2(y) \subset V = \pi_N^{-1}(U(\pi_N(x+y))) \Leftrightarrow \pi_N(V_1(x)) + \pi_N(V_2(y))\subset U(\pi_N(x) + \pi_N(y))
		$$
		в силу открытости $\pi_N$, $\pi_N(V_1(x)), \pi_N(V_2(y)) \in \tau_{X / N}$, но тогда и сумма открыта, таким образом сложение непрерывно.
		\item Проверим непрерывность умножения. 
		$$
		\forall x \in X, \forall \lambda \in \R \Rightarrow \lambda \pi_N(x) = \pi_N(\lambda x) \in X / N
		$$
		Рассмотрим произвольную окрестность $U(\lambda \pi_N(x)) = U( \pi_N(\lambda x)) \in \tau_{X / N}$, тогда положим
		$$
		V := \pi_N^{-1} U( \pi_N(\lambda x))  \ni \lambda x.
		$$
		В силу того, что $(X, \tau)$ --- топологическое векторное пространство, получаем 
		$$
		\exists \delta > 0 \ \exists W(x) \in \tau: \ \forall \alpha: \ |\lambda - \alpha| < \delta \Rightarrow \alpha W(x) \subset V
		$$
		Тогда 
		$$
		\pi_N(\alpha W(x)) \subset U(\pi_N(\lambda x))
		$$
		Образ открытого множество открыт, тогда мы нашли нужную окрестность $\pi_N(x)$, таким образом умножение на скаляр непрерывно.
	\end{itemize}
\end{proof}
\begin{remark}
	Отметим, что для непрерывности операций замкнутость подпространства $N$ не потребовалась.
\end{remark}

\begin{theorem}[О замкнутости суммы конечномерного и замкнутого подпространства из ТВП]\label{th:tvp_sum}
	Пусть $(X, \tau)$ --- топологическое векторное пространство, $L \subset X$ --- подпространство $\dim L = m < \infty$, $N \subset X$ --- $\tau$-замкнутое подпространство. Тогда $L + N$ --- $\tau$-замкнуто.
\end{theorem}
\begin{proof}
	Рассмотрим $(X / N, \tau_{X / N})$ --- является ТВП в силу того, что $N$ --- $\tau$-замкнуто. $L = Lin\{e_1, \dots , e_m\}$ --- базис в $L$. Тогда 
	$$
	\pi_N(L) = \Lin \{\pi_N(e_1), \dots, \pi_N(e_m)\} \text{ является конечномерным}
	$$
	Тогда в силу теоремы (\ref{th:clfinds}) $\pi_N(L)$ --- $\tau_{X / N}$-замкнуто. Тогда в силу топологической непрерывности $\pi_N$: $\pi_N^{-1}(\pi_N(L))$ --- является $\tau$-замкнутым в $(X, \tau)$, но 
	$$
	\pi_N^{-1}(\pi_N(L)) = L + N
	$$
	Что и требовалось. Все!
\end{proof}